\documentclass[accentcolor=tud1a,11pt]{tudexercise}
\usepackage{template}
\begin{document}
\nutzerstudie
\section*{Anlegen von Clients, Programmen und Dateien}
\begin{itemize}
	\item Legen Sie einen Client mit dem Namen \textbf{Test1} an. Die IP-Adresse soll \textbf{127.0.0.1} sein und die MAC-Adresse{12:34:56:78:ab:cd}.
	\item Fügen Sie diesem Client ein Programm mit dem Namen \textbf{Calculator} hinzu. Der Path soll dabei \textbf{gnome-calculator} sein, Argumente sollen nicht hinzugefügt werden und die Startzeit soll bei -1 bleiben. 
	\item Fügen Sie nun die Datei mit dem Namen \textbf{Testfile} ein, diese soll als SourcePath \textbf{/tmp/testfile} und als DestinationPath \textbf{/home/testfile} besitzen.
	\item Nun sollen Sie das Programm \textbf{Calculator} starten. Daraufhin soll sich ein Taschenrechner öffnen den Sie wieder schließen.
\end{itemize}
\section*{Erstellen und Ausführen von Skripten}
\begin{itemize}
	\item Erstellen Sie ein neues Skript mit dem Namen \textbf{Hallo Welt}.
	\item Fügen Sie das Programm Calculator zu diesem Skript hinzu. 
	\item Speichern Sie das Skript und führen es aus.
	\item Überprüfen Sie, ob sich der Taschenrechner geöffnet hat, schließen Sie ihn.
	\item Prüfen Sie nun, ob das der Status des Skripts auf \textbf{Done} steht.
\end{itemize}
\section*{Ändern und Löschen von Clients und Programmen}
\begin{itemize}
	\item Ändern Sie den Namen des Clients \textbf{Test2} auf \textbf{Vision}.
	\item Löschen Sie den Client \textbf{Test3}.
	\item Ändern Sie den Namen des Programms \textbf{Firefox} des Clients \textbf{Vision} zu \textbf{Google-Chrome} und den Pfad von \textbf{/bin/firefox} zu \textbf{google-chrome}.
\end{itemize}
\end{document}

