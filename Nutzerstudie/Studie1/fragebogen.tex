\documentclass[accentcolor=tud1a,11pt,colorbacktitle,nochapname]{tudexercise}
\RequirePackage[utf8]{inputenc}
\usepackage{wasysym, enumitem, color, forloop, ifthen}

\title{Nutzerstudie zum Webinterface - Fragebogen}
\subtitle{BP-Gruppe 19}


%% \Qq = Questionaire question. Oh, this is just too simple. It helps
%% making it easy to globally change the appearance of questions.
\newcommand{\Qq}[1]{\textbf{#1}}

%% \QO = Circle or box to be ticked. Used both by direct call and by
%% \Qrating and \Qlist.
\newcommand{\QO}{$\Box$}% or: $\ocircle$

%% \Qrating = Automatically create a rating scale with NUM steps, like
%% this: 0--0--0--0--0.
\newcounter{qr}
\newcommand{\Qrating}[1]{\QO\forloop{qr}{1}{\value{qr} < #1}{---\QO}}

%% \Qline = Again, this is very simple. It helps setting the line
%% thickness globally. Used both by direct call and by \Qlines.
\newcommand{\Qline}[1]{\noindent\rule{#1}{0.6pt}}

%% \Qlines = Insert NUM lines with width=\linewith. You can change the
%% \vskip value to adjust the spacing.
\newcounter{ql}
\newcommand{\Qlines}[1]{\forloop{ql}{0}{\value{ql}<#1}{\vskip0em\Qline{\linewidth}}}

%% \Qlist = This is an environment very similar to itemize but with
%% \QO in front of each list item. Useful for classical multiple
%% choice. Change leftmargin and topsep accourding to your taste.
\newenvironment{Qlist}{%
\renewcommand{\labelitemi}{\QO}
\begin{itemize}[leftmargin=1.5em,topsep=-.5em]
}{%
\end{itemize}
}

%% \Qtab = A "tabulator simulation". The first argument is the
%% distance from the left margin. The second argument is content which
%% is indented within the current row.
\newlength{\qt}
\newcommand{\Qtab}[2]{
\setlength{\qt}{\linewidth}
\addtolength{\qt}{-#1}
\hfill\parbox[t]{\qt}{\raggedright #2}
}

%% \Qitem = Item with automatic numbering. The first optional argument
%% can be used to create sub-items like 2a, 2b, 2c, ... The item
%% number is increased if the first argument is omitted or equals 'a'.
%% You will have to adjust this if you prefer a different numbering
%% scheme. Adjust topsep and leftmargin as needed.
\newcounter{itemnummer}
\newcommand{\Qitem}[2][]{% #1 optional, #2 notwendig
\ifthenelse{\equal{#1}{}}{\stepcounter{itemnummer}}{}
\ifthenelse{\equal{#1}{a}}{\stepcounter{itemnummer}}{}
\begin{enumerate}[topsep=2pt,leftmargin=2.8em]
\item[\textbf{\arabic{itemnummer}#1.}] #2
\end{enumerate}
}

%% \QItem = Like \Qitem but with alternating background color. This
%% might be error prone as I hard-coded some lengths (-5.25pt and
%% -3pt)! I do not yet understand why I need them.
\definecolor{bgodd}{rgb}{0.8,0.8,0.8}
\definecolor{bgeven}{rgb}{0.9,0.9,0.9}
\newcounter{itemoddeven}
\newlength{\gb}
\newcommand{\QItem}[2][]{% #1 optional, #2 notwendig
\setlength{\gb}{\linewidth}
\addtolength{\gb}{-5.25pt}
\ifthenelse{\equal{\value{itemoddeven}}{0}}{%
\noindent\colorbox{bgeven}{\hskip-3pt\begin{minipage}{\gb}\Qitem[#1]{#2}\end{minipage}}%
\stepcounter{itemoddeven}%
}{%
\noindent\colorbox{bgodd}{\hskip-3pt\begin{minipage}{\gb}\Qitem[#1]{#2}\end{minipage}}%
\setcounter{itemoddeven}{0}%
}
}

\begin{document}
\maketitle

\section{Fragen zu den Aufgaben}
\Qitem[a]{ \Qq{Waren die Aufgaben machbar?} \Qtab{3cm}{schwierig \Qrating{5} einfach}}

\section{Bedienbarkeit}
\Qitem[a]{ \Qq{Wie fanden Sie das Anlegen von Clients?} \Qtab{3cm}{schwierig \Qrating{5} einfach}}
\Qitem[b]{Was könnte man verbessern?}
\framebox(\textwidth,80){}
\Qitem[c]{ \Qq{Wie fanden Sie das Anlegen von Programs?} \Qtab{3cm}{schwierig \Qrating{5} einfach}}
\Qitem[d]{Was könnte man verbessern?}
\framebox(\textwidth,80){}
\Qitem[e]{ \Qq{Wie fanden Sie das Anlegen von Files?} \Qtab{3cm}{schwierig \Qrating{5} einfach}}
\Qitem[f]{Was könnte man verbessern?}
\framebox(\textwidth,80){}
\section{Design}
\Qitem[a]{ \Qq{War alles leicht auffindbar?} \Qtab{3cm}{eher nein \Qrating{5} eher ja}}
\newpage
\Qitem[b]{Was war nicht leicht auffindbar?}
\framebox(\textwidth,80){}
\Qitem[c]{ \Qq{Wie fanden Sie die Farbwahl?} \Qtab{3cm}{sehr schlecht \Qrating{5} sehr gut}}
\Qitem[d]{Würden Sie Farben ändern? Wenn ja, welche?}
\framebox(\textwidth,80){}
\Qitem[e]{ \Qq{Wie fanden Sie das Layout?} \Qtab{3cm}{sehr schlecht \Qrating{5} sehr gut}}
\Qitem[f]{Was würden Sie am Layout ändern?}
\framebox(\textwidth,80){}
\section{Abschließennd}
\Qitem[a]{Was fanden Sie gut?}
\framebox(\textwidth,80){}
\Qitem[b]{Was könnte man verbessern?}
\framebox(\textwidth,80){}
\newpage
\Qitem[c]{Welche Funktionen haben Sie vermisst?}
\framebox(\textwidth,80){}
\Qitem[d]{ \Qq{Wie schätzen Sie ihre persönlichen \\Computerkenntnisse ein?} \Qtab{3cm}{sehr schlecht \Qrating{5} sehr gut}}
\Qitem[e]{ \Qq{Wie häufig arbeiten Sie mit dem Simulator} \Qtab{3cm}{sehr selten \Qrating{5} sehr häufig}}





\end{document}
