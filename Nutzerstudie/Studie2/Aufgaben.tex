\documentclass[accentcolor=tud1a,11pt]{tudexercise}
\usepackage{template}
\begin{document}
\nutzerstudie
\section*{Anlegen, bearbeiten und löschen von Skripts}
\begin{itemize}
	\item Fügen Sie eine Datei mit dem Titel \textbf{Bild} hinzu, dabei ist der Quellpfad \textbf{C:\Users\simuser\Desktop\datei} und der Zielpfad soll \textbf{C:\Users\simuser\Desktop\bild.png} sein.
	\item Erstellen Sie ein neues Skript mit dem Namen \textbf{Hallo Welt}.
	\item Fügen Sie dem Skript das Programm \textbf{Paint} und die Datei \textbf{Bild} hinzu, dabei soll \textbf{Paint} erst starten, wenn \textbf{Bild} verschoben wurde, nutzen Sie dafür verschiedene Stages.
	\item Speichern Sie das Skript und führen es aus.
	\item Prüfen Sie, ob das Bild in Paint geöffnet wurde.
	\item Beenden Sie das gestartete Programm.
	\item Löschen Sie das Skript \textbf{Hallo Welt}		
\end{itemize}
\section*{Erkennen und beheben von Fehlern}
\begin{itemize}
	\item Führen Sie das Skript \textbf{Startscript} aus, dabei entsteht ein Fehler. Finden Sie heraus, woher der Fehler kommt und beheben Sie ihn.
\end{itemize}
\end{document}

