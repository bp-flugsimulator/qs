\documentclass[accentcolor=tud1a,11pt]{tudexercise}
\usepackage{template}
\begin{document}
\nutzerstudie
\section*{Anlegen, bearbeiten und löschen von Skripts}
\begin{itemize}
	\item Erstellen Sie ein neues Skript mit dem Namen \textbf{Hallo Welt}.
	\item Fügen Sie dem Skript das Programm \textbf{<Programm1>} und \textbf{<Programm2>}, dabei soll \textbf{<Programm2>} erst starten, wenn \textbf{<Programm1>} gestartet wurde.
	\item Speichern Sie das Skript und führen es aus.
	\item Prüfen Sie, ob die Programme in der richtigen Reihenfolge gestartet wurden und beenden Sie diese.
	\item Fügen Sie die Datei \textbf{<Datei1>} zum Client \textbf{<Client1>} hinzu.
	\item Bearbeiten Sie das von Ihnen bereits erstellte Skript \textbf{Halo Welt}; Fügen Sie dabei \textbf{<Datei1>} hinzu und entfernen Sie \textbf{<Programm1>}
	\item Führen Sie das Skript aus und prüfen, ob \textbf{<Datei1>} kopiert wurde.
	\item Beenden Sie das gestartete Programm.
	\item Löschen Sie das Skript \textbf{Hallo Welt}		
\end{itemize}
\section*{Erkennen und beheben von Fehlern}
\begin{itemize}
	\item Führen Sie das Skript \textbf{<Skript2>} aus, dabei entsteht ein Fehler. Finden Sie heraus, woher der Fehler kommt und beheben Sie ihn.
\end{itemize}
\end{document}

