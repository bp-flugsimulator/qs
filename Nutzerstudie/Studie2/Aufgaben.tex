\documentclass[accentcolor=tud1a,11pt]{tudexercise}
\usepackage{template}
\begin{document}
\nutzerstudie
\section*{Anlegen, Bearbeiten und Löschen von Skripts}
Alle Angaben beziehen sich auf den Client \textbf{Xplane}.
\begin{itemize}
	\item Fügen Sie eine Datei mit dem Titel \textbf{Bild} zum Filesystem hinzu, \\
		dabei ist der Quellpfad (Source) \textbf{C:\textbackslash Users\textbackslash simuser\textbackslash Desktop\textbackslash datei} \\ 
		und der Zielpfad (Destination) soll \textbf{C:\textbackslash Users\textbackslash simuser\textbackslash Desktop\textbackslash bild.png} sein.
	\item Erstellen Sie ein neues Skript mit dem Namen \textbf{Hallo Welt} im Menü \textbf{scripts}.
	\item Fügen Sie dem Skript das Programm \textbf{Paint} und die Datei \textbf{Bild} hinzu, dabei soll \textbf{Paint} erst starten, wenn \textbf{Bild} verschoben wurde. Nutzen Sie dafür verschiedene Inidizes. Skriptkomponenten starten nach aufsteigender Indexreihenfolge.
	\item Speichern Sie das Skript und führen es aus.
	\item Prüfen Sie am Hauptrechner, ob das Bild in Paint geöffnet wurde.
	\item Beenden Sie das gestartete Programm. \\
		Hinweis: Da das Beenden von gesamten Skripten noch nicht möglich ist, können Sie dies nur im Client Menü tun.
	\item Löschen Sie das Skript \textbf{Hallo Welt}		
\end{itemize}
\section*{Erkennen und Beheben von Fehlern}
Ab hier beziehen sich alle Angaben auf den Client \textbf{vision}.
\begin{itemize}
	\item Legen Sie das Programm \textbf{Bat} im Menü des Clients an, \\
		dabei ist der Pfad \textbf{C:\textbackslash Users\textbackslash Vision\textbackslash Desktop\textbackslash script.bat}.
	\item Starten Sie das Programm.
	\item Beim Starten des Programmes ist ein Fehler entstanden. Finden Sie den Fehler im Log und beheben Sie ihn.\\
		Hinweis: Dafür müssen Sie das Programm modifizieren.
	\item Führen Sie das Skript erneut aus und stellen Sie sicher, dass der Fehler behoben wurde.
	\item Löschen Sie das Programm \textbf{Bat}.
\end{itemize}

\end{document}

