\documentclass[accentcolor=tud1a,11pt]{tudexercise}
\usepackage{template}
\begin{document}
\nutzerstudie
\section*{Anlegen, bearbeiten und löschen von Skripts}
Alle Angaben beziehen sich auf den Client Xplane.
\begin{itemize}
	\item Fügen Sie eine Datei mit dem Titel \textbf{Bild} hinzu, \\
		dabei ist der Quellpfad \textbf{C:\textbackslash Users\textbackslash simuser\textbackslash Desktop\textbackslash datei} \\ 
		und der Zielpfad soll \textbf{C:\textbackslash Users\textbackslash simuser\textbackslash Desktop\textbackslash bild.png} sein.
	\item Erstellen Sie ein neues Skript mit dem Namen \textbf{Hallo Welt}.
	\item Fügen Sie dem Skript das Programm \textbf{Paint} und die Datei \textbf{Bild} hinzu, dabei soll \textbf{Paint} erst starten, wenn \textbf{Bild} verschoben wurde. Nutzen Sie dafür verschiedene Stages.
	\item Speichern Sie das Skript und führen es aus.
	\item Prüfen Sie, ob das Bild in Paint geöffnet wurde.
	\item Beenden Sie das gestartete Programm.
	\item Löschen Sie das Skript \textbf{Hallo Welt}		
\end{itemize}
\section*{Erkennen und beheben von Fehlern}
\begin{itemize}
	\item Legen Sie das Programm \textbf{Script} an, \\
		dabei ist der Pfad \textbf{C:\textbackslash Users\textbackslash vision\textbackslash Desktop\textbackslash script.bat}.
	\item Starten Sie das Programm.
	\item Beim Starten des Programmes ist ein Fehler entstanden. Finden Sie den Fehler im Log und beheben Sie ihn.
	\item Führen Sie das Skript erneut aus und stellen Sie sicher, dass der Fehler behoben wurde.
\end{itemize}
\end{document}

