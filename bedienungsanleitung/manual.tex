\documentclass[accentcolor=tud1c, paper=a4, colorback]{tudreport}
\usepackage[ngerman]{babel}
\usepackage[utf8]{inputenc}

\usepackage[german]{todonotes}
\usepackage[most]{tcolorbox}
\usepackage{amsmath, imakeidx, graphicx, hyperref, xcolor}

%opening
\title{Manual Flightsim control software}
\subtitle{BP-Team 19}
\subsubtitle{Wintersemester 17/18}
\makeindex

\newcommand{\ind}[1]{#1\index{#1}}
\graphicspath{ {img/} }

\begin{document}
	\maketitle
	\tableofcontents

	\todototoc
	\listoftodos

	\chapter{Präambel}
	\todo[inline]{basic Infos einfügen}
	\todo[inline]{Präambel zu geschlechtergerechten Sprache}

	\chapter{Introduction}

	\section{Project structure}
	Based on the network structure the software is based on a
	\ind{client-server system}\footnote{\url{https://en.wikipedia.org/wiki/Client-server\_model}}.
	\begin{figure}[h]
		\centering
		\includegraphics[width=\textwidth]{softwarelayout}
		\caption{Topology of the software layout}
	\end{figure}
	\\
	The user communicate with the system using a web interface running on the master which
	forwards the users input to the clients.
	\\
	Both, the client and the server, are written in \ind{Python 3}\footnote{\url{https://www.python.org/}}.
	To shrinken the necessary codebase popular frameworks were used during development. For example
	the website is using the \ind{Django framework}\footnote{\url{https://www.djangoproject.com/}}.
	A step by step guide for setting up the master software can be found in the project repository
	\footnote{\url{https://github.com/bp-flugsimulator/server/blob/master/README.md}}.
	\\
	The client is also written in Python and does not need any configuration besides the address of
	the master server during startup. All further configuration is then handled by the master.

	\section{Nomenclature}
	There are a few terms needing a more in depth definition. This section describes them in detail
	to ease the understanding of further chapters.

	\subsection{\ind{Master}}
	Central computer, which runs the web server. Currently this is the \ind{simcontrol} computer.
	Usually the user opens the web ui on this computer but if there is network access every other
	machine inside the institute network can be used to access the web ui. Internet access without
	a firewall should not be permitted, since there is no authentication for accessing this component.

	\subsection{\ind{Client}}
	Every computer which is part of the network besides the master is called a client in this context.
	An example would be \ind{Vision}. A client can run multiple programs.

	\subsection{\ind{Program}}
	A program in this context is all software which needs to be started when the simulator is used e.g. X-Plane.
	An example for the config options of a program can be found in figure \ref{add_program}.

	\subsection{\ind{Files}}
	Since most simulations require a few plugins it is possible to copy files and folders to freely
	configurable locations during a script run. An example of possible configuration options can be
	found in figure \ref{add_filesystem}.

	\subsection{\ind{Script}}\label{script}
	A script is the combination of Programs and file movements. It can be created by a user to model
	a scenario necessary for a simulation. The order and the needed programs and files can be configured
	freely by using the graphical editor inside the web ui.
	\\
	A script is divided into several \textit{\ind{stages}}. A stage can consist of multiple programs
	and file movements. This concept is used to make dependency management between programs more easy.
	If for example a plugin for a simulation is needed, it is most times necessary to copy it before starting
	X-Plane. The plugin copy would therefore be done in stage 0 while X-Plane would be started in stage 1.

	\chapter{Workflows}
	\todo[inline]{Standardworkflows a la Eine Node Hinzufügen}
	\section{Adding a client}
	After starting the web server the ui has to be opened in a browser.
	If you work on \ind{simcontrol} just type \url{http://localhost:8000/slaves} inside
	your browser. If you are using a different computer inside the network you
	have to replace \texttt{localhost} with the ip address of simcontrol.
	\\
	\missingfigure{Weboberfläche ohne konfigurierte Nodes}
	Using the \texttt{+} button next to the client header on the left side
	a new client can be added. Necessary parameters are a name, an ip address,
	and a MAC adress. Using windows the last two can be retrieved using
	the command \texttt{ipconfig}. The name is only used for the user interface
	and can be set freely.
	\\
	After creation a new client appears on the left sidebar. Clicking on its tab
	opens an overview with the configured programs and files for this node.
	Additionally using the buttons \texttt{Switch On}, \texttt{Edit}, and \texttt{Delete}
	the client can be booted up, deleted or its configuration changed.

	\section{Adding a program}
	\begin{figure}[h]
		\centering
		\label{add_program}
		\includegraphics[width=.4\textwidth]{add_program}
		\caption{The add program dialog with example values}
	\end{figure}
	To add a program a client has to be selected first. By clicking on the \texttt{+}
	button next to the program tab a dialog opens. The following fields have to be filled
	out:
	\begin{center}
	\begin{tabular}{l|l}
		Field & Description \\\hline
		Display Name &  Display name of the program, can be chosen freely\\
		Path to Executable & Path to the executable file\\
		Arguments & Command-line arguments for the program\\
		Start time & Seconds to wait before executing the next stage during a script run\\
	\end{tabular}
		\begin{tcolorbox}[width=\textwidth, colback=red!30, arc=0pt, boxrule=0pt]
		{\color{red}\textbf{Warning}}\hspace{0.5cm}
		Windows XP can behave unexpectedly when the Path to the executable contains spaces.
		\end{tcolorbox}
	\end{center}
	\\
	Start time is only used during script runs. If it is set to -1, there is
	no wait time after executing this step. For a more complete description of
	stages please look at section \ref{script}.

	\section{Adding a file/folder movement}

	\begin{figure}[h]
		\centering
		\label{add_filesystem}
		\includegraphics[width=.4\textwidth]{add_filesystem}
		\caption{The add file dialog with example values}
	\end{figure}
	\\
	If a special plugin is necessary for a simulation scenario, it can be copied by using
	the file/folder movement function.
	\\
	In the client overview a new file movement can be added by clicking the \texttt{+} button
	next to the file management button. The following fields have to be filled out:
	\\
	\begin{center}
	\begin{tabular}{l|l}
		Field & Description \\\hline
		Display Name &  Display name of the program, can be chosen freely\\
		Source Path & Path to the source file/folder\\
		Source Type & Type of the source\\
		Destination Path & Destination path (and/or new name)\\
		Destination Type & Copy mode\\
	\end{tabular}
	\end{center}
	The destination type can be one of two options, \texttt{Replace With} or \texttt{Insert Into}.
	The first one renames the source object during copying. The last name in the destination path
	is the new name of the source path. For example in figure \ref{add_filesystem} the folder 
	"heli" will be copied and renamed to "helicopter".
	\texttt{Insert Into} copies the source into a subfolder. In the figure this would mean that
	a new folder "heli" below "helicopter" would be created and the contents of the source will
	be copied inside it.

	\chapter{Appendix}

	\clearpage
	\addcontentsline{toc}{chapter}{Index}
	\printindex


\end{document}
