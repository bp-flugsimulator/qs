\section{Userstories}

\begin{table}[htbp]
    \begin{minipage}{\linewidth}
        \setlength{\tymax}{0.5\linewidth}
        \centering
        \small
        \begin{tabulary}{\textwidth}{|l|p{10cm}|} \hline
            ID   & 1 \\\hline
            Name  & Willkommensnachricht\\\hline
            Beschreibung& Wenn sich der User mit Hilfe eines Browsers mit dem Server verbindet wird eine Willkommensnachricht angezeigt. \\\hline
            Akzeptanz &Nachdem der Benutzer eine korrekte Anfrage an \texttt{\slash welcome} gesendet hat, gibt der Server eine Webseite mit dem Inhalt ``welcome'' zurück.\\\hline
            Story Points&10\\\hline
            Entwickler &Heiko\\\hline
            Iteration &1\\\hline
            Stunden  &2\\\hline
            Velocity &5 SP\slash h\\\hline
            Bemerkung &Der Server wurde in dieser Userstory aufgesetzt. Diese Userstory ist im Endprodukt nicht mehr vorhanden, da diese kein Sinnvolle Verwendung hat.\\\hline
        \end{tabulary}
    \end{minipage}
\end{table}



\begin{table}[htbp]
    \begin{minipage}{\linewidth}
        \setlength{\tymax}{0.5\linewidth}
        \centering
        \small
        \begin{tabulary}{\textwidth}{|l|p{10cm}|} \hline
            ID   &2\\\hline
            Name  &Client registrieren\\\hline
            Beschreibung&Als Benutzer muss ich die Möglichkeit haben einen Client zu registrieren, um diesen später verwalten zu können.\\\hline
            Akzeptanz &Durch das Betätigen des entsprechenden Knopfes öffnet sich ein Dialog. In diesem Dialog befinden sich Felder für Name, IP-Adresse und MAC-Adresse, in die der Benutzer die Werte für den anzulegenden Client eintragen kann. Ungültige Eingaben werden abgewiesen und der Benutzer wird darauf hingewiesen. Nach korrekter Eingabe, wird der Client in der Datenbank gespeichert.\\\hline
            Story Points&8\\\hline
            Entwickler &Tim, Jonas, Frederik\\\hline
            Iteration &1\\\hline
            Stunden  &30\\\hline
            Velocity &0.26 SP\slash h\\\hline
        \end{tabulary}
    \end{minipage}
\end{table}



\begin{table}[htbp]
    \begin{minipage}{\linewidth}
        \setlength{\tymax}{0.5\linewidth}
        \centering
        \small
        \begin{tabulary}{\textwidth}{|l|p{10cm}|} \hline
            ID   &3\\\hline
            Name  &Hochfahren eines Clients\\\hline
            Beschreibung&Als Benutzer will ich die Möglichkeit haben einen Client mit der Hilfe des Webinterfaces hochzufahren. \\\hline
            Akzeptanz &Nachdem der Benutzer den entsprechenden Knopf betätigt hat, wird der Client gestartet und der Benutzer darüber benachrichtigt. Dies kann nur geschehen, wenn der Benutzer zuvor die richtige MAC-Adresse des Clients eingetragen hat.\\\hline
            Story Points&3\\\hline
            Entwickler &Heiko\\\hline
            Iteration &2\\\hline
            Stunden  &7\\\hline
            <<<<<<< Updated upstream
            Velocity &0.42 SP\slash h\\\hline
            =======
            Velocity &0.42 SP\slash h\\\hline
            Bemerkung &Der Benutzer muss eine Autostart Datei anlegen, welche die Client Software startet. Damit sich der Client mit dem Server automatisch verbindet.\\\hline
            >>>>>>> Stashed changes
        \end{tabulary}
    \end{minipage}
\end{table}


\begin{table}[htbp]
    \begin{minipage}{\linewidth}
        \setlength{\tymax}{0.5\linewidth}
        \centering
        \small
        \begin{tabulary}{\textwidth}{|l|p{10cm}|} \hline
            ID   &4\\\hline
            Name  &Status eines Clients\\\hline
            Beschreibung&Als Benutzer will ich den Status (verbunden, nicht verbunden) jedes bereits registrierten Clients einsehen können, um mit diesem zu interagieren.\\\hline
            Akzeptanz &Der Status wird korrekt angezeigt und aktualisiert, sobald sich der reale Status des Clients ändert.
            Story Points&4\\\hline
            Entwickler &Tim\\\hline
            Iteration &4\\\hline
            Stunden  &6\\\hline
            Velocity &0.666 SP\slash h\\\hline
        \end{tabulary}
    \end{minipage}
\end{table}


\begin{table}[htbp]
    \begin{minipage}{\linewidth}
        \setlength{\tymax}{0.5\linewidth}
        \centering
        \small
        \begin{tabulary}{\textwidth}{|l|p{10cm}|} \hline
            ID   &5\\\hline
            Name  &Client herunterfahren\\\hline
            Beschreibung&Als Benutzer will ich einen Client per Webinterface herunterfahren können, um dies nicht auf dem entsprechendem Client selbst tun zu müssen.\\\hline
            Akzeptanz &Nachdem der Benutzer den entsprechenden Knopf betätigt hat, wird der Client heruntergefahren. Es wird dabei keine Rücksicht auf andere Programme genommen. \\\hline
            Story Points&5\\\hline
            Entwickler &Tim\\\hline
            Iteration &3\\\hline
            Stunden  &4\\\hline
            Velocity &1.25 SP\slash h\\\hline
        \end{tabulary}
    \end{minipage}
\end{table}



\begin{table}[htbp]
    \begin{minipage}{\linewidth}
        \setlength{\tymax}{0.5\linewidth}
        \centering
        \small
        \begin{tabulary}{\textwidth}{|l|p{10cm}|} \hline
            ID   &6\\\hline
            Name  &Kontextmenü\\\hline
            Beschreibung&Als Benutzer will ich alle möglichen Aktionen (starten, stoppen, bearbeiten, löschen, etc.), welche ich für einen Client ausführen kann, in einem Menü auswählen können.
            Das Menü soll die Bedienung des Webinterfaces erleichtern.\\\hline
            Akzeptanz &Nach der Auswahl eines Client werden die zugehörigen Aktionen in einem Menü angezeigt.\\\hline
            Story Points&10\\\hline
            Entwickler &Leonardo, Jonas, Tim\\\hline
            Iteration &1\\\hline
            Stunden  &10\\\hline
            Velocity &1 SP\slash h\\\hline
            Bemerkung & Die einzelnen Aktionen sind nicht mehr in einem Menü zusammengefasst.\\\hline
        \end{tabulary}
    \end{minipage}
\end{table}



\begin{table}[htbp]
    \begin{minipage}{\linewidth}
        \setlength{\tymax}{0.5\linewidth}
        \centering
        \small
        \begin{tabulary}{\textwidth}{|l|p{10cm}|} \hline
            ID   &7\\\hline
            Name  &Starten eines Programmes\\\hline
            Beschreibung&Als Benutzer will ich die Möglichkeit haben, ein bereits registriertes Programm auf einem Client starten zu können. Dadurch muss ich Programme nicht selbst auf den Clients starten. \\\hline
            Akzeptanz &Nachdem der Benutzer das Programm gestartet hat, indem er den entsprechenden Knopf betätigt hat, wird ein Startbefehl an den Client gesendet.
            Story Points&25\\\hline
            Entwickler &Jonas, Tim\\\hline
            Iteration &3\\\hline
            Stunden  &50\\\hline
            Velocity &0.5 SP\slash h\\\hline
        \end{tabulary}
    \end{minipage}
\end{table}



\begin{table}[htbp]
    \begin{minipage}{\linewidth}
        \setlength{\tymax}{0.5\linewidth}
        \centering
        \small
        \begin{tabulary}{\textwidth}{|l|p{10cm}|} \hline
            ID   &8\\\hline
            Name  &Statusabfrage von Programmen auf einem Client\\\hline
            Beschreibung&Als Benutzer will ich die Möglichkeit haben, den Status eines registrierten Programmes (Wird ausgeführt, Ist beendet) einzusehen, damit ich mögliche Fehler im System erkennen kann.\\\hline
            Akzeptanz &Der Status des Programms ist aktuell, korrekt und wird im Webinterface angezeigt. Das Programm wird grün markiert, wenn es läuft und rot, wenn es einen Fehler hatte. Falls das Programm nicht läuft, ist es auch nicht farblich markiert.\\\hline
            Story Points&4\\\hline
            Entwickler &Tim\\\hline
            Iteration &4\\\hline
            Stunden  &6\\\hline
            Velocity &0.666 SP\slash h\\\hline
        \end{tabulary}
    \end{minipage}
\end{table}



\begin{table}[htbp]
    \begin{minipage}{\linewidth}
        \setlength{\tymax}{0.5\linewidth}
        \centering
        \small
        \begin{tabulary}{\textwidth}{|l|p{10cm}|} \hline
            ID   &9\\\hline
            Name  &Datei bewegen\\\hline
            Beschreibung&Als Benutzer will ich die Möglichkeit haben, eine auf dem Server registrierte Datei auf einem ebenfalls auf dem Server registrierten Client zu bewegen.
            Dadurch kann ich einfacher Dateien innerhalb des Systems austauschen. Auch will ich, den Ausgangsstatus wiederherstellen können.\\\hline
            Akzeptanz &Nachdem der Benutzer den entsprechenden Kopf betätigt hat, wird die Datei auf dem Client verschoben und der Benutzer wird darüber benachrichtigt (Erfolg oder Misserfolg). Falls am Zielort bereits eine Datei existiert, wird diese separat gespeichert. Der Knopf ändert nach dem Bewegen seinen Anzeigetext auf ''restore''. Sofern eine Datei bereits bewegt wurde und der Benutzer auf den Zurücksetzen Knopf drückt, wird die Datei entfernt. Dies ist nur möglich, wenn die Datei in der Zwischenzeit nicht verändert wurde. Es ist nicht mögliche die Aktion Datei bewegen auszuführen, falls die Datei bereits bewegt worden ist. Ebenfalls ist es nicht möglich eine Datei zurückzusetzten, wenn sie nicht verschoben, oder bereits zurückgesetzt wurde.\\\hline
            Story Points&10\\\hline
            Entwickler &Leonardo, Jonas\\\hline
            Iteration &7\\\hline
            Stunden  &30\\\hline
            Velocity &0.33 SP\slash h\\\hline
        \end{tabulary}
    \end{minipage}
\end{table}



\begin{table}[htbp]
    \begin{minipage}{\linewidth}
        \setlength{\tymax}{0.5\linewidth}
        \centering
        \small
        \begin{tabulary}{\textwidth}{|l|p{10cm}|} \hline
            ID   &10\\\hline
            Name  &Dateien herunterladen\\\hline
            Beschreibung&Als Benutzer will ich die Möglichkeit haben eine Datei vom Server auf den Client zu übertragen, damit ich dies nicht per USB oder ähnlichem Weg erledigen muss.\\\hline
            Akzeptanz &Die Datei wird auf den Client kopiert und kann vom Benutzer von dort aus verschoben werden.\\\hline
            Story Points&-\\\hline
            Entwickler &-\\\hline
            Iteration &-\\\hline
            Stunden  &-\\\hline
            Velocity &-\\\hline
            Bemerkung & Bevor diese Userstory durchgeführt werden konnte, haben die Auftraggeber sich gegen diese Funktionalität entschieden. Die Funktionalität wird dennoch in einer anderen Variante bereitgestellt (Siehe Userstory 36). \\\hline
        \end{tabulary}
    \end{minipage}
\end{table}



\begin{table}[htbp]
    \begin{minipage}{\linewidth}
        \setlength{\tymax}{0.5\linewidth}
        \centering
        \small
        \begin{tabulary}{\textwidth}{|l|p{10cm}|} \hline
            ID   &11\\\hline
            Name  &Datei Löschen\\\hline
            Beschreibung&Als Benutzer will ich die Möglichkeit haben eine Datei auf dem Client zu löschen, damit ich dies nicht selbst auf dem System erledigen muss.\\\hline
            Akzeptanz &Der Benutzer wird vor dem Löschen gewarnt und muss es erst bestätigen. Falls er die Aktion bestätigt, wird die korrekte Datei auf dem Client permanent gelöscht.\\\hline
            Story Points&4\\\hline
            Entwickler &?\\\hline
            Iteration &?\\\hline
            Stunden  &?\\\hline
            Velocity &?\\\hline
            Bemerkung & Bevor diese Userstory durchgeführt werden konnte, haben die Auftraggeber sich gegen diese Funktionalität entschieden.
            \\\hline
        \end{tabulary}
    \end{minipage}
\end{table}



\begin{table}[htbp]
    \begin{minipage}{\linewidth}
        \setlength{\tymax}{0.5\linewidth}
        \centering
        \small
        \begin{tabulary}{\textwidth}{|l|p{10cm}|} \hline
            ID   &12\\\hline
            Name  &Starten eines Webbrowsers beim Softwarestart\\\hline
            Beschreibung&Als Benutzer will ich die Möglichkeit haben einen Webbrowser meiner Wahl automatisch zu öffnen, sobald ich die Server Software starte. Der Webbrowser soll auch auf die Seite des Webinterfaces navigieren, damit ich diese nicht manuell tun muss.\\\hline
            Akzeptanz &Nachdem Starten der Server Software, wird der ausgewählte Webbrowser mit der Startseite des Webinterfaces geöffnet.\\\hline
            Story Points&2\\\hline
            Entwickler &?\\\hline
            Iteration &?\\\hline
            Stunden  &?\\\hline
            Velocity &?\\\hline
            Bemerkung & Auf Wunsch der Auftraggeber wurde diese Funktion über die Autostartfunktionalität des Betriebssystems realisiert.
            \\\hline
        \end{tabulary}
    \end{minipage}
\end{table}

\todo[inline]{Us 13 fehlt}

\begin{table}[htbp]
    \begin{minipage}{\linewidth}
        \setlength{\tymax}{0.5\linewidth}
        \centering
        \small
        \begin{tabulary}{\textwidth}{|l|p{10cm}|} \hline
            ID   &14\\\hline
            Name  &Client löschen\\\hline
            Beschreibung&Als Benutzer will die Möglichkeit haben, einen bereits registrierten Client zu entfernen. Dadurch muss ein Client, der nicht mehr existiert, auch nicht im Webinterface angezeigt werden.\\\hline
            Akzeptanz &Nachdem der Benutzer den entsprechenden Knopf betätigt hat, öffnet sich ein Bestätigungsdialog. Nach der Bestätigung ist der Client nicht mehr für den Benutzer im Webinterface zu sehen und wird aus der Datenbank entfernt. Falls der Benutzer die Aktion nicht bestätigt, wird der Client nicht aus der Datenbank gelöscht und wird weiterhin im Webinterface angezeigt.\\\hline
            Story Points&4\\\hline
            Entwickler &Frederik, Leonardo, Tim\\\hline
            Iteration &1\\\hline
            Stunden  &9\\\hline
            Velocity &0.444 SP\slash h\\\hline
        \end{tabulary}
    \end{minipage}
\end{table}



\begin{table}[htbp]
    \begin{minipage}{\linewidth}
        \setlength{\tymax}{0.5\linewidth}
        \centering
        \small
        \begin{tabulary}{\textwidth}{|l|p{10cm}|} \hline
            ID   &15\\\hline
            Name  &Client bearbeiten\\\hline
            Beschreibung&Als Benutzer will ich die Möglichkeit haben, die Werte Name, IP-Adresse und MAC-Adresse eines bereits registrierten Client zu bearbeiten. Dadurch muss bei einer Änderung dieser Werte der Client nicht erneut hinzugefügt werden.\\\hline
            Akzeptanz &Nachdem der Benutzer den entsprechenden Knopf betätigt hat, öffnet sich ein Dialog in dem sich die aktuellen Werte des Clients befinden. Der Benutzer kann nun die Werte verändern. Sollten die Werte nicht konform zu entsprechenden IP- und MAC- Adressstandards sein, wird der Benutzer darauf hingewiesen. Sollten die Werte konform sein, werden bei betätigung des Speichern Knopfes diese in der Datenbank gespeichert. Der Benutzer kann den Vorgang auch abbrechen, in diesem Fall werden die Werte nicht gespeichert. \\\hline
            Story Points&4\\\hline
            Entwickler &Tim\\\hline
            Iteration &1\\\hline
            Stunden  &3\\\hline
            Velocity &1.333 SP\slash h\\\hline
        \end{tabulary}
    \end{minipage}
\end{table}



\begin{table}[htbp]
    \begin{minipage}{\linewidth}
        \setlength{\tymax}{0.5\linewidth}
        \centering
        \small
        \begin{tabulary}{\textwidth}{|l|p{10cm}|} \hline
            ID   &16\\\hline
            Name  &Client anzeigen\\\hline
            Beschreibung&Als Benutzer will ich die Möglichkeit haben, einen Client mit seinem Namen, seiner IP-Adresse und seiner MAC-Adresse im Webinterface zu sehen, um Informationen über den Client aus dem Webinterface bekommen zu können.\\\hline
            Akzeptanz &Für einen Client befindet sich ein Eintrag mit seinem Namen innerhalb des Webinterfaces. Klickt man auf diesen Eintrag, werden IP- und Mac-Adresse ebenfalls angezeigt.\\\hline
            Story Points&4\\\hline
            Entwickler &Jonas, Tim\\\hline
            Iteration &1\\\hline
            Stunden  &4\\\hline
            Velocity &1 SP\slash h\\\hline
        \end{tabulary}
    \end{minipage}
\end{table}



\begin{table}[htbp]
    \begin{minipage}{\linewidth}
        \setlength{\tymax}{0.5\linewidth}
        \centering
        \small
        \begin{tabulary}{\textwidth}{|l|p{10cm}|} \hline
            ID   &17\\\hline
            Name  &Wechsel zwischen Unterseiten\\\hline
            Beschreibung&Als Benutzer will ich die Möglichkeit haben, die verschiedenen Seiten über eine Navigationsleiste zu erreichen.\\\hline
            Akzeptanz &Der Benutzer sieht alle für ihn zugängliche Seiten in der Navigationsleiste und kann zu diesen, durch ein betätigen des Knopfes, navigieren.\\\hline
            Story Points&1\\\hline
            Entwickler &Jonas\\\hline
            Iteration &3\\\hline
            Stunden  &1\\\hline
            Velocity &1 SP\slash h\\\hline
        \end{tabulary}
    \end{minipage}
\end{table}



\begin{table}[htbp]
    \begin{minipage}{\linewidth}
        \setlength{\tymax}{0.5\linewidth}
        \centering
        \small
        \begin{tabulary}{\textwidth}{|l|p{10cm}|} \hline
            ID   &19\\\hline
            Name  &Programm anzeigen\\\hline
            Beschreibung&Als Benutzer will ich die Möglichkeit haben, alle registrierten Programme, zugeordnet zu ihrem jeweiligen Client, einzusehen.\\\hline
            Akzeptanz &Die Programme werden korrekt zu dem zugehörigen Client angezeigt. Auch sind alle Programme die der Benutzer registriert hat zu sehen.\\\hline
            Story Points&8\\\hline
            Entwickler &Leonardo, Tim\\\hline
            Iteration &2\\\hline
            Stunden  &6\\\hline
            Velocity &1.3 SP\slash h\\\hline
        \end{tabulary}
    \end{minipage}
\end{table}



\begin{table}[htbp]
    \begin{minipage}{\linewidth}
        \setlength{\tymax}{0.5\linewidth}
        \centering
        \small
        \begin{tabulary}{\textwidth}{|l|p{10cm}|} \hline
            ID   &20\\\hline
            Name  &Programm hinzufügen\\\hline
            Beschreibung&Als Benutzer will ich die Möglichkeit haben, ein Programm, welches auf einem Client liegt, mit bestimmten Parametern zu registrieren.\\\hline
            Akzeptanz &Nachdem der Benutzer den entsprechenden Knopf betätigt hat, öffnet sich ein Dialog, welcher Felder für Programmnamen, Pfad zur ausführbaren Datei und Aufrufargumenten beinhaltet. Falls der Benutzer fehlerhafte Werte angibt, wird dies mitgeteilt, sonst wird das Programm in der Datenbank gespeichert.\\\hline
            Story Points&6\\\hline
            Entwickler &Tim\\\hline
            Iteration &2\\\hline
            Stunden  &5\\\hline
            Velocity &1.2 SP\slash h\\\hline
        \end{tabulary}
    \end{minipage}
\end{table}



\begin{table}[htbp]
    \begin{minipage}{\linewidth}
        \setlength{\tymax}{0.5\linewidth}
        \centering
        \small
        \begin{tabulary}{\textwidth}{|l|p{10cm}|} \hline
            ID   &21\\\hline
            Name  &Programm löschen\\\hline
            Beschreibung&Als Benutzer will ich die Möglichkeit haben, ein bereits registriertes Programm wieder zu entfernen.\\\hline
            Akzeptanz &Nachdem der Benutzer den entsprechenden Knopf betätigt hat öffnet sich Dialog, indem der Benutzer den Löschvorgang bestätigen muss. Nach Bestätigung wird das Programm aus der Datenbank entfernt und dem Benutzer nicht mehr angezeigt.\\\hline
            Story Points&4\\\hline
            Entwickler &Frederik\\\hline
            Iteration &2\\\hline
            Stunden  &4,5\\\hline
            Velocity &0.89 SP\slash h\\\hline
        \end{tabulary}
    \end{minipage}
\end{table}



\begin{table}[htbp]
    \begin{minipage}{\linewidth}
        \setlength{\tymax}{0.5\linewidth}
        \centering
        \small
        \begin{tabulary}{\textwidth}{|l|p{10cm}|} \hline
            ID   &22\\\hline
            Name  &Programm bearbeiten\\\hline
            Beschreibung&Als Benutzer will ich die Möglichkeit haben, ein bereits registriertes Programm zu bearbeiten.\\\hline
            Akzeptanz &Nach dem Betätigen des entsprechenden Knopfes öffnet sich ein Dialog, in welchem die letzte Konfiguration angezeigt wird. Die Änderungen werden nur gespeichert, sofern sie immer noch eine korrekte Konfiguration bilden. Auf fehlerhafte Eingaben wird hingewiesen und das Programm lässt sich nicht speichern.\\\hline
            Story Points&5\\\hline
            Entwickler &Tim\\\hline
            Iteration &3\\\hline
            Stunden  &3\\\hline
            Velocity &1.7 SP\slash h\\\hline
        \end{tabulary}
    \end{minipage}
\end{table}



\begin{table}[htbp]
    \begin{minipage}{\linewidth}
        \setlength{\tymax}{0.5\linewidth}
        \centering
        \small
        \begin{tabulary}{\textwidth}{|l|p{10cm}|} \hline
            ID   &23\\\hline
            Name  &Datei anzeigen\\\hline
            Beschreibung&Als Benutzer will ich die Möglichkeit haben, eine registrierte Datei, die zu einem Client gehört, im Webinterface einzusehen.\\\hline
            Akzeptanz &Der Benutzer sieht alle konfigurierten Dateien für den dazugehörigen Client, nachdem er in der Clientübersicht den Programmreiter im zugehörigen Programm geöffnet hat.\\\hline
            Story Points&3\\\hline
            Entwickler &Frederik\\\hline
            Iteration &3\\\hline
            Stunden  &3\\\hline
            Velocity &1 SP\slash h\\\hline
        \end{tabulary}
    \end{minipage}
\end{table}



\begin{table}[htbp]
    \begin{minipage}{\linewidth}
        \setlength{\tymax}{0.5\linewidth}
        \centering
        \small
        \begin{tabulary}{\textwidth}{|l|p{10cm}|} \hline
            ID   &24\\\hline
            Name  &Datei hinzufügen\\\hline
            Beschreibung&Als Benutzer will ich die Möglichkeit haben, eine Datei für einen Client zu registrieren.\\\hline
            Akzeptanz &Nachdem der Benutzer den entsprechenden Knopf betätigt hat, öffnet sich ein Dialog, welcher Dateiname, Quell- und Zielpfad sowie der Typ (Ordner oder Datei) abfragt. Es werden nur korrekte Eingaben gespeichert. Falls der Benutzer keine korrekte Daten angibt, wird dies ihm beim Klick auf den Speichern Knopf mitgeteilt und es kann erst fortgefahren werden, wenn die Fehler behoben wurden.\\\hline
            Story Points&4\\\hline
            Entwickler &Frederik\\\hline
            Iteration &3\\\hline
            Stunden  &4\\\hline
            Velocity &1 SP\slash h\\\hline
        \end{tabulary}
    \end{minipage}
\end{table}



\begin{table}[htbp]
    \begin{minipage}{\linewidth}
        \setlength{\tymax}{0.5\linewidth}
        \centering
        \small
        \begin{tabulary}{\textwidth}{|l|p{10cm}|} \hline
            ID   &25\\\hline
            Name  &Datei löschen\\\hline
            Beschreibung&Als Benutzer will ich die Möglichkeit haben, eine bereits registrierte Datei wieder zu löschen.\\\hline
            Akzeptanz &Nachdem Betätigten des entsprechenden Knopfes öffnet sich ein Dialog, indem der Benutzer den Löschvorgang bestätigen muss. Nur falls der Benutzer den Löschvorgang bestätigt wird die registrierte Datei aus der Datenbank gelöscht.\\\hline
            Story Points&3\\\hline
            Entwickler &Frederik\\\hline
            Iteration &5\\\hline
            Stunden  &1.5\\\hline
            Velocity &2 SP\slash h\\\hline
        \end{tabulary}
    \end{minipage}
\end{table}



\begin{table}[htbp]
    \begin{minipage}{\linewidth}
        \setlength{\tymax}{0.5\linewidth}
        \centering
        \small
        \begin{tabulary}{\textwidth}{|l|p{10cm}|} \hline
            ID   &26\\\hline
            Name  &Datei editieren\\\hline
            Beschreibung&Als Benutzer will ich die Möglichkeit haben, eine bereits registrierte Datei zu bearbeiten.\\\hline
            Akzeptanz &Nach dem Betätigen des Knopfes, öffnet sich das Dialog mit den momentanen Informationen. Nur wenn der Benutzer korrekte Änderungen angibt, werden diese gespeichert. Fehlerhafte Änderungen werden dem Benutzer angezeigt.\\\hline
            Story Points&5\\\hline
            Entwickler &Frederik\\\hline
            Iteration &7\\\hline
            Stunden  &2\\\hline
            Velocity &2.5 SP\slash h\\\hline
        \end{tabulary}
    \end{minipage}
\end{table}



\begin{table}[htbp]
    \begin{minipage}{\linewidth}
        \setlength{\tymax}{0.5\linewidth}
        \centering
        \small
        \begin{tabulary}{\textwidth}{|l|p{10cm}|} \hline
            ID   &27\\\hline
            Name  &Skript anzeigen\\\hline
            Beschreibung&Als Benutzer will ich die Möglichkeit haben ein registriertes Skript im Webinterface einzusehen.\\\hline
            Akzeptanz &Das Skript wird korrekt angezeigt, wenn der Benutzer das Skript in der Skriptübersicht auswählt.\\\hline
            Story Points&15\\\hline
            Entwickler &Jonas\\\hline
            Iteration &3\\\hline
            Stunden  &20\\\hline
            Velocity &0,75 SP\slash h\\\hline
        \end{tabulary}
    \end{minipage}
\end{table}



\begin{table}[htbp]
    \begin{minipage}{\linewidth}
        \setlength{\tymax}{0.5\linewidth}
        \centering
        \small
        \begin{tabulary}{\textwidth}{|l|p{10cm}|} \hline
            ID   &28\\\hline
            Name  &Skript hinzufügen\\\hline
            Beschreibung&Als Benutzer will ich die Möglichkeit haben, ein Skript hinzuzufügen. In einem Skript kann ich Dateien und Programme mit einer Startreihenfolge spezifizieren.\\\hline
            Akzeptanz &Nachdem der Benutzer den entsprechenden Knopf betätigt hat, wird er auf eine weitere Seite geleitet, wo er ein Skript erstellen kann. Bei fehlerhaften Eingaben wird der Benutzer darauf hingewiesen. Es können keine fehlerhafte Skripte hinzugefügt werden.\\\hline
            Story Points&7\\\hline
            Entwickler &Jonas\\\hline
            Iteration &4\\\hline
            Stunden  &5\\\hline
            Velocity &1.4 SP\slash h\\\hline
        \end{tabulary}
    \end{minipage}
\end{table}



\begin{table}[htbp]
    \begin{minipage}{\linewidth}
        \setlength{\tymax}{0.5\linewidth}
        \centering
        \small
        \begin{tabulary}{\textwidth}{|l|p{10cm}|} \hline
            ID   &29\\\hline
            Name  &Skript löschen\\\hline
            Beschreibung&Als Benutzer will ich die Möglichkeit haben ein bereits registrierte Skript zu löschen.\\\hline
            Akzeptanz &Nachdem der Benutzer den entsprechenden Knopf betätigt, wird ein Dialog geöffnet, in dem der Benutzer den Löschvorgang bestätigen muss. Sofern der Benutzer den Löschvorgang bestätigt, wird das Skript gelöscht.\\\hline
            Story Points&5\\\hline
            Entwickler &Jonas\\\hline
            Iteration &4\\\hline
            Stunden  &5\\\hline
            Velocity &1 SP\slash h\\\hline
        \end{tabulary}
    \end{minipage}
\end{table}



\begin{table}[htbp]
    \begin{minipage}{\linewidth}
        \setlength{\tymax}{0.5\linewidth}
        \centering
        \small
        \begin{tabulary}{\textwidth}{|l|p{10cm}|} \hline
            ID   &30\\\hline
            Name  &Skript bearbeiten\\\hline
            Beschreibung&Als Benutzer will ich die Möglichkeit haben, ein bereits registriertes Skript zu bearbeiten.\\\hline
            Akzeptanz &Der Benutzer wird auf die Bearbeitungsseite weitergeleitet, nachdem er den entsprechenden Knopf betätigt hat. Auf dieser Seite kann der der Benutzer das Skript bearbeiten kann. Falls die Änderungen des Benutzers korrekt sind, wird das Skript gespeichert. Falls das nicht der Fall ist, wird dem Benutzer das mitgeteilt.\\\hline
            Story Points&5\\\hline
            Entwickler &Leonardo\\\hline
            Iteration &7\\\hline
            Stunden  &7\\\hline
            Velocity &0.714 SP\slash h\\\hline
        \end{tabulary}
    \end{minipage}
\end{table}



\begin{table}[htbp]
    \begin{minipage}{\linewidth}
        \setlength{\tymax}{0.5\linewidth}
        \centering
        \small
        \begin{tabulary}{\textwidth}{|l|p{10cm}|} \hline
            ID   &31\\\hline
            Name  &Starten von Skripten\\\hline
            Beschreibung&Als Benutzer will ich die Möglichkeit haben, ein Skript meiner Wahl zu starten.\\\hline
            Akzeptanz &Auf der Skriptübersicht kann ein Benutzer ein Skript durch Klick auf den Startknopf starten lassen. Das Skript wird im Hintergrund ausgeführt und er wird zur Skriptstatusseite weitergeleitet\\\hline
            Story Points&15\\\hline
            Entwickler &Jonas\\\hline
            Iteration &5\\\hline
            Stunden  &30\\\hline
            Velocity &0.5 SP\slash h\\\hline
        \end{tabulary}
    \end{minipage}
\end{table}



\begin{table}[htbp]
    \begin{minipage}{\linewidth}
        \setlength{\tymax}{0.5\linewidth}
        \centering
        \small
        \begin{tabulary}{\textwidth}{|l|p{10cm}|} \hline
            ID   &32\\\hline
            Name  &Status eines Startvorgangs\\\hline
            Beschreibung&Als Benutzer will ich die Möglichkeit haben, den momentanen Status des ``Startvorganges'' einzusehen. Des Weiteren soll der aktuelle Fortschritt der einzelnen Skriptstufen angezeigt werden.\\\hline
            Akzeptanz &Für jede Stufe wird ein Dropdown-Menü angezeigt. Das Menü der derzeit ausgeführten Stufe wird automatisch geöffnet. In diesem Menü wird jedes Programm und jede Datei angezeigt, welche/s Teil dieser Stufe ist. Falls ein Fehler während des Vorgangs auftritt, wird korrekt angezeigt, welches Programm \slash  welche Datei ihn erzeugt hat.\\\hline
            Story Points&10\\\hline
            Entwickler &Tim\\\hline
            Iteration &9\\\hline
            Stunden  &10\\\hline
            Velocity &1 SP\slash h\\\hline
        \end{tabulary}
    \end{minipage}
\end{table}



\begin{table}[htbp]
    \begin{minipage}{\linewidth}
        \setlength{\tymax}{0.5\linewidth}
        \centering
        \small
        \begin{tabulary}{\textwidth}{|l|p{10cm}|} \hline
            ID   &33\\\hline
            Name  &Beenden eines Programmes\\\hline
            Beschreibung&Als Benutzer will ich die Möglichkeit haben, ein bereits registriertes und gestartetes Programm beenden zu können.\\\hline
            Akzeptanz &Nachdem der Benutzer das Programm seiner Wahl beendet hat, in dem er den entsprechenden Knopf betätigt hat, ändert sich der Status des Programmes und der Benutzer bekommt eine Rückmeldung über den Endstatus des Programmes.\\\hline
            Story Points&8\\\hline
            Entwickler &Tim\\\hline
            Iteration &4\\\hline
            Stunden  &9\\\hline
            Velocity &0.888 SP\slash h\\\hline
        \end{tabulary}
    \end{minipage}
\end{table}



\begin{table}[htbp]
    \begin{minipage}{\linewidth}
        \setlength{\tymax}{0.5\linewidth}
        \centering
        \small
        \begin{tabulary}{\textwidth}{|l|p{10cm}|} \hline
            ID   &34\\\hline
            Name  &Startzeit eines Programmes\\\hline
            Beschreibung&Als Benutzer will ich die Möglichkeit haben, für eine Programm einzutragen, wie lange dieses zum Starten braucht, um das gleichzeitige Starten zweier Programme, die Abhängigkeiten zueinander besitzen, zu verhindern.\\\hline
            Akzeptanz &Beim Anlegen und beim Editieren eines Programmes kann der Nutzer über das "`Startzeit"' Feld die Startzeit des Programms in Sekunden angeben.\\\hline
            Story Points&1\\\hline
            Entwickler &Jonas\\\hline
            Iteration &4\\\hline
            Stunden  &1\\\hline
            Velocity &1 SP\slash h\\\hline
        \end{tabulary}
    \end{minipage}
\end{table}



\begin{table}[htbp]
    \begin{minipage}{\linewidth}
        \setlength{\tymax}{0.5\linewidth}
        \centering
        \small
        \begin{tabulary}{\textwidth}{|l|p{10cm}|} \hline
            ID   &35\\\hline
            Name  &Überprüfen von Argumenten in Programmen\\\hline
            Beschreibung&Als Benutzer möchte ich beim Eintragen von Argumenten eines Programmes darauf hingewiesen werden, wenn ein Argument Sonderzeichen enthält, welche vom System nicht unterstützt werden.\\\hline
            Akzeptanz &Nachdem der Benutzer eine Argumentenliste mit Sonderzeichen in das entsprechenden Feld während dem Hinzufügen oder dem Editieren eines Programmes eingetragen und bestätigt hat, erscheint eine entsprechende Fehlermeldung und die Änderung wird nicht abgespeichert.\\\hline
            Story Points&1\\\hline
            Entwickler &Tim\\\hline
            Iteration &4\\\hline
            Stunden  &1\\\hline
            Velocity &1 SP\slash h\\\hline
        \end{tabulary}
    \end{minipage}
\end{table}



\begin{table}[htbp]
    \begin{minipage}{\linewidth}
        \setlength{\tymax}{0.5\linewidth}
        \centering
        \small
        \begin{tabulary}{\textwidth}{|l|p{10cm}|} \hline
            ID   &36\\\hline
            Name  &Dateien vom Server herunterladen\\\hline
            Beschreibung&Als Benutzer will ich die Möglichkeit haben, Dateien, die auf dem Server in einem bestimmten Ordner liegen, herunterzuladen.\\\hline
            Akzeptanz &Über einen Eintrag in der Navigationsleiste kann der Benutzer auf die Downloadseite wechseln. Dort werden ihm alle Dateien in einem vorher eingestellten Ordner mit Namen und Größe angezeigt. Über einen Klick auf den Dateinamen kann die Datei heruntergeladen werden.\\\hline
            Story Points&3\\\hline
            Entwickler &Tim\\\hline
            Iteration &4\\\hline
            Stunden  &3\\\hline
            Velocity &1 SP\slash h\\\hline
        \end{tabulary}
    \end{minipage}
\end{table}



\begin{table}[htbp]
    \begin{minipage}{\linewidth}
        \setlength{\tymax}{0.5\linewidth}
        \centering
        \small
        \begin{tabulary}{\textwidth}{|l|p{10cm}|} \hline
            ID   &37\\\hline
            Name  &Anzeige von Programmausgaben\\\hline
            Beschreibung&Als Benutzer will ich die Möglichkeit haben, die Ausgaben eines Programmes während und nach seiner Ausführung im Webinterface einzusehen.\\\hline
            Akzeptanz &Nachdem der Benutzer den ensprechenden Knopf neben einem Programmeintrag gedrückt hat, wird unter dem Programmeintrag die zuletzt gespeicherte Logdatei in einem Textfeld angezeigt.\\\hline
            Story Points&30\\\hline
            Entwickler &Tim\\\hline
            Iteration &7\\\hline
            Stunden  &40\\\hline
            Velocity &0.75 SP\slash h\\\hline
        \end{tabulary}
    \end{minipage}
\end{table}



\begin{table}[htbp]
    \begin{minipage}{\linewidth}
        \setlength{\tymax}{0.5\linewidth}
        \centering
        \small
        \begin{tabulary}{\textwidth}{|l|p{10cm}|} \hline
            ID   &38\\\hline
            Name  &Dateibewegung in Skripten\\\hline
            Beschreibung&Als Benutzer will ich die Möglichkeit haben, auch Dateien mithilfe von Skripten verschieben zu lassen.\\\hline
            Akzeptanz &Nachdem der Benutzer das Skript gestartet hat, werden auch die vorher spezifizierten Dateien bewegt. Dies geschieht zur vorher spezifizierten Stufe des Skriptdurchlaufs. Falls ein kritischer Fehler bei der Bewegung von Dateien auftritt, wird die Ausführung des Skripts gestoppt.\\\hline
            Story Points&5\\\hline
            Entwickler &Jonas\\\hline
            Iteration &7\\\hline
            Stunden  &12\\\hline
            Velocity &0.41 SP\slash h\\\hline
        \end{tabulary}
    \end{minipage}
\end{table}



\begin{table}[htbp]
    \begin{minipage}{\linewidth}
        \setlength{\tymax}{0.5\linewidth}
        \centering
        \small
        \begin{tabulary}{\textwidth}{|l|p{10cm}|} \hline
            ID   &39\\\hline
            Name  &Skripte duplizieren\\\hline
            Beschreibung&Als Benutzer will ich die Möglichkeit haben, ein bereits erstelltes Skript duplizieren, editieren und abspeichern zu können.\\\hline
            Akzeptanz &Nachdem der Benutzer den entsprechenden Knopf für das Duplizieren gedrückt hat, wird die Kopie in einem separaten Menüeintrag angezeigt und im Editor geöffnet. Der Name der Kopie ist der originale Name mit dem Suffix ``\_copy''. Namenskonflikte werden mit aufsteigenden Zahlen als Zusatz zum Suffix gehandhabt.\\\hline
            Story Points&3\\\hline
            Entwickler &Tim\\\hline
            Iteration &7\\\hline
            Stunden  &1.5\\\hline
            Velocity &2 SP\slash h\\\hline
        \end{tabulary}
    \end{minipage}
\end{table}



\begin{table}[htbp]
    \begin{minipage}{\linewidth}
        \setlength{\tymax}{0.5\linewidth}
        \centering
        \small
        \begin{tabulary}{\textwidth}{|l|p{10cm}|} \hline
            ID   &40\\\hline
            Name  &Warnung vor Datenverlust\\\hline
            Beschreibung&Benutzer sollen gewarnt werden, wenn sie Dialoge bzw. Browsertabs schließen, welche ungespeicherte Daten enthalten.\\\hline
            Akzeptanz &Der Benutzer wird mit einem Pop-Up Dialog gewarnt, sollte seine Aktion eine ungespeicherte Eingabe verwerfen. Erst nach expliziter Bestätigung, wird die Aktion durchgeführt.\\\hline
            Story Points&5\\\hline
            Entwickler &Heiko\\\hline
            Iteration &7\\\hline
            Stunden  &4\\\hline
            Velocity &1.25 SP\slash h\\\hline
        \end{tabulary}
    \end{minipage}
\end{table}



\begin{table}[htbp]
    \begin{minipage}{\linewidth}
        \setlength{\tymax}{0.5\linewidth}
        \centering
        \small
        \begin{tabulary}{\textwidth}{|l|p{10cm}|} \hline
            ID   &41\\\hline
            Name  &Skript stoppen\\\hline
            Beschreibung&Als Benutzer will ich die Möglichkeit haben, ein Skript während der Ausführung zu stoppen.\\\hline
            Akzeptanz &Nachdem der Benutzer den entsprechenden Knopf betätigt hat, wird das Ausführen des Skripts gestoppt. Bereits geschehene Änderungen (gestartete Programme und bewegte Dateien) werden nicht rückgängig gemacht.\\\hline
            Story Points&3\\\hline
            Entwickler &Frederik\\\hline
            Iteration &8\\\hline
            Stunden  &2\\\hline
            Velocity &1.5 SP\slash h\\\hline
        \end{tabulary}
    \end{minipage}
\end{table}



\begin{table}[htbp]
    \begin{minipage}{\linewidth}
        \setlength{\tymax}{0.5\linewidth}
        \centering
        \small
        \begin{tabulary}{\textwidth}{|l|p{10cm}|} \hline
            ID   &42\\\hline
            Name  &Starten eines Skriptes\\\hline
            Beschreibung&Als Benutzer will ich die Möglichkeit haben, beim Aufruf einer bestimmten Seite automatisch ein Skript starten zu lassen.\\\hline
            Akzeptanz &Sobald man die Seite aufruft wird ein sichtbarer Timer angezeigt. Sobald dieser abgelaufen ist, wird das Defaultskript gestartet.\\\hline
            Story Points&4\\\hline
            Entwickler &Jonas\\\hline
            Iteration &9\\\hline
            Stunden  &3\\\hline
            Velocity &1.3 SP\slash h\\\hline
        \end{tabulary}
    \end{minipage}
\end{table}
\todo[inline]{Ähnlichkeit mit US 42}



\begin{table}[htbp]
    \begin{minipage}{\linewidth}
        \setlength{\tymax}{0.5\linewidth}
        \centering
        \small
        \begin{tabulary}{\textwidth}{|l|p{10cm}|} \hline
            ID   &43\\\hline
            Name  &Alle Programme stoppen\\\hline
            Beschreibung&Als Benutzer will ich die Möglichkeit haben, alle laufende Programme auf allen Clients zu stoppen.\\\hline
            Akzeptanz &Nachdem der Benutzer den entsprechenden Knopf betätigt hat, werden alle derzeit laufenden Programme beendet. Verschobene Dateien werden dabei nicht zurückgesetzt. Falls ein Skript ausgeführt wird, wird es beendet.\\\hline
            Story Points&4\\\hline
            Entwickler &Frederik\\\hline
            Iteration &9\\\hline
            Stunden  &3\\\hline
            Velocity &1.33 SP\slash h\\\hline
        \end{tabulary}
    \end{minipage}
\end{table}



\begin{table}[htbp]
    \begin{minipage}{\linewidth}
        \setlength{\tymax}{0.5\linewidth}
        \centering
        \small
        \begin{tabulary}{\textwidth}{|l|p{10cm}|} \hline
            ID   &44\\\hline
            Name  &Alle Dateien zurücksetzen\\\hline
            Beschreibung&Als Benutzer will ich die Möglichkeit haben, alle verschobenen Dateien zu entfernen und den Ursprungszustand aller Clients wiederherzustellen.\\\hline
            Akzeptanz &Nachdem der Benutzer den entsprechenden Knopf betätigt hat, werden alle derzeit verschobenen Dateien entfernt. Wurde eine Datei beim ursprünglichen Kopieren ersetzt, wird sie wieder hergestellt werden. Ein laufendes Skript wird ebenfalls beendet.\\\hline
            Story Points&4\\\hline
            Entwickler &Frederik\\\hline
            Iteration &9\\\hline
            Stunden  &3.5\\\hline
            Velocity &1.14 SP\slash h\\\hline
        \end{tabulary}
    \end{minipage}
\end{table}



\begin{table}[htbp]
    \begin{minipage}{\linewidth}
        \setlength{\tymax}{0.5\linewidth}
        \centering
        \small
        \begin{tabulary}{\textwidth}{|l|p{10cm}|} \hline
            ID   &45\\\hline
            Name  &Hilfstexte für Formularfelder\\\hline
            Beschreibung&Als Benutzer will ich die Möglichkeit haben, mir zu allen Formularfeldern einen kleinen Hilfetext anzeigen zu lassen.\\\hline
            Akzeptanz &Nachdem der Nutzer auf das Hilfsicon, welches sich neben jedem Formularfeld befindet, geklickt hat, öffnet sich ein Popup mit einem Hinweistext\\\hline
            Story Points&4\\\hline
            Entwickler &Leonardo\\\hline
            Iteration &8\\\hline
            Stunden  &4\\\hline
            Velocity &1 SP\slash h\\\hline
        \end{tabulary}
    \end{minipage}
\end{table}



\begin{table}[htbp]
    \begin{minipage}{\linewidth}
        \setlength{\tymax}{0.5\linewidth}
        \centering
        \small
        \begin{tabulary}{\textwidth}{|l|p{10cm}|} \hline
            ID   &46\\\hline
            Name  &Zentrales Ausschaltmenü\\\hline
            Beschreibung&Als Benutzer möchte ich alle Aktionen, welche auf allen Clients ausgeführt werden können (alle Programme und Dateien zurücksetzen, alle Clients herunterfahren) über ein zentrales Menü steuern können.\\\hline
            Akzeptanz &Nachdem der Nutzer auf das Schutdownicon geklickt hat, öffnet sich ein Dropdown mit allen verfügbaren Auswahlmöglichkeiten\\\hline
            Story Points&1\\\hline
            Entwickler &Leonardo\\\hline
            Iteration &9\\\hline
            Stunden  &1\\\hline
            Velocity &1 SP\slash h\\\hline
        \end{tabulary}
    \end{minipage}
\end{table}



\begin{table}[htbp]
    \begin{minipage}{\linewidth}
        \setlength{\tymax}{0.5\linewidth}
        \centering
        \small
        \begin{tabulary}{\textwidth}{|l|p{10cm}|} \hline
            ID   &47\\\hline
            Name  &Alles ausschalten\\\hline
            Beschreibung&Als Benutzer will ich die Möglichkeit haben, alle laufenden Clients und den Server über das Webinterface auszuschalten.\\\hline
            Akzeptanz &Nachdem der Benutzer den entsprechenden Knopf betätigt hat, werden alle Clients und der Server heruntergefahren. Falls ein Skript läuft, wird dieses beendet, verschobene Dateien werden zurückgesetzt und laufende Programme werden beendet.\\\hline
            Story Points&8\\\hline
            Entwickler &Frederik, Heiko\\\hline
            Iteration &9\\\hline
            Stunden  &20\\\hline
            Velocity &0.4 SP\slash h\\\hline
        \end{tabulary}
    \end{minipage}
\end{table}



\begin{table}[htbp]
    \begin{minipage}{\linewidth}
        \setlength{\tymax}{0.5\linewidth}
        \centering
        \small
        \begin{tabulary}{\textwidth}{|l|p{10cm}|} \hline
            ID   &48\\\hline
            Name  &Standardskript setzen\\\hline
            Beschreibung&Als Benutzer will ich die Möglichkeit haben, ein Skript festzulegen, welches automatisch beim nächsten Start des Servers gestartet wird.\\\hline
            Akzeptanz &Nachdem der Benutzer den entsprechen Knopf gedrückt hat, wird bei jedem Start des Servers das ausgewählte Skript gestartet.\\\hline
            Story Points&5\\\hline
            Entwickler &Jonas\\\hline
            Iteration &9\\\hline
            Stunden  &3\\\hline
            Velocity &1,6 SP\slash h\\\hline
        \end{tabulary}
    \end{minipage}
\end{table}



\begin{table}[htbp]
    \begin{minipage}{\linewidth}
        \setlength{\tymax}{0.5\linewidth}
        \centering
        \small
        \begin{tabulary}{\textwidth}{|l|p{10cm}|} \hline
            ID   &49\\\hline
            Name  &Automatische Neuverbindung eines Clients\\\hline
            Beschreibung&Als Nutzer möchte ich nicht einen Client neu starten müssen, nur weil er die Verbindung mit dem Server verloren hat. Ein Client soll selbst versuchen die Verbindung neu aufzubauen.\\\hline
            Akzeptanz &Nachdem der Client die Verbindung verloren hat, versucht dieser die Verbindung neu aufzubauen. Dies wird solange wiederholt, bis eine Verbindung zum Server besteht. Danach ist der Client wieder uneingeschränkt nutzbar.\\\hline
            Story Points&3\\\hline
            Entwickler &Tim\\\hline
            gteration &keine\\\hline
            Stunden  &2\\\hline
            Velocity &1.5 SP\slash h\\\hline
            Bemerkung & Diese Userstory wurde in Absprache mit den Auftraggebern zwar implementiert, aber nicht in die Software integriert, da der Server der Belastung durch die erhöhte Anfragemenge nicht gewachsen war. Eine Implementierung im Rahmen des Projekts war nicht möglich. Dennoch wurde der Code zur Verfügung gestellt, sodass nachfolgende Entwicklern eine Referenzimplementierung zur Verfügung steht.\\\hline
        \end{tabulary}
    \end{minipage}
\end{table}
