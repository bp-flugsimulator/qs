\section{User Stories}
Es folgen die User Stories des Projekts. Die IDs 13 und 18 existieren nicht, da sie, 
bevor eine Implementierung begonnen wurde, von den Auftraggebern verworfen wurden.

\vspace{5em}

\begin{table}[htbp]
    \begin{minipage}{\linewidth}
        \setlength{\tymax}{0.5\linewidth}
        \centering
        \small
        \begin{tabulary}{\textwidth}{|l|p{10cm}|} \hline
            ID   & 1 \\\hline
            Name  & Willkommensnachricht\\\hline
	    Beschreibung& Als Benutzer möchte ich eine Willkommensnachricht angezeigt bekommen, wenn ich auf die Startseite der Software navigiere. \\\hline
	    Akzeptanz &Nachdem der Benutzer auf die Startseite der Software mit seinem Browser navigiert, gibt der Server eine Webseite mit dem Inhalt ``welcome'' zurück.\\\hline
            Story Points&10\\\hline
            Entwickler &Heiko\\\hline
            Iteration &1\\\hline
            Stunden  &2\\\hline
            Velocity &5 SP\slash h\\\hline
	    Bemerkung &Der Server wurde in dieser User Story aufgesetzt.\\\hline
        \end{tabulary}
    \end{minipage}
\end{table}



\begin{table}[htbp]
    \begin{minipage}{\linewidth}
        \setlength{\tymax}{0.5\linewidth}
        \centering
        \small
        \begin{tabulary}{\textwidth}{|l|p{10cm}|} \hline
            ID   &2\\\hline
            Name  &Client registrieren\\\hline
            Beschreibung&Als Benutzer muss ich die Möglichkeit haben, einen Client zu registrieren, um diesen später verwalten zu können.\\\hline
	    Akzeptanz &Durch das Betätigen des entsprechenden Knopfes öffnet sich ein Dialog. In diesem Dialog befinden sich Felder für "`Name"', "`IP-Adresse"' und "`MAC-Adresse"', in die der Benutzer die Werte für den anzulegenden Client eintragen kann. Ungültige Eingaben werden abgewiesen und der Benutzer wird darauf hingewiesen. Nach korrekter Eingabe, wird der Client in der Datenbank gespeichert.\\\hline
            Story Points&8\\\hline
            Entwickler &Tim, Jonas, Frederik\\\hline
            Iteration &1\\\hline
            Stunden  &30\\\hline
            Velocity &0.26 SP\slash h\\\hline
        \end{tabulary}
    \end{minipage}
\end{table}



\begin{table}[htbp]
    \begin{minipage}{\linewidth}
        \setlength{\tymax}{0.5\linewidth}
        \centering
        \small
        \begin{tabulary}{\textwidth}{|l|p{10cm}|} \hline
            ID   &3\\\hline
            Name  &Hochfahren eines Clients\\\hline
	    Beschreibung&Als Benutzer will ich die Möglichkeit haben, einen Client mit der Hilfe des Webinterfaces hochzufahren. \\\hline
	    Akzeptanz &Nachdem der Benutzer den entsprechenden Knopf betätigt hat, wird der Client gestartet und der Benutzer über diesen Vorgang benachrichtigt.\\\hline
            Story Points&3\\\hline
            Entwickler &Heiko\\\hline
            Iteration &2\\\hline
            Stunden  &7\\\hline
            Velocity &0.42 SP\slash h\\\hline
            Bemerkung &Der Benutzer muss eine Autostart Datei anlegen, welche die Client Software startet. Damit sich der Client mit dem Server automatisch verbindet.\\\hline
        \end{tabulary}
    \end{minipage}
\end{table}


\begin{table}[htbp]
    \begin{minipage}{\linewidth}
        \setlength{\tymax}{0.5\linewidth}
        \centering
        \small
        \begin{tabulary}{\textwidth}{|l|p{10cm}|} \hline
            ID   &4\\\hline
            Name  &Status eines Clients\\\hline
	    Beschreibung&Als Benutzer will ich den aktuellen Status (verbunden, nicht verbunden) jedes bereits registrierten Clients einsehen können.\\\hline
	    Akzeptanz &Der Status wird korrekt angezeigt und aktualisiert, sollte sich der reale Status des Clients ändert. \\\hline
            Story Points&4\\\hline
            Entwickler &Tim\\\hline
            Iteration &4\\\hline
            Stunden  &6\\\hline
            Velocity &0.66 SP\slash h\\\hline
        \end{tabulary}
    \end{minipage}
\end{table}


\begin{table}[htbp]
    \begin{minipage}{\linewidth}
        \setlength{\tymax}{0.5\linewidth}
        \centering
        \small
        \begin{tabulary}{\textwidth}{|l|p{10cm}|} \hline
            ID   &5\\\hline
            Name  &Client herunterfahren\\\hline
	    Beschreibung&Als Benutzer will ich einen Client per Webinterface herunterfahren können.\\\hline
	    Akzeptanz &Nachdem der Benutzer den entsprechenden Knopf betätigt hat, wird der Client heruntergefahren. Auf dem Client laufende Programme und verschobene Dateien werden nicht gestoppt bzw. zurückgesetzt. \\\hline
            Story Points&5\\\hline
            Entwickler &Tim\\\hline
            Iteration &3\\\hline
            Stunden  &4\\\hline
            Velocity &1.25 SP\slash h\\\hline
        \end{tabulary}
    \end{minipage}
\end{table}



\begin{table}[htbp]
    \begin{minipage}{\linewidth}
        \setlength{\tymax}{0.5\linewidth}
        \centering
        \small
        \begin{tabulary}{\textwidth}{|l|p{10cm}|} \hline
            ID   &6\\\hline
            Name  &Kontextmenü\\\hline
	    Beschreibung&Als Benutzer will ich alle möglichen Aktionen (starten, stoppen, bearbeiten, löschen, etc.), welche ich für einen Client ausführen kann, in einem Menü auswählen können.\\\hline
	    Akzeptanz &Nach einem Klick auf einen Client in der Clientübersicht werden die zugehörigen Aktionen in einem Menü angezeigt.\\\hline
            Story Points&10\\\hline
            Entwickler &Leonardo, Jonas, Tim\\\hline
            Iteration &1\\\hline
            Stunden  &10\\\hline
            Velocity &1 SP\slash h\\\hline
            Bemerkung & Die einzelnen Aktionen sind nicht mehr in einem Menü zusammengefasst.\\\hline
        \end{tabulary}
    \end{minipage}
\end{table}



\begin{table}[htbp]
    \begin{minipage}{\linewidth}
        \setlength{\tymax}{0.5\linewidth}
        \centering
        \small
        \begin{tabulary}{\textwidth}{|l|p{10cm}|} \hline
            ID   &7\\\hline
            Name  &Starten eines Programmes\\\hline
	    Beschreibung&Als Benutzer will ich die Möglichkeit haben, ein, vorher über die Weboberfläche angelegtes, Programm auf einem Client starten zu können. \\\hline
	    Akzeptanz &Nachdem der Benutzer in der Clientübersicht den entsprechenden Client ausgewählt hat, kann er auf den Startknopf neben dem Programmnamen klicken. Daraufhin wird ein Startbefehl an den Client gesendet und der Nutzer wird per Benachrichtigung über diesen Vorgang informiert.\\\hline
            Story Points&25\\\hline
            Entwickler &Jonas, Tim\\\hline
            Iteration &3\\\hline
            Stunden  &50\\\hline
            Velocity &0.5 SP\slash h\\\hline
        \end{tabulary}
    \end{minipage}
\end{table}



\begin{table}[htbp]
    \begin{minipage}{\linewidth}
        \setlength{\tymax}{0.5\linewidth}
        \centering
        \small
        \begin{tabulary}{\textwidth}{|l|p{10cm}|} \hline
            ID   &8\\\hline
            Name  &Statusabfrage von Programmen auf einem Client\\\hline
	    Beschreibung&Als Benutzer will ich die Möglichkeit haben, den Status eines registrierten Programmes (Wird ausgeführt, Ist beendet) einzusehen.\\\hline
	    Akzeptanz &Der Status des Programms ist aktuell, korrekt und wird im Webinterface angezeigt. Das Programm wird grün markiert, wenn es läuft und rot, wenn während der Ausführung ein Fehler aufgetreten ist. Falls das Programm nicht läuft, ist es nicht farblich markiert.\\\hline
            Story Points&4\\\hline
            Entwickler &Tim\\\hline
            Iteration &4\\\hline
            Stunden  &6\\\hline
            Velocity &0.66 SP\slash h\\\hline
        \end{tabulary}
    \end{minipage}
\end{table}



\begin{table}[htbp]
    \begin{minipage}{\linewidth}
        \setlength{\tymax}{0.5\linewidth}
        \centering
        \small
        \begin{tabulary}{\textwidth}{|l|p{10cm}|} \hline
            ID   &9\\\hline
            Name  &Datei bewegen\\\hline
	    Beschreibung&Als Benutzer will ich die Möglichkeit haben, eine auf dem Server registrierte Datei auf einem ebenfalls auf dem Server registrierten Client zu bewegen. 
	    Auch will ich, den Ausgangsstatus wiederherstellen können.\\\hline
            Akzeptanz &Nachdem der Benutzer den entsprechenden Kopf betätigt hat, wird die Datei auf dem Client verschoben und der Benutzer wird darüber benachrichtigt (Erfolg oder Misserfolg). Falls am Zielort bereits eine Datei existiert, wird diese separat gespeichert. Der Knopf ändert nach dem Bewegen seinen Anzeigetext auf ''restore''. Sofern eine Datei bereits bewegt wurde und der Benutzer auf den Zurücksetzen Knopf drückt, wird die Datei entfernt. Dies ist nur möglich, wenn die Datei in der Zwischenzeit nicht verändert wurde. Es ist nicht mögliche die Aktion Datei bewegen auszuführen, falls die Datei bereits bewegt worden ist. Ebenfalls ist es nicht möglich eine Datei zurückzusetzten, wenn sie nicht verschoben, oder bereits zurückgesetzt wurde.
	    Um dies zu ermöglichen, müssen dem Dialog aus User Story 24 die Felder ''Source Type'' und ''Destination Type'' hinzugefügt werden. ''Source Type'' spezifiziert dabei, ob es sich um eine Datei oder einen Ordner handelt und ''Destination Type'' definiert dabei, ob der Name Datei/des Ordners behalten oderr geändert werden soll.\\\hline
            Story Points&10\\\hline
            Entwickler &Leonardo, Jonas\\\hline
            Iteration &7\\\hline
            Stunden  &30\\\hline
            Velocity &0.33 SP\slash h\\\hline
        \end{tabulary}
    \end{minipage}
\end{table}



\begin{table}[htbp]
    \begin{minipage}{\linewidth}
        \setlength{\tymax}{0.5\linewidth}
        \centering
        \small
        \begin{tabulary}{\textwidth}{|l|p{10cm}|} \hline
            ID   &10\\\hline
            Name  &Dateien herunterladen\\\hline
	    Beschreibung&Als Benutzer will ich die Möglichkeit haben, eine Datei vom Server auf den Client zu übertragen.\\\hline
            Akzeptanz &Die Datei wird auf den Client kopiert und kann vom Benutzer von dort aus verschoben werden.\\\hline
            Story Points&-\\\hline
            Entwickler &-\\\hline
            Iteration &-\\\hline
            Stunden  &-\\\hline
            Velocity &-\\\hline
            Bemerkung & Bevor diese User Story durchgeführt werden konnte, haben die Auftraggeber sich gegen diese Funktionalität entschieden. Die Funktionalität wird dennoch in einer anderen Variante bereitgestellt (Siehe User Story 36). \\\hline
        \end{tabulary}
    \end{minipage}
\end{table}



\begin{table}[htbp]
    \begin{minipage}{\linewidth}
        \setlength{\tymax}{0.5\linewidth}
        \centering
        \small
        \begin{tabulary}{\textwidth}{|l|p{10cm}|} \hline
            ID   &11\\\hline
            Name  &Datei Löschen\\\hline
	    Beschreibung&Als Benutzer will ich die Möglichkeit haben, eine Datei auf dem Client zu löschen.\\\hline
	    Akzeptanz &Der Benutzer wird vor dem Löschen mit einem Pop-Up Fenster gewarnt und muss seine gewünschte Aktion bestätigen. Falls er die Aktion bestätigt, wird die korrekte Datei auf dem Client permanent gelöscht, sonst wird die Aktion abgebrochen.\\\hline
            Story Points&4\\\hline
            Entwickler &?\\\hline
            Iteration &?\\\hline
            Stunden  &?\\\hline
            Velocity &?\\\hline
            Bemerkung & Bevor diese User Story durchgeführt werden konnte, haben die Auftraggeber sich gegen diese Funktionalität entschieden.
            \\\hline
        \end{tabulary}
    \end{minipage}
\end{table}



\begin{table}[htbp]
    \begin{minipage}{\linewidth}
        \setlength{\tymax}{0.5\linewidth}
        \centering
        \small
        \begin{tabulary}{\textwidth}{|l|p{10cm}|} \hline
            ID   &12\\\hline
            Name  &Starten eines Webbrowsers beim Softwarestart\\\hline
            Beschreibung&Als Benutzer will ich die Möglichkeit haben einen Webbrowser meiner Wahl automatisch zu öffnen, sobald ich die Server Software starte. Der Webbrowser soll auch auf die Seite des Webinterfaces navigieren, damit ich diese nicht manuell tun muss.\\\hline
            Akzeptanz &Nachdem Starten der Server Software, wird der ausgewählte Webbrowser mit der Startseite des Webinterfaces geöffnet.\\\hline
            Story Points&2\\\hline
            Entwickler &?\\\hline
            Iteration &?\\\hline
            Stunden  &?\\\hline
            Velocity &?\\\hline
            Bemerkung & Auf Wunsch der Auftraggeber wurde diese Funktion über die Autostartfunktionalität des Betriebssystems realisiert.
            \\\hline
        \end{tabulary}
    \end{minipage}
\end{table}

\begin{table}[htbp]
    \begin{minipage}{\linewidth}
        \setlength{\tymax}{0.5\linewidth}
        \centering
        \small
        \begin{tabulary}{\textwidth}{|l|p{10cm}|} \hline
            ID   &14\\\hline
            Name  &Client löschen\\\hline
		Beschreibung&Als Benutzer will die Möglichkeit haben, einen bereits registrierten Client aus dem Webinterface zu entfernen.\\\hline
	    Akzeptanz &Nachdem der Benutzer den entsprechenden Knopf betätigt hat, öffnet sich ein Bestätigungsdialog. Nach der Bestätigung ist der Client nicht mehr für den Benutzer im Webinterface zu sehen und wird aus der Datenbank entfernt. Falls der Benutzer die Aktion nicht bestätigt, wird der Client nicht aus der Datenbank gelöscht und wird weiterhin im Webinterface angezeigt.\\\hline
            Story Points&4\\\hline
            Entwickler &Frederik, Leonardo, Tim\\\hline
            Iteration &1\\\hline
            Stunden  &9\\\hline
            Velocity &0.44 SP\slash h\\\hline
        \end{tabulary}
    \end{minipage}
\end{table}



\begin{table}[htbp]
    \begin{minipage}{\linewidth}
        \setlength{\tymax}{0.5\linewidth}
        \centering
        \small
        \begin{tabulary}{\textwidth}{|l|p{10cm}|} \hline
            ID   &15\\\hline
            Name  &Client bearbeiten\\\hline
	    Beschreibung&Als Benutzer will ich die Möglichkeit haben, die Werte "`Name"', "`IP-Adresse"' und "`MAC-Adresse"' eines bereits registrierten Client zu bearbeiten.\\\hline
	    Akzeptanz &Nachdem der Benutzer den entsprechenden Knopf betätigt hat, öffnet sich ein Dialog, in dem sich die aktuellen Werte des Clients befinden. Sollten die Werte nicht konform zu entsprechenden IP- und MAC- Adressstandards sein, wird der Benutzer während der Eingabe darauf hingewiesen. Sollten die Werte konform sein, werden bei Betätigung des Speichern Knopfes diese in der Datenbank gespeichert. Der Benutzer kann den Vorgang auch abbrechen, in diesem Fall werden die Werte nicht gespeichert. \\\hline
            Story Points&4\\\hline
            Entwickler &Tim\\\hline
            Iteration &1\\\hline
            Stunden  &3\\\hline
            Velocity &1.33 SP\slash h\\\hline
        \end{tabulary}
    \end{minipage}
\end{table}



\begin{table}[htbp]
    \begin{minipage}{\linewidth}
        \setlength{\tymax}{0.5\linewidth}
        \centering
        \small
        \begin{tabulary}{\textwidth}{|l|p{10cm}|} \hline
            ID   &16\\\hline
	    Name  &Client anzeigen\\\hline
	    Beschreibung&Als Benutzer will ich die Möglichkeit haben, einen Client mit seinem Namen, seiner IP-Adresse und seiner MAC-Adresse im Webinterface zu sehen.\\\hline
	    Akzeptanz &Für einen Client befindet sich ein Eintrag mit seinem Namen innerhalb des Webinterfaces in der Clientübersicht. Klickt man auf diesen Eintrag, werden IP- und Mac-Adresse des Clients angezeigt.\\\hline
            Story Points&4\\\hline
            Entwickler &Jonas, Tim\\\hline
            Iteration &1\\\hline
            Stunden  &4\\\hline
            Velocity &1 SP\slash h\\\hline
        \end{tabulary}
    \end{minipage}
\end{table}



\begin{table}[htbp]
    \begin{minipage}{\linewidth}
        \setlength{\tymax}{0.5\linewidth}
        \centering
        \small
        \begin{tabulary}{\textwidth}{|l|p{10cm}|} \hline
            ID   &17\\\hline
	    Name  &Navigation im Webinterface\\\hline
	    Beschreibung&Als Benutzer will ich die Möglichkeit haben, die verschiedenen Seiten über eine Navigationsleiste zu erreichen. \\\hline
	    Akzeptanz &Der Benutzer sieht alle für ihn zugängliche Seiten in der Navigationsleiste und kann zu diesen navigieren, indem er auf den entsprechenden Eintrag klickt.\\\hline
            Story Points&1\\\hline
            Entwickler &Jonas\\\hline
            Iteration &3\\\hline
            Stunden  &1\\\hline
            Velocity &1 SP\slash h\\\hline
        \end{tabulary}
    \end{minipage}
\end{table}


\begin{table}[htbp]
    \begin{minipage}{\linewidth}
        \setlength{\tymax}{0.5\linewidth}
        \centering
        \small
        \begin{tabulary}{\textwidth}{|l|p{10cm}|} \hline
            ID   &19\\\hline
            Name  &Programm anzeigen\\\hline
	    Beschreibung&Als Benutzer will ich die Möglichkeit haben, alle registrierten Programme, zugeordnet zu ihrem jeweiligen Client, einzusehen. Dadurch ist es leichter einen Überblick über die vorhandenen Programme der Clients zu erhalten.\\\hline
	    Akzeptanz &Die Programme werden korrekt zu dem zugehörigen Client angezeigt, dabei wird für jedes registrierte Programm nur sein Name genannt.\\\hline
            Story Points&8\\\hline
            Entwickler &Leonardo, Tim\\\hline
            Iteration &2\\\hline
            Stunden  &6\\\hline
            Velocity &1.3 SP\slash h\\\hline
        \end{tabulary}
    \end{minipage}
\end{table}



\begin{table}[htbp]
    \begin{minipage}{\linewidth}
        \setlength{\tymax}{0.5\linewidth}
        \centering
        \small
        \begin{tabulary}{\textwidth}{|l|p{10cm}|} \hline
            ID   &20\\\hline
            Name  &Programm hinzufügen\\\hline
	    Beschreibung&Als Benutzer will ich die Möglichkeit haben, ein Programm, welches auf einem Client liegt, mit bestimmten Argumenten zu registrieren. Dadurch ist es später möglich Programme über das Webinterface zu verwalten. (starten, bearbeiten, löschen, etc.)\\\hline
	    Akzeptanz &Nachdem der Benutzer den entsprechenden Knopf betätigt hat, öffnet sich ein Dialog, welcher Felder für Programmnamen, Pfad zur ausführbaren Datei und Aufrufargumenten beinhaltet. Falls der Benutzer fehlerhafte Werte angibt, wird ihm dies mitgeteilt. Ansonsten wird das Programm in der Datenbank gespeichert.\\\hline
            Story Points&6\\\hline
            Entwickler &Tim\\\hline
            Iteration &2\\\hline
            Stunden  &5\\\hline
            Velocity &1.2 SP\slash h\\\hline
        \end{tabulary}
    \end{minipage}
\end{table}



\begin{table}[htbp]
    \begin{minipage}{\linewidth}
        \setlength{\tymax}{0.5\linewidth}
        \centering
        \small
        \begin{tabulary}{\textwidth}{|l|p{10cm}|} \hline
            ID   &21\\\hline
            Name  &Programm löschen\\\hline
	    Beschreibung&Als Benutzer will ich die Möglichkeit haben, ein bereits registriertes Programm wieder zu entfernen. Dies ist nötig, falls ein Programm nicht mehr benötigt wird.\\\hline
	    Akzeptanz &Nachdem der Benutzer den entsprechenden Knopf betätigt hat öffnet sich Bestätigungsdialog für den Löschvorgang. Nach der Bestätigung wird das Programm aus der Datenbank entfernt und dem Benutzer im Webinterface nicht mehr angezeigt.\\\hline
            Story Points&4\\\hline
            Entwickler &Frederik\\\hline
            Iteration &2\\\hline
            Stunden  &4,5\\\hline
            Velocity &0.89 SP\slash h\\\hline
        \end{tabulary}
    \end{minipage}
\end{table}



\begin{table}[htbp]
    \begin{minipage}{\linewidth}
        \setlength{\tymax}{0.5\linewidth}
        \centering
        \small
        \begin{tabulary}{\textwidth}{|l|p{10cm}|} \hline
            ID   &22\\\hline
            Name  &Programm bearbeiten\\\hline
	    Beschreibung&Als Benutzer will ich die Möglichkeit haben, ein bereits registriertes Programm zu bearbeiten. Dadurch muss bei einer Änderungen nicht ein neues Programm registriert werden.\\\hline
	    Akzeptanz &Nach dem Betätigen des entsprechenden Knopfes öffnet sich ein Dialog, in welchem die letzte Konfiguration angezeigt wird. Die Änderungen werden nur gespeichert, wenn sie keine fehlerhaften Werte beinhalten. Der Nutzer wird daher auf fehlerhafte Werte hingewiesen.\\\hline
            Story Points&5\\\hline
            Entwickler &Tim\\\hline
            Iteration &3\\\hline
            Stunden  &3\\\hline
            Velocity &1.7 SP\slash h\\\hline
        \end{tabulary}
    \end{minipage}
\end{table}



\begin{table}[htbp]
    \begin{minipage}{\linewidth}
        \setlength{\tymax}{0.5\linewidth}
        \centering
        \small
        \begin{tabulary}{\textwidth}{|l|p{10cm}|} \hline
            ID   &23\\\hline
	    Name  &Datei/Ordner anzeigen\\\hline
	    Beschreibung&Als Benutzer will ich die Möglichkeit haben, eine registrierte Datei, die zu einem Client gehört, oder einen registrierten Ordner, der zu einem Client gehört, im Webinterface einzusehen. Dadurch ist es leichter eine Übersicht über die registrierten Dateien und Ordner zu erhalten\\\hline
	    Akzeptanz &Die Dateien und Ordner werden für jeden Client im Webinterface angezeigt, dabei wird jeweils nur der Name des Ordners oder der Datei angezeigt.\\\hline
            Story Points&3\\\hline
            Entwickler &Frederik\\\hline
            Iteration &3\\\hline
            Stunden  &3\\\hline
            Velocity &1 SP\slash h\\\hline
        \end{tabulary}
    \end{minipage}
\end{table}



\begin{table}[htbp]
    \begin{minipage}{\linewidth}
        \setlength{\tymax}{0.5\linewidth}
        \centering
        \small
        \begin{tabulary}{\textwidth}{|l|p{10cm}|} \hline
            ID   &24\\\hline
	    Name  &Datei/Ordner hinzufügen\\\hline
	    Beschreibung&Als Benutzer will ich die Möglichkeit haben, eine Datei oder einen Ornder für einen Client zu registrieren. Dadurch ist es später möglich Dateien und Ordner über das Webinterface zu verwalten. (bewegen, löschen, editieren, etc.)\\\hline
	    Akzeptanz &Nachdem der Benutzer den entsprechenden Knopf betätigt hat, öffnet sich ein Dialog, welcher Dateiname/Ordnername, Quell- und Zielpfad abfragt. Es werden nur korrekte Eingaben gespeichert. Falls der Benutzer keine korrekte Daten angibt, wird ihm dies beim Betätigen des Speichern Knopfs mitgeteilt und die Daten werden nicht übernommen. Sind die Eingaben korrekt, wird ein Eintrag in der Datenbank erstellt.\\\hline
            Story Points&4\\\hline
            Entwickler &Frederik\\\hline
            Iteration &3\\\hline
            Stunden  &4\\\hline
            Velocity &1 SP\slash h\\\hline
        \end{tabulary}
    \end{minipage}
\end{table}



\begin{table}[htbp]
    \begin{minipage}{\linewidth}
        \setlength{\tymax}{0.5\linewidth}
        \centering
        \small
        \begin{tabulary}{\textwidth}{|l|p{10cm}|} \hline
            ID   &25\\\hline
	    Name  &Datei/Ordner löschen\\\hline
	    Beschreibung&Als Benutzer will ich die Möglichkeit haben, eine registrierte Datei oder einen registrierten Ordner zu löschen.\\\hline
	    Akzeptanz &Nach dem Betätigten des entsprechenden Knopfes öffnet sich ein Bestätigungsdialog für den Löschvorgang. Nur wenn der Benutzer den Löschvorgang bestätigt, wird die registrierte Datei/ der registrierte Ordner aus der Datenbank gelöscht und nicht mehr im Webinterface angezeigt.\\\hline
            Story Points&3\\\hline
            Entwickler &Frederik\\\hline
            Iteration &5\\\hline
            Stunden  &1.5\\\hline
            Velocity &2 SP\slash h\\\hline
        \end{tabulary}
    \end{minipage}
\end{table}



\begin{table}[htbp]
    \begin{minipage}{\linewidth}
        \setlength{\tymax}{0.5\linewidth}
        \centering
        \small
        \begin{tabulary}{\textwidth}{|l|p{10cm}|} \hline
            ID   &26\\\hline
	    Name  &Datei/Ordner editieren\\\hline
	    Beschreibung&Als Benutzer will ich die Möglichkeit haben, eine registrierte Datei oder einen registrierten Ordner zu bearbeiten.\\\hline
	    Akzeptanz &Nach dem Betätigen des Knopfes, öffnet sich ein Dialog mit den momentanen Werten. Nur wenn der Benutzer korrekte Änderungen angibt, werden diese gespeichert. Fehlerhafte Änderungen werden dem Benutzer angezeigt.\\\hline
            Story Points&5\\\hline
            Entwickler &Frederik\\\hline
            Iteration &7\\\hline
            Stunden  &2\\\hline
            Velocity &2.5 SP\slash h\\\hline
        \end{tabulary}
    \end{minipage}
\end{table}



\begin{table}[htbp]
    \begin{minipage}{\linewidth}
        \setlength{\tymax}{0.5\linewidth}
        \centering
        \small
        \begin{tabulary}{\textwidth}{|l|p{10cm}|} \hline
            ID   &27\\\hline
            Name  &Skript anzeigen\\\hline
            Beschreibung&Als Benutzer will ich die Möglichkeit haben ein registriertes Skript im Webinterface einzusehen.\\\hline
	    Akzeptanz &Wenn der Benutzer das Skript auswählt, wird es mit seinem Namen, seinen Programmen und seinen Datei- und Ordneroperationen korrekt im Webinterface angezeigt.\\\hline
            Story Points&15\\\hline
            Entwickler &Jonas\\\hline
            Iteration &3\\\hline
            Stunden  &20\\\hline
            Velocity &0.75 SP\slash h\\\hline
        \end{tabulary}
    \end{minipage}
\end{table}



\begin{table}[htbp]
    \begin{minipage}{\linewidth}
        \setlength{\tymax}{0.5\linewidth}
        \centering
        \small
        \begin{tabulary}{\textwidth}{|l|p{10cm}|} \hline
            ID   &28\\\hline
            Name  &Skript hinzufügen\\\hline
	    Beschreibung&Als Benutzer will ich die Möglichkeit haben, ein Skript hinzuzufügen. In einem Skript kann ich Dateien, Ordner und Programme mit einer Startreihenfolge spezifizieren. Um ein Skript erstellen zu können, muss ein Skripteditor hinzugefügt werden.\\\hline
	    Akzeptanz &Im Skripteditor ist es möglich aus den registrierten Programmen, Dateien und Ordnern zu wählen und den gewählten Elementen eine Startreihenfolge zu geben. Um eine Startreihenfolge zu realisieren, vergibt der Nutzer Startindizes an die Elemente. Ebenfalls muss ein Skript einen Namen besitzen. Falls ein Nutzer ein Skript mit einem bereits vorhandenen Namen erstellt, kann es nicht gespeichert werden und der Nutzer wird darüber informiert. Beim Betätigen des Speichern Knopfes wird das Skript in die Datenbank hinzugefügt. Falls das geänderte Skript Fehler enthält, wird es nicht hinzugefügt und der Nutzer wird über die Fehler informiert.\\\hline
            Story Points&7\\\hline
            Entwickler &Jonas\\\hline
            Iteration &4\\\hline
            Stunden  &5\\\hline
            Velocity &1.4 SP\slash h\\\hline
        \end{tabulary}
    \end{minipage}
\end{table}



\begin{table}[htbp]
    \begin{minipage}{\linewidth}
        \setlength{\tymax}{0.5\linewidth}
        \centering
        \small
        \begin{tabulary}{\textwidth}{|l|p{10cm}|} \hline
            ID   &29\\\hline
            Name  &Skript löschen\\\hline
            Beschreibung&Als Benutzer will ich die Möglichkeit haben ein bereits registrierte Skript zu löschen.\\\hline
	    Akzeptanz &Nachdem der Benutzer den entsprechenden Knopf betätigt, wird ein Bestätigungsdialog für den Löschvorgang geöffnet. Wenn der Benutzer den Löschvorgang bestätigt, wird das Skript gelöscht.\\\hline
            Story Points&5\\\hline
            Entwickler &Jonas\\\hline
            Iteration &4\\\hline
            Stunden  &5\\\hline
            Velocity &1 SP\slash h\\\hline
        \end{tabulary}
    \end{minipage}
\end{table}



\begin{table}[htbp]
    \begin{minipage}{\linewidth}
        \setlength{\tymax}{0.5\linewidth}
        \centering
        \small
        \begin{tabulary}{\textwidth}{|l|p{10cm}|} \hline
            ID   &30\\\hline
            Name  &Skript bearbeiten\\\hline
            Beschreibung&Als Benutzer will ich die Möglichkeit haben, ein bereits registriertes Skript zu bearbeiten.\\\hline
	    Akzeptanz &Zum Bearbeiten eines Skriptes wird der Skripteditor mit den bereits definierten Werten des Skripts geöffnet. Der Benutzer kann die vorhandenen Einträge ändern, oder neue Einträge hinzufügen. Beim Betätigen des Speichern Knopfes wird das Skript in der Datenbank aktualisiert. Falls das geänderte Skript fehler enthält, wird es nicht aktualisiert und der Nutzer wird über die Fehler informiert.\\\hline
            Story Points&5\\\hline
            Entwickler &Leonardo\\\hline
            Iteration &7\\\hline
            Stunden  &7\\\hline
            Velocity &0.71 SP\slash h\\\hline
        \end{tabulary}
    \end{minipage}
\end{table}



\begin{table}[htbp]
    \begin{minipage}{\linewidth}
        \setlength{\tymax}{0.5\linewidth}
        \centering
        \small
        \begin{tabulary}{\textwidth}{|l|p{10cm}|} \hline
            ID   &31\\\hline
            Name  &Starten von Skripten\\\hline
            Beschreibung&Als Benutzer will ich die Möglichkeit haben, ein Skript meiner Wahl zu starten.\\\hline
	    Akzeptanz &Der Benutzer kann ein ausgewähltes Skript durch Betätigen des Startknopfes starten. Es kann maximal ein Skript gleichzeitig ausgeführt werden. Falls ein Skript erfolgreich ausgeführt wurde, wird es automatisch als Standardskript gesetzte. Beim einem Fehler in einem Program wird das Skript gestoppt.\\\hline
            Story Points&15\\\hline
            Entwickler &Jonas\\\hline
            Iteration &5\\\hline
            Stunden  &30\\\hline
            Velocity &0.5 SP\slash h\\\hline
        \end{tabulary}
    \end{minipage}
\end{table}



\begin{table}[htbp]
    \begin{minipage}{\linewidth}
        \setlength{\tymax}{0.5\linewidth}
        \centering
        \small
        \begin{tabulary}{\textwidth}{|l|p{10cm}|} \hline
            ID   &32\\\hline
	    Name  &Status eines Skripts\\\hline
	    Beschreibung&Als Benutzer will ich die Möglichkeit haben, den momentanen Status eines Skripts einzusehen. Des Weiteren soll der aktuelle Fortschritt der einzelnen Skriptstufen angezeigt werden.\\\hline
	    Akzeptanz &Für jede Skriptstufe gibt es ein Dropdown-Menü. Das Menü der derzeit ausgeführten Skriptstufe wird automatisch geöffnet. In diesem Menü wird jedes Programm, jeder Ordner und jede Datei angezeigt, welche/s/r Teil dieser Stufe ist. Falls ein Fehler während des Vorgangs auftritt, wird korrekt angezeigt, welches Programm \slash welche Datei \slash welcher Ordner ihn erzeugt hat.\\\hline
            Story Points&10\\\hline
            Entwickler &Tim\\\hline
            Iteration &9\\\hline
            Stunden  &10\\\hline
            Velocity &1 SP\slash h\\\hline
        \end{tabulary}
    \end{minipage}
\end{table}



\begin{table}[htbp]
    \begin{minipage}{\linewidth}
        \setlength{\tymax}{0.5\linewidth}
        \centering
        \small
        \begin{tabulary}{\textwidth}{|l|p{10cm}|} \hline
            ID   &33\\\hline
            Name  &Beenden eines Programmes\\\hline
	    Beschreibung&Als Benutzer will ich die Möglichkeit haben, ein registriertes und gestartetes Programm beenden zu können.\\\hline
	    Akzeptanz &Nachdem der Benutzer den entsprechenden Knopf betätigt hat, wird der Status des Programmes entsprechend in der Datenbank aktualisiert und der Benutzer bekommt eine Rückmeldung über den Endstatus des Programmes.\\\hline
            Story Points&8\\\hline
            Entwickler &Tim\\\hline
            Iteration &4\\\hline
            Stunden  &9\\\hline
            Velocity &0.88 SP\slash h\\\hline
        \end{tabulary}
    \end{minipage}
\end{table}



\begin{table}[htbp]
    \begin{minipage}{\linewidth}
        \setlength{\tymax}{0.5\linewidth}
        \centering
        \small
        \begin{tabulary}{\textwidth}{|l|p{10cm}|} \hline
            ID   &34\\\hline
            Name  &Startzeit eines Programmes\\\hline
	    Beschreibung&Als Benutzer will ich die Möglichkeit haben, für eine Programm einzutragen, wie lange es zum Starten braucht, um das gleichzeitige Starten zweier Programme, die Abhängigkeiten zueinander besitzen, zu verhindern.\\\hline
	    Akzeptanz &Beim Anlegen und beim Editieren eines Programmes kann der Nutzer über das ''Startzeit'' Feld die Startzeit des Programms in Sekunden angeben. Werte < Null geben an, dass der Startprozess des Programmes erst terminieren muss und sind daher ebenfalls valide. Bei einem Wert >= Null wird nicht auf das Terminieren des Programmes gewartet, sonder nur den angegebenen Wert in Sekunden.\\\hline
            Story Points&1\\\hline
            Entwickler &Jonas\\\hline
            Iteration &4\\\hline
            Stunden  &1\\\hline
            Velocity &1 SP\slash h\\\hline
        \end{tabulary}
    \end{minipage}
\end{table}



\begin{table}[htbp]
    \begin{minipage}{\linewidth}
        \setlength{\tymax}{0.5\linewidth}
        \centering
        \small
        \begin{tabulary}{\textwidth}{|l|p{10cm}|} \hline
            ID   &35\\\hline
            Name  &Überprüfen von Argumenten in Programmen\\\hline
            Beschreibung&Als Benutzer möchte ich beim Eintragen von Argumenten eines Programmes darauf hingewiesen werden, wenn ein Argument Sonderzeichen enthält, welche vom System nicht unterstützt werden.\\\hline
	    Akzeptanz &Nachdem der Benutzer eine Argumentenliste mit Sonderzeichen in das entsprechenden Feld, während dem Hinzufügen oder dem Editieren eines Programmes eingetragen und bestätigt hat, erscheint eine entsprechende Fehlermeldung und die Änderung wird nicht abgespeichert.\\\hline
            Story Points&1\\\hline
            Entwickler &Tim\\\hline
            Iteration &4\\\hline
            Stunden  &1\\\hline
            Velocity &1 SP\slash h\\\hline
        \end{tabulary}
    \end{minipage}
\end{table}



\begin{table}[htbp]
    \begin{minipage}{\linewidth}
        \setlength{\tymax}{0.5\linewidth}
        \centering
        \small
        \begin{tabulary}{\textwidth}{|l|p{10cm}|} \hline
            ID   &36\\\hline
            Name  &Dateien vom Server herunterladen\\\hline
            Beschreibung&Als Benutzer will ich die Möglichkeit haben, Dateien, die auf dem Server in einem bestimmten Ordner liegen, herunterzuladen.\\\hline
	    Akzeptanz &Über einen Eintrag in der Navigationsleiste kann der Benutzer auf die Downloadseite wechseln. Dort werden ihm alle Dateien in einem vorher festgelegten Ordner mit Namen und Größe angezeigt. Über einen Klick auf den Dateinamen kann die Datei heruntergeladen werden.\\\hline
            Story Points&3\\\hline
            Entwickler &Tim\\\hline
            Iteration &4\\\hline
            Stunden  &3\\\hline
            Velocity &1 SP\slash h\\\hline
        \end{tabulary}
    \end{minipage}
\end{table}



\begin{table}[htbp]
    \begin{minipage}{\linewidth}
        \setlength{\tymax}{0.5\linewidth}
        \centering
        \small
        \begin{tabulary}{\textwidth}{|l|p{10cm}|} \hline
            ID   &37\\\hline
            Name  &Anzeige von Programmausgaben\\\hline
            Beschreibung&Als Benutzer will ich die Möglichkeit haben, die Ausgaben eines Programmes während und nach seiner Ausführung im Webinterface einzusehen.\\\hline
            Akzeptanz &Nachdem der Benutzer den entsprechenden Knopf neben einem Programmeintrag gedrückt hat, wird unter dem Programmeintrag die zuletzt gespeicherte Logdatei in einem Textfeld angezeigt. Wenn das Programm weiteren Text ausgibt wird dieser ebenfalls an die Weboberfläche geschickt und zusätzlich zu der Logdatei im Textfeld angezeigt. Falls keine Logdatei vorhanden ist, kann der Knopf nicht betätigt werden.\\\hline
            Story Points&30\\\hline
            Entwickler &Tim\\\hline
            Iteration &7\\\hline
            Stunden  &40\\\hline
            Velocity &0.75 SP\slash h\\\hline
        \end{tabulary}
    \end{minipage}
\end{table}



\begin{table}[htbp]
    \begin{minipage}{\linewidth}
        \setlength{\tymax}{0.5\linewidth}
        \centering
        \small
        \begin{tabulary}{\textwidth}{|l|p{10cm}|} \hline
            ID   &38\\\hline
            Name  &Dateibewegung in Skripten\\\hline
            Beschreibung&Als Benutzer will ich die Möglichkeit haben, auch Dateien mithilfe von Skripten verschieben zu lassen.\\\hline
            Akzeptanz &Nachdem der Benutzer das Skript gestartet hat, werden auch die vorher spezifizierten Dateien bewegt. Dies geschieht zur vorher spezifizierten Stufe des Skriptdurchlaufs. Falls ein kritischer Fehler bei der Bewegung von Dateien auftritt, wird die Ausführung des Skripts gestoppt.\\\hline
            Story Points&5\\\hline
            Entwickler &Jonas\\\hline
            Iteration &7\\\hline
            Stunden  &12\\\hline
            Velocity &0.41 SP\slash h\\\hline
        \end{tabulary}
    \end{minipage}
\end{table}



\begin{table}[htbp]
    \begin{minipage}{\linewidth}
        \setlength{\tymax}{0.5\linewidth}
        \centering
        \small
        \begin{tabulary}{\textwidth}{|l|p{10cm}|} \hline
            ID   &39\\\hline
            Name  &Skripte duplizieren\\\hline
	    Beschreibung&Als Benutzer will ich die Möglichkeit haben, ein bereits erstelltes Skript zu duplizieren.\\\hline
	    Akzeptanz &Nachdem der Benutzer den entsprechenden Knopf für das Duplizieren gedrückt hat, wird das Duplikat in einem separaten Menüeintrag angezeigt und im Editor geöffnet. Der Name des Duplikats ist der Name des Originals mit dem Suffix ``\_copy''. Namenskonflikte werden mit aufsteigenden Zahlen als Zusatz zum Suffix gehandhabt.\\\hline
            Story Points&3\\\hline
            Entwickler &Tim\\\hline
            Iteration &7\\\hline
            Stunden  &1.5\\\hline
            Velocity &2 SP\slash h\\\hline
        \end{tabulary}
    \end{minipage}
\end{table}



\begin{table}[htbp]
    \begin{minipage}{\linewidth}
        \setlength{\tymax}{0.5\linewidth}
        \centering
        \small
        \begin{tabulary}{\textwidth}{|l|p{10cm}|} \hline
            ID   &40\\\hline
            Name  &Warnung vor Datenverlust\\\hline
            Beschreibung&Benutzer sollen gewarnt werden, wenn sie Dialoge bzw. Browsertabs schließen, welche ungespeicherte Daten enthalten.\\\hline
	    Akzeptanz &Der Benutzer wird mit einem Dialog gewarnt, falls seine Aktion ungespeicherte Eingaben verwerfen würde. Erst nach expliziter Bestätigung des Dialogs, wird die Aktion durchgeführt.\\\hline
            Story Points&5\\\hline
            Entwickler &Heiko\\\hline
            Iteration &7\\\hline
            Stunden  &4\\\hline
            Velocity &1.25 SP\slash h\\\hline
        \end{tabulary}
    \end{minipage}
\end{table}



\begin{table}[htbp]
    \begin{minipage}{\linewidth}
        \setlength{\tymax}{0.5\linewidth}
        \centering
        \small
        \begin{tabulary}{\textwidth}{|l|p{10cm}|} \hline
            ID   &41\\\hline
            Name  &Skript stoppen\\\hline
            Beschreibung&Als Benutzer will ich die Möglichkeit haben, ein Skript während der Ausführung zu stoppen.\\\hline
	    Akzeptanz &Nachdem der Benutzer den entsprechenden Knopf betätigt hat, wird das Ausführen des Skripts gestoppt. Bereits geschehene Änderungen (gestartete Programme und bewegte Dateien und Ordner) werden nicht rückgängig gemacht.\\\hline
            Story Points&3\\\hline
            Entwickler &Frederik\\\hline
            Iteration &8\\\hline
            Stunden  &2\\\hline
            Velocity &1.5 SP\slash h\\\hline
        \end{tabulary}
    \end{minipage}
\end{table}



\begin{table}[htbp]
    \begin{minipage}{\linewidth}
        \setlength{\tymax}{0.5\linewidth}
        \centering
        \small
        \begin{tabulary}{\textwidth}{|l|p{10cm}|} \hline
            ID   &42\\\hline
	    Name  &Automatisches Starten eines Skriptes\\\hline
	    Beschreibung&Als Benutzer will ich die Möglichkeit haben, beim Aufruf der Startseite automatisch ein Skript starten zu lassen, falls nicht bereits ein Skript läuft.\\\hline
	    Akzeptanz &Sobald man die Startseite aufruft, wird ein Timer angezeigt. Sobald dieser abgelaufen ist, wird das Standardskript gestartet.\\\hline
            Story Points&4\\\hline
            Entwickler &Jonas\\\hline
            Iteration &9\\\hline
            Stunden  &3\\\hline
            Velocity &1.3 SP\slash h\\\hline
        \end{tabulary}
    \end{minipage}
\end{table}



\begin{table}[htbp]
    \begin{minipage}{\linewidth}
        \setlength{\tymax}{0.5\linewidth}
        \centering
        \small
        \begin{tabulary}{\textwidth}{|l|p{10cm}|} \hline
            ID   &43\\\hline
            Name  &Alle Programme stoppen\\\hline
	    Beschreibung&Als Benutzer will ich die Möglichkeit haben, alle über das Webinterface gestarteten Programme auf allen Clients zu stoppen.\\\hline
	    Akzeptanz &Nachdem der Benutzer den entsprechenden Knopf betätigt hat, werden alle registrierten Programme, die ausgeführt werden, beendet. Falls ein Skript ausgeführt wird, wird es vor dem Stoppen der Programme beendet.\\\hline
            Story Points&4\\\hline
            Entwickler &Frederik\\\hline
            Iteration &9\\\hline
            Stunden  &3\\\hline
            Velocity &1.33 SP\slash h\\\hline
        \end{tabulary}
    \end{minipage}
\end{table}



\begin{table}[htbp]
    \begin{minipage}{\linewidth}
        \setlength{\tymax}{0.5\linewidth}
        \centering
        \small
        \begin{tabulary}{\textwidth}{|l|p{10cm}|} \hline
            ID   &44\\\hline
	    Name  &Alle Dateien und Ordner zurücksetzen\\\hline
	    Beschreibung&Als Benutzer will ich die Möglichkeit haben, alle über das Webinterface auf den Clients verschobenen Dateien und Ordner in ihren Ursprungszustand zu versetzen.\\\hline
	    Akzeptanz &Nachdem der Benutzer den entsprechenden Knopf betätigt hat, werden alle registrierten und derzeit verschobenen Dateien und Ordner zurückgesetzt. Wurde eine Datei beim ursprünglichen Bewegen ersetzt, wird sie wiederhergestellt. Laufende Programme und Skripte werden vor dem Zurücksetzen der Dateien und Ordner gestoppt.\\\hline
            Story Points&4\\\hline
            Entwickler &Frederik\\\hline
            Iteration &9\\\hline
            Stunden  &3.5\\\hline
            Velocity &1.14 SP\slash h\\\hline
        \end{tabulary}
    \end{minipage}
\end{table}



\begin{table}[htbp]
    \begin{minipage}{\linewidth}
        \setlength{\tymax}{0.5\linewidth}
        \centering
        \small
        \begin{tabulary}{\textwidth}{|l|p{10cm}|} \hline
            ID   &45\\\hline
            Name  &Hilfstexte für Formularfelder\\\hline
            Beschreibung&Als Benutzer will ich die Möglichkeit haben, mir zu allen Formularfeldern einen kleinen Hilfetext anzeigen zu lassen.\\\hline
	    Akzeptanz &Nachdem der Nutzer auf das Hilfssymbol, welches sich neben jedem Formularfeld befindet, geklickt hat, öffnet sich ein Popup mit einem Hinweistex.\\\hline
            Story Points&4\\\hline
            Entwickler &Leonardo\\\hline
            Iteration &8\\\hline
            Stunden  &4\\\hline
            Velocity &1 SP\slash h\\\hline
        \end{tabulary}
    \end{minipage}
\end{table}



\begin{table}[htbp]
    \begin{minipage}{\linewidth}
        \setlength{\tymax}{0.5\linewidth}
        \centering
        \small
        \begin{tabulary}{\textwidth}{|l|p{10cm}|} \hline
            ID   &46\\\hline
            Name  &Zentrales Ausschaltmenü\\\hline
	    Beschreibung&Als Benutzer möchte ich die Aktionen ''Stop Programs'', ''Restore Files'', ''Shutdown Clients'' und ''Shutdown'' über ein zentrales Menü wählen können.\\\hline
	    Akzeptanz &Nachdem der Nutzer auf das Menüsymbol geklickt hat, öffnet sich ein Dropdown-Dialog mit allen verfügbaren Aktionen. Ein Benutzer kann diese Aktionen ausführen, indem er sie anklickt.\\\hline
            Story Points&1\\\hline
            Entwickler &Leonardo\\\hline
            Iteration &9\\\hline
            Stunden  &1\\\hline
            Velocity &1 SP\slash h\\\hline
        \end{tabulary}
    \end{minipage}
\end{table}



\begin{table}[htbp]
    \begin{minipage}{\linewidth}
        \setlength{\tymax}{0.5\linewidth}
        \centering
        \small
        \begin{tabulary}{\textwidth}{|l|p{10cm}|} \hline
            ID   &47\\\hline
            Name  &Alles ausschalten\\\hline
	    Beschreibung&Als Benutzer will ich die Möglichkeit haben, alle laufenden Clients und auch den Server über das Webinterface auszuschalten.\\\hline
	    Akzeptanz &Nachdem der Benutzer den entsprechenden Knopf betätigt hat, werden alle Clients und der Server heruntergefahren. Falls ein Skript läuft, wird dieses beendet, verschobene Dateien werden zurückgesetzt und laufende Programme werden beendet. Ebenfalls gibt es einen weiteren Knopf, welcher dieselben Aktionen ohne das Ausschalten des Servers realisiert. \\\hline
            Story Points&8\\\hline
            Entwickler &Frederik, Heiko\\\hline
            Iteration &9\\\hline
            Stunden  &20\\\hline
            Velocity &0.4 SP\slash h\\\hline
        \end{tabulary}
    \end{minipage}
\end{table}



\begin{table}[htbp]
    \begin{minipage}{\linewidth}
        \setlength{\tymax}{0.5\linewidth}
        \centering
        \small
        \begin{tabulary}{\textwidth}{|l|p{10cm}|} \hline
            ID   &48\\\hline
            Name  &Standardskript setzen\\\hline
            Beschreibung&Als Benutzer will ich die Möglichkeit haben, ein Skript festzulegen, welches automatisch beim nächsten Start des Servers gestartet wird.\\\hline
	    Akzeptanz &Nachdem der Benutzer den entsprechen Knopf gedrückt hat, wird das ausgewählte Skript als Standardskript festgelegt.\\\hline
            Story Points&5\\\hline
            Entwickler &Jonas\\\hline
            Iteration &9\\\hline
            Stunden  &3\\\hline
            Velocity &1,6 SP\slash h\\\hline
        \end{tabulary}
    \end{minipage}
\end{table}



\begin{table}[htbp]
    \begin{minipage}{\linewidth}
        \setlength{\tymax}{0.5\linewidth}
        \centering
        \small
        \begin{tabulary}{\textwidth}{|l|p{10cm}|} \hline
            ID   &49\\\hline
            Name  &Automatische Neuverbindung eines Clients\\\hline
            Beschreibung&Als Nutzer möchte ich nicht einen Client neu starten müssen, nur weil er die Verbindung mit dem Server verloren hat. Ein Client soll selbst versuchen die Verbindung neu aufzubauen.\\\hline
            Akzeptanz &Nachdem der Client die Verbindung verloren hat, versucht dieser die Verbindung neu aufzubauen. Dies wird solange wiederholt, bis eine Verbindung zum Server besteht. Danach ist der Client wieder uneingeschränkt nutzbar.\\\hline
            Story Points&3\\\hline
            Entwickler &Tim\\\hline
            gteration &keine\\\hline
            Stunden  &2\\\hline
            Velocity &1.5 SP\slash h\\\hline
            Bemerkung & Diese User Story wurde in Absprache mit den Auftraggebern zwar implementiert, aber nicht in die Software integriert, da der Server der Belastung durch die erhöhte Anfragemenge nicht gewachsen war. Eine Implementierung im Rahmen des Projekts war nicht möglich. Dennoch wurde der Code zur Verfügung gestellt, sodass nachfolgende Entwicklern eine Referenzimplementierung zur Verfügung steht.\\\hline
        \end{tabulary}
    \end{minipage}
\end{table}
