\documentclass[accentcolor=tud9c,12pt,paper=a4]{tudreport}

\usepackage[utf8]{inputenc}
\usepackage{ngerman}
\usepackage{parcolumns}
\usepackage{hyperref}

\newcommand{\titlerow}[2]{
	\begin{parcolumns}[colwidths={1=.17\linewidth}]{2}
		\colchunk[1]{#1:}
		\colchunk[2]{#2}
	\end{parcolumns}
	\vspace{0.2cm}
}

\title{Steuerungsprogramm für einen Flugsimulator}
\subtitle{Qualitätssicherungsdokument}
\subsubtitle{%
	\titlerow{Gruppe 19}{%
		Frederik Bark <frederikalexander.bark@stud.tu-darmstadt.de>\\
		Heiko Carrasco <heiko.carrascohuertas@stud.tu-darmstadt.de>\\
		Jonas Meurer <jonas.meurer@stud.tu-darmstadt.de>\\
		Tim Weißmantel <tim.weissmantel@stud.tu-darmstadt.de>\\
		Leonardo Zaninelli <leonardo.zaninelli@stud.tu-darmstadt.de>}
	\titlerow{Teamleiter}{Hendrik Bode <hendrik.bode@stud.tu-darmstadt.de>}
	\titlerow{Auftraggeber}{%
		Jonas Schulze <Schulze@fsr.tu-darmstadt.de>\\
		Torben Bernatzky <Bernatzky@fsr.tu-darmstadt.de>\\
		Technische Universität Darmstadt\\
		Flugsysteme und Regelungstechnik}
	\titlerow{Abgabedatum}{31.03.2018}}
\institution{Bachelor-Praktikum WS-2017/18\\ Fachbereich Informatik}

\begin{document}

	\maketitle
	\tableofcontents

	\chapter{Einleitung}
		Die Auftraggeber sind im Besitz eines Flugsimulators, der aus vielen 
		Teilprogrammen besteht. Diese Teilprogramme sind über mehrere Computer im
		gleichen Netzwerk verteilt und müssen alle in einer komplexen Reihenfolge per Hand 
		gestartet werden. Aus diesem Grund wünschen sich die Auftraggeber eine Software, welche
		das Starten des Simulators automatisiert.\\[5pt]
		Es ist dabei zu beachten, dass auf den Computern unterschiedliche 
		Betriebsysteme benutzt werden (verschiedene Windowsversionen und ggf. Linux). 
		Der Nutzer soll neue Ablaufpläne selbst erstellen und verändern können,
		weswegen die Programmierung dieser Pläne mithilfe von generischen Komponenten
		wie "`Start eines Programmes"' oder "`Starten eines Computers"' möglich sein sollte.
		\\[5pt]
		Hierzu wird ein Master/Slave-System eingesetzt. Der Master stellt eine Weboberfläche zur
		Verfügung, in welcher der Nutzer Startpläne erstellen und ausführen lassen kann.
		Alle anderen Computer im Flugsimulator verbinden sich als Slaves mit dem Master und
		führen die spezifizierten Befehle aus.
		Für das Projekt wird sowohl die Master- als auch die Clientsoftware
		konzipiert und geschrieben. Zusätzlich wird die Oberfläche erstellt und ein
		Kommunikationsprotokoll definiert, welches die Kommunikation zwischen
		Master und Slaves beschreibt.
		\\[5pt]
		Die Software für Master und Slaves ist in Python geschrieben und benutzt
		das Django Framework für das Webinterface sowie RPC zur Kommunikation zwischen 
		Master und Slaves.
		

	\chapter{Qualitätsziele}
		\section{Portabilität}
		Aufgrund der Heterogenität der verwendenten Betriebssysteme im
		Simulator, muss die Slave Software sowohl unter verschiedenen Windowsversionen
		(XP bis 10) als auch unter Linux laufen.
		Mit den Auftraggebern wurde abgesprochen, dass der Master keine 
		Betriebsystemversion unter Windows 7 einsetzt, weswegen die Tests für Windows
		XP hier entfallen.
		\\[5pt]
		Um Portabilität sicherzustellen, werden alle Unit-Tests auf allen eingesetzten
		Betriebssystemen nach jedem Commit ausgeführt. Durch die 90\% Testabdeckung
		wird somit sichergestellt, dass der Kern der Software unter mehreren Betriebssystemen
		fehlerfrei läuft. Spätestens zur Abgabe einer
		Userstory müssen diese Tests fehlerfrei durchlaufen, sonst wird die Userstory nicht
		zur Abnahme präsentiert. Die Entscheidung zur Präsentation trifft der für diese
		Userstory bestimmte Reviewer. Der/die zuständige/n Entwickler behebt/beheben dann die Fehler
		in der nächsten Iteration.
		Als Testumgebung werden verwendet:
		\begin{itemize}
			\item Coveralls zum Bestimmen der Testabdeckung
			\item Travis CI zum Testen unter Linux
			\item AppVeyor zum Testen für Windowsversionen ab Windows 7
			\item Jenkins zum Testen auf Windows XP
		\end{itemize}
		Um die Installation auf allen Betriebssystemen zu vereinfachen, werden alle benötigten
		Bibliotheken mitgeliefert. Diese werden von den Testumgebungen verwendet, sodass
		auch die Installation auf den verschiedenen Betriebsystemen getestet wird.
		\newpage
		\section{Korrektheit}
		Der Flugsimulator ist ein komplexes System, welches aus mehreren verbundenen Komponenten besteht.
		Sollten zum Beispiel beim Start Fehler auftreten und werden diese nicht korrekt behandelt, so
		ist ein korrekter Start des Simulators nicht möglich. Daher ist es für das Projekt
		sehr wichtig, dass alle Komponenten korrekt spezifiziert und implementiert sind,
		sodass z.B. eventuell auftretende Fehler richtig behandelt werden können.
		\\[5pt]
		Der Prozess zur Sicherstellung dieses QS-Ziels sieht daher wie folgt aus:
		Nach der Implementierung wird die Userstory von mindestens einem weiteren
		Entwickler geprüft. Dazu greift dieser auf die Reports der Testingstools zurück,
		welche unter Portabilität genannt wurden, da es sich um die gleichen Tests handelt.
		Der Reviewer gestellt damit relativ einfach sicher, dass die Testabdeckung des Projekts
		nicht unter 90\% fällt, sonst wird die Userstory nicht zur Abnahme empfohlen und
		muss vom zuständigen Entwickler innerhalb der nächsten Iteration angepasst werden.
		\\[5pt]
		Zudem wird überprüft, ob die Dokumentation aller implementierten
		Codeteile mit dem zugehörigen Code übereinstimmen und ob die Implementierung der
		Userstory entspricht. Dabei wird die angehängte Checkliste als Grundlage verwendet.
		Beispielsweise muss der Code korrekt formatiert sein, keine der automatisierten Tools
		darf negative Rückmeldungen gegeben haben (z.B. Code nicht korrekt getestet, Codeformat
		nicht eingehalten, Tests laufen nicht durch) und der Code ist entsprechend dokumentiert.
		Sollten Punkte nicht eingehalten werden, wird die Userstory nicht den Auftraggebern zur
		Abnahme vorgestellt. Der/die verantwortliche/n Entwickler beheben dann die Anmerkungen.
		Erst wenn alle Punkte der Checkliste erfüllt sind, wird die Userstory den Auftraggebern
		zur Abnahme empfohlen.
		\newpage		
		\section{Bedienbarkeit}
		Damit auch weitere, eventuell nicht technisch versierte, Mitarbeiter die Software
		problemlos verwenden können, muss die 
		Bedienoberfläche verständlich und leicht bedienbar sein. Eine Benutzung wird ohne
		vorherige, längere Einarbeitung möglich sein.
		\\[5pt]
		Durch die Verwendung des Webframeworks Bootstrap\footnote{\url{https://getbootstrap.com/}}
		wird das Design der Seite vereinheitlicht. Dies wird durch die Checkliste (im Anhang)
		vor der Präsentation einer Userstory für die Auftraggeber durch den Reviewer überprüft. 
		Die Icons\footnote{\url{https://material.io/icons/}}
		stammen aus dem Material Design, welches zur Zeit auf über 79.8\% der Android Geräte
		verwendet wird\footnote{\url{https://developer.android.com/about/dashboards/index.html#Platform}}. 
		Damit ist es einer Vielzahl an Nutzern bereits bekannt.
		\\[5pt]
		Um den Einstieg zu erleichtern erstellt das BP-Team eine Bedienungsanleitung (siehe Anhang),
		welche die Hauptfunktionalitäten der Software beschreibt und erklärt. Mit der Erstellung
		der Anleitung wird Anfang Februar begonnen, da dann ein Großteil der Funktionalität
		implementiert ist.
		\\[5pt]
		Da die Bedienbarkeit nicht objektiv messbar ist, werden zusätzlich Nutzerstudien durchgeführt.
		Diese sind für Anfang Februar und Anfang März 2018 angesetzt.
		Die Testkandidaten kommen vorzugsweise aus dem
		Fachgebiet des Arbeitgebers, da diese am häufigsten mit der Automatisierungssoftware interagieren werden.
		Zufällig ausgewählte Kandidaten aus anderen Fachbereichen werden ebenfalls um Teilnahme gebeten, da
		so verglichen werden kann, wie benutzerfreundlich die Software auf Menschen wirkt, welche nicht täglich
		mit Flugsimulatoren arbeiten.
		\\[5pt]
		Für die Studie erhalten die Nutzer eine Auswahl an Aufgaben (siehe Anhang), 
		welche sie zunächst ohne Anleitung durchführen sollen. 
		Gibt es bei einer Aufgabe Probleme, so wird die Bedienungsanleitung 
		der Software zur Verfügung gestellt.
		Am Ende wird ein Fragebogen ausgehändigt (ebenfalls im Anhang),
		in welchem die Nutzer das Design (z.B. Farbschema, Größe und Beschriftung von Buttons), den
		durch die Software vorgegeben Arbeitsablauf sowie die Anleitung bewerten können. Zudem
		besteht die Möglichkeit allgemeines Feedback abzugeben.
		\\[5pt]
		Mithilfe der ausgefüllten Fragebögen erstellt das Team einen Katalog an
		Maßnahmen bezüglich des Designs, Workflows oder der Anleitung, 
		welche den Auftraggebern beim nächsten Treffen präsentiert werden. 
		Diese entscheiden, welche Maßnahmen umgesetzt werden, woraufhin ein
		Entwickler diese in der nächsten Iteration umsetzt. Die Abnahme erfolgt dann
		wiederum durch die Auftraggeber.
		Die zweite Nutzerstudie dient vor allem als Überprüfung für die Auftraggeber, ob die
		in der ersten Studie erkannten Probleme bis zur zweiten Studie behoben wurden.

\appendix
	\chapter{Anhang}
		(Am Ende des Projekts nachzureichen)\\
		Belege für durchgeführte Maßnahmen, bzw. falls nicht durchgeführt eine Begründung 
		wieso die Durchführung nicht möglich oder nicht erfolgt ist. \\
		Weitere Anforderungen sind den Unterlagen und der Vorlesung zur Projektbegleitung 
		zu entnehmen.

\end{document}
