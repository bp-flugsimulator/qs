\documentclass[accentcolor=tud9c,12pt,paper=a4]{tudreport}

\usepackage[utf8]{inputenc}
\usepackage{hyperref}
\usepackage{ngerman}
\usepackage{parcolumns}
\usepackage{pdfpages}
\usepackage{float}
\usepackage{caption}
\usepackage{todonotes}
\usepackage{listings}
\usepackage{tabulary}
\usepackage{booktabs}
\usepackage{longtable}

% Real paragraphs (with spaces between them instead of identation)
\parindent 0pt
\parskip 6pt
\newcommand{\titlerow}[2]{
	\begin{parcolumns}[colwidths={1=.17\linewidth}]{2}
		\colchunk[1]{#1:}
		\colchunk[2]{#2}
	\end{parcolumns}
	\vspace{0.2cm}
}


\lstdefinestyle{logs}{
    frame=single,
    postbreak=\mbox{\textcolor{red}{$\hookrightarrow$}\space},
    breaklines=true
}

\lstset{style=logs}

\graphicspath{ {img/} }

\title{Steuerungsprogramm für einen Flugsimulator}
\subtitle{Qualitätssicherungsdokument}
\subsubtitle{%
	\titlerow{Gruppe 19}{%
		Frederik Bark <frederikalexander.bark@stud.tu-darmstadt.de>\\
		Heiko Carrasco <heiko.carrascohuertas@stud.tu-darmstadt.de>\\
		Jonas Meurer <jonas.meurer@stud.tu-darmstadt.de>\\
		Tim Weißmantel <tim.weissmantel@stud.tu-darmstadt.de>\\
		Leonardo Zaninelli <leonardo.zaninelli@stud.tu-darmstadt.de>}
	\titlerow{Teamleiter}{Hendrik Bode <hendrik.bode@stud.tu-darmstadt.de>}
	\titlerow{Auftraggeber}{%
		Jonas Schulze <Schulze@fsr.tu-darmstadt.de>\\
		Torben Bernatzky <Bernatzky@fsr.tu-darmstadt.de>\\
		Technische Universität Darmstadt\\
		Flugsysteme und Regelungstechnik}
	\titlerow{Abgabedatum}{31.03.2018}}
\institution{Bachelor-Praktikum WS-2017/18\\ Fachbereich Informatik}

\begin{document}

	\maketitle
	\tableofcontents

	\chapter{Einleitung}
		Die Auftraggeber sind im Besitz eines Flugsimulators, der aus viele
		Teilprogrammen besteht. Diese Teilprogramme sind über mehrere Computer im
		gleichen Netzwerk verteilt und müssen alle in einer komplexen Reihenfolge per Hand
		gestartet werden. Aus diesem Grund wünschen sich die Auftraggeber eine Software, welche
		das Starten des Simulators automatisiert.\\[5pt]
		Es ist dabei zu beachten, dass auf den Computern unterschiedliche
		Betriebsysteme benutzt werden (verschiedene Windowsversionen und ggf. Linux).
		Der Nutzer soll neue Ablaufpläne selbst erstellen und verändern können,
		weswegen die Programmierung dieser Pläne mithilfe von generischen Komponenten
		wie "`Start eines Programmes"' oder "`Starten eines Computers"' möglich sein sollte.
		\\[5pt]
		Hierzu wird ein Master/Slave-System eingesetzt. Der Master stellt eine Weboberfläche zur
		Verfügung, in welcher der Nutzer Startpläne erstellen und ausführen lassen kann.
		Alle anderen Computer im Flugsimulator verbinden sich als Slaves mit dem Master und
		führen die spezifizierten Befehle aus.
		Für das Projekt wird sowohl die Master- als auch die Clientsoftware
		konzipiert und geschrieben. Zusätzlich wird die Oberfläche erstellt und ein
		Kommunikationsprotokoll definiert, welches die Kommunikation zwischen
		Master und Slaves beschreibt.
		\\[5pt]
		Die Software für Master und Slaves ist in Python geschrieben und benutzt
		das Django Framework für das Webinterface sowie RPC zur Kommunikation zwischen
		Master und Slaves.


	\chapter{Qualitätsziele}
		\section{Portabilität}
		Aufgrund der Heterogenität der verwendenten Betriebssysteme im
		Simulator, muss die Slave Software sowohl unter verschiedenen Windowsversionen
		(XP bis 10) als auch unter Linux laufen.
		Mit den Auftraggebern wurde abgesprochen, dass der Master keine
		Betriebsystemversion unter Windows 7 einsetzt, weswegen die Tests für Windows
		XP hier entfallen.
		\\[5pt]
		Um Portabilität sicherzustellen, werden alle Unit-Tests auf allen eingesetzten
		Betriebssystemen nach jedem Commit ausgeführt. Durch die 90\% Testabdeckung
		wird somit sichergestellt, dass der Kern der Software unter mehreren Betriebssystemen
		fehlerfrei läuft. Spätestens zur Abgabe einer
		Userstory müssen diese Tests fehlerfrei durchlaufen, sonst wird die Userstory nicht
		zur Abnahme präsentiert. Die Entscheidung zur Präsentation trifft der für diese
		Userstory bestimmte Reviewer. Der/die zuständige/n Entwickler behebt/beheben dann die Fehler
		in der nächsten Iteration.
		Als Testumgebung werden verwendet:
		\begin{itemize}
			\item Coveralls zum Bestimmen der Testabdeckung
			\item Travis CI zum Testen unter Linux
			\item AppVeyor zum Testen für Windowsversionen ab Windows 7
			\item Jenkins zum Testen auf Windows XP
		\end{itemize}
		Um die Installation auf allen Betriebssystemen zu vereinfachen, werden alle benötigten
		Bibliotheken mitgeliefert. Diese werden von den Testumgebungen verwendet, sodass
		auch die Installation auf den verschiedenen Betriebsystemen getestet wird.
		\newpage
		\section{Korrektheit}
		Der Flugsimulator ist ein komplexes System, welches aus mehreren verbundenen Komponenten besteht.
		Sollten zum Beispiel beim Start Fehler auftreten und werden diese nicht korrekt behandelt, so
		ist ein korrekter Start des Simulators nicht möglich. Daher ist es für das Projekt
		sehr wichtig, dass alle Komponenten korrekt spezifiziert und implementiert sind,
		sodass z.B. eventuell auftretende Fehler richtig behandelt werden können.
		\\[5pt]
		Der Prozess zur Sicherstellung dieses QS-Ziels sieht daher wie folgt aus:
		Nach der Implementierung wird die Userstory von mindestens einem weiteren
		Entwickler geprüft. Dazu greift dieser auf die Reports der Testingstools zurück,
		welche unter Portabilität genannt wurden, da es sich um die gleichen Tests handelt.
		Der Reviewer gestellt damit relativ einfach sicher, dass die Testabdeckung des Projekts
		nicht unter 90\% fällt, sonst wird die Userstory nicht zur Abnahme empfohlen und
		muss vom zuständigen Entwickler innerhalb der nächsten Iteration angepasst werden.
		\\[5pt]
		Zudem wird überprüft, ob die Dokumentation aller implementierten
		Codeteile mit dem zugehörigen Code übereinstimmen und ob die Implementierung der
		Userstory entspricht. Dabei wird die angehängte Checkliste als Grundlage verwendet.
		Beispielsweise muss der Code korrekt formatiert sein, keine der automatisierten Tools
		darf negative Rückmeldungen gegeben haben (z.B. Code nicht korrekt getestet, Codeformat
		nicht eingehalten, Tests laufen nicht durch) und der Code ist entsprechend dokumentiert.
		Die automatischen Tests prüfen die Einhaltung der PEP8-Spezifikation\footnote{\url{https://www.python.org/dev/peps/pep-0008/}}
		(Python), des csslint\footnote{\url{https://github.com/CSSLint/csslint}} CSS-Stils und
		des JavaScript Stils (definiert durch das eslint-Kommitee\footnote{\url{https://github.com/eslint/eslint}}).
		Sollten Punkte nicht eingehalten werden, wird die Userstory nicht den Auftraggebern zur
		Abnahme vorgestellt. Der/die verantwortliche/n Entwickler beheben dann die Anmerkungen.
		Erst wenn alle Punkte der Checkliste erfüllt sind, wird die Userstory den Auftraggebern
		zur Abnahme empfohlen.
		\newpage
		\section{Bedienbarkeit}\label{bedienbarkeit_qs}
		Damit auch weitere, eventuell nicht technisch versierte, Mitarbeiter die Software
		problemlos verwenden können, muss die
		Bedienoberfläche verständlich und leicht bedienbar sein. Eine Benutzung wird ohne
		vorherige, längere Einarbeitung möglich sein.
		\\[5pt]
		Durch die Verwendung des Webframeworks Bootstrap\footnote{\url{https://getbootstrap.com/}}
		wird das Design der Seite vereinheitlicht. Dies wird durch die Checkliste (im Anhang)
		vor der Präsentation einer Userstory für die Auftraggeber durch den Reviewer überprüft.
		Die Icons\footnote{\url{https://material.io/icons/}}
		stammen aus dem Material Design, welches zur Zeit auf über 79.8\% der Android Geräte
		verwendet wird\footnote{\url{https://developer.android.com/about/dashboards/index.html\#Platform}}.
		Damit ist es einer Vielzahl an Nutzern bereits bekannt.
		\\[5pt]
		Um den Einstieg zu erleichtern erstellt das BP-Team eine Bedienungsanleitung (siehe Anhang),
		welche die Hauptfunktionalitäten der Software beschreibt und erklärt. Mit der Erstellung
		der Anleitung wird Anfang Februar begonnen, da dann ein Großteil der Funktionalität
		implementiert ist.
		\\[5pt]
		Da die Bedienbarkeit nicht objektiv messbar ist, werden zusätzlich Nutzerstudien durchgeführt.
		Diese sind für Anfang Februar und Anfang März 2018 angesetzt.
		Die Testkandidaten kommen vorzugsweise aus dem
		Fachgebiet des Arbeitgebers, da diese am häufigsten mit der Automatisierungssoftware interagieren werden.
		Zufällig ausgewählte Kandidaten aus anderen Fachbereichen werden ebenfalls um Teilnahme gebeten, da
		so verglichen werden kann, wie benutzerfreundlich die Software auf Menschen wirkt, welche nicht täglich
		mit Flugsimulatoren arbeiten.
		\\[5pt]
		Für die Studie erhalten die Nutzer eine Auswahl an Aufgaben (siehe Anhang),
		welche sie zunächst ohne Anleitung durchführen sollen.
		Gibt es bei einer Aufgabe Probleme, so wird die Bedienungsanleitung
		der Software zur Verfügung gestellt.
		Am Ende wird ein Fragebogen ausgehändigt (ebenfalls im Anhang),
		in welchem die Nutzer das Design (z.B. Farbschema, Größe und Beschriftung von Buttons), den
		durch die Software vorgegeben Arbeitsablauf sowie die Anleitung bewerten können. Zudem
		besteht die Möglichkeit allgemeines Feedback abzugeben.
		\\[5pt]
		Mithilfe der ausgefüllten Fragebögen erstellt das Team einen Katalog an
		Maßnahmen bezüglich des Designs, Workflows oder der Anleitung,
		welche den Auftraggebern beim nächsten Treffen präsentiert werden.
		Diese entscheiden, welche Maßnahmen umgesetzt werden, woraufhin ein
		Entwickler diese in der nächsten Iteration umsetzt. Die Abnahme erfolgt dann
		wiederum durch die Auftraggeber.
		Die zweite Nutzerstudie dient vor allem als Überprüfung für die Auftraggeber, ob die
		in der ersten Studie erkannten Probleme bis zur zweiten Studie behoben wurden.

\appendix
	\chapter{Anhang}
	% Userstorys
	




\begin{table}[htbp]
\begin{minipage}{\linewidth}
\setlength{\tymax}{0.5\linewidth}
\centering
\small
\begin{tabulary}{\textwidth}{|l| p{10cm}|} \toprule
 ID   &11\\


Name  &Datei löschen\\
Beschreibung&Über das Kontextmenü kann der User Dateien auf den Slaves Löschen. Zusätzlich wird eine Warnung bei jedem Löschvorgang angezeigt\\
Akzeptanz &Der Pfad zur datei wird über ein Popup eingegeben. Nur die Angegebene Datei wird gelöscht, und die Warnung wird korrekt angezeigt\\
Story Points&4\\
Entwickler &?\\
Iteration &?\\
Stunden  &?\\
Velocity &?\\
Bemerkung &Obsolet\\
\bottomrule

\end{tabulary}
\end{minipage}
\end{table}



\begin{table}[htbp]
\begin{minipage}{\linewidth}
\setlength{\tymax}{0.5\linewidth}
\centering
\small
\begin{tabulary}{\textwidth}{|l| p{10cm}|} \toprule
 ID   &12\\


Name  &Interface am Start\\
Beschreibung&Als Benutzer will ich die Möglichkeit haben ein Webbrowser meiner Wahl automatisch zu öffnen sobald ich die Master Software starte. Der Webbrowser soll auch auf die Seite des Webinterfaces navigieren, damit ich diese nicht manuell tun muss.\\
Akzeptanz &Die Master Software wird gestartet und das Webbrowser Fenster öffnet sich zu der passenden Seite.\\
Story Points&2\\
Entwickler &?\\
Iteration &?\\
Stunden  &?\\
Velocity &?\\
\bottomrule

\end{tabulary}
\end{minipage}
\end{table}



\begin{table}[htbp]
\begin{minipage}{\linewidth}
\setlength{\tymax}{0.5\linewidth}
\centering
\small
\begin{tabulary}{\textwidth}{|l| p{10cm}|} \toprule
 ID   &14\\


Name  &Client löschen\\
Beschreibung&Als Benutzer will die Möglichkeit haben einen bereits registrierten Client wieder zu entfernen, damit ich nicht mehr vorhandende Komponenten löschen kann.\\
Akzeptanz &Nachdem der Benutzer den entsprechenden Knopf betätigt hat öffnet sich ein Dialog. In diesem Dialog muss der Benutzer den Löschvorgang bestätigen. Nach der Bestätigung ist der Client nicht mehr für den Benutzer im Webinterface zu sehen.\\
Story Points&4\\
Entwickler &Frederik, Leonardo, Tim\\
Iteration &1\\
Stunden  &9\\
Velocity &0.444 sp\slash std\\
\bottomrule

\end{tabulary}
\end{minipage}
\end{table}



\begin{table}[htbp]
\begin{minipage}{\linewidth}
\setlength{\tymax}{0.5\linewidth}
\centering
\small
\begin{tabulary}{\textwidth}{|l| p{10cm}|} \toprule
 ID   &15\\


Name  &Client bearbeiten\\
Beschreibung&Als Benutzer will ich die Möglichkeit haben eine bereits registrierten Client zu bearbeiten, damit ich Änderung einfach übernehmen kann.\\
Akzeptanz &Nachdem der Benutzer den entsprechenden Knopf betätigt hat, öffnet sich ein Dialog in dem sich momentanen Daten des Clients befinden. Der Benutzer kann nur korrekte Daten angeben. Bei nicht korrekten Daten wir der Benutzer auf die Fehler hingewiesen.\\
Story Points&4\\
Entwickler &Tim\\
Iteration &1\\
Stunden  &3\\
Velocity &1.333\\
\bottomrule

\end{tabulary}
\end{minipage}
\end{table}



\begin{table}[htbp]
\begin{minipage}{\linewidth}
\setlength{\tymax}{0.5\linewidth}
\centering
\small
\begin{tabulary}{\textwidth}{|l| p{10cm}|} \toprule
 ID   &16\\


Name  &Client Anzeigen\\
Beschreibung&Als Benutzer will ich die Möglichkeit haben einen Client mit seiner IP Adresse und seiner MAC Adresse, im Webinterface zu sehen. Da ich so eine besseren Überblick über das aktuell System bekomme.\\
Akzeptanz &Der Benutzer sieht die Ip Adresse und die MAC Adresse des entsprechenden Clients und dessen Status.\\
Story Points&4\\
Entwickler &Jonas, Tim\\
Iteration &1\\
Stunden  &4\\
Velocity &1\\
\bottomrule

\end{tabulary}
\end{minipage}
\end{table}



\begin{table}[htbp]
\begin{minipage}{\linewidth}
\setlength{\tymax}{0.5\linewidth}
\centering
\small
\begin{tabulary}{\textwidth}{|l| p{10cm}|} \toprule
ID   &17\\


Name  &Website Navigation\\
Beschreibung&Als Benutzer will ich die Möglichkeit haben die verschiedenen Seiten über eine Navigationsleiste zu erreichen. Damit ich mich einfach durch das Webinterface navigieren kann.\\
Akzeptanz &Der Benutzer sieht alle für ihn zugängliche Seiten in der Navigationsleiste und kann mit diesen auch zu den Seiten navigieren.\\
Story Points&1\\
Entwickler &Jonas\\
Iteration &3\\
Stunden  &1\\
Velocity &1\\
\bottomrule

\end{tabulary}
\end{minipage}
\end{table}



\begin{table}[htbp]
\begin{minipage}{\linewidth}
\setlength{\tymax}{0.5\linewidth}
\centering
\small
\begin{tabulary}{\textwidth}{|l| p{10cm}|} \toprule
 ID   &19\\


Name  &Programm anzeigen\\
Beschreibung&Als Benutzer will ich die Möglichkeit haben alle registrierten Programme, zugeordnet zu ihrem jeweiligen Client, einzusehen. Damit ich nachvollziehen kann welche Programme registriert sind.\\
Akzeptanz &Die Programme werden korrekt zu dem zugehörigen Client angezeigt. Auch sind alle Programme, die der Benutzer registriert hat, zu sehen.\\
Story Points&8\\
Entwickler &Leonardo,Tim\\
Iteration &2\\
Stunden  &6\\
Velocity &1.3\\
\bottomrule

\end{tabulary}
\end{minipage}
\end{table}
\begin{table}[htbp]
\begin{minipage}{\linewidth}
\setlength{\tymax}{0.5\linewidth}
\centering
\small
\begin{tabulary}{\textwidth}{|l| p{10cm}|} \toprule
 ID   & 1 \\


Name  & Willkommensnachricht\\
Beschreibung& Wenn sich der User mithilfe eines Browsers mit dem Master verbindet wird eine Willkommensnachricht angezeigt \\
Akzeptanz &Nach dem Senden einer Korrekten Anfrage an \texttt{\slash welcome} gibt der Server eine Webpage mit dem Inhalt ``welcome'' zurück\\
Story Points&10\\
Entwickler &Heiko\\
Iteration &1\\
Stunden  &2\\
Velocity &5 sp\slash std\\
Bemerkung &Der Master Server wird aufgesetzt\\
\bottomrule

\end{tabulary}
\end{minipage}
\end{table}



\begin{table}[htbp]
\begin{minipage}{\linewidth}
\setlength{\tymax}{0.5\linewidth}
\centering
\small
\begin{tabulary}{\textwidth}{|l| p{10cm}|} \toprule
 ID   &20\\


Name  &Programm hinzufügen\\
Beschreibung&Als Benutzer will ich die Möglichkeit haben ein Programm, welches auf einem Client liegt, mit bestimmten Parametern zu registrieren. Damit ich neue Programme hinzufügen kann.\\
Akzeptanz &Nachdem der Benutzer den entsprechenden Knopf betätigt hat öffnet sich ein Dialog welches die entsprechenden Felder beinhaltet, die die benötigten Information abfragen. Falls der Benutzer invalide Dateien angibt, wird dies dem Benutzer mitgeteilt.\\
Story Points&6\\
Entwickler &Tim\\
Iteration &2\\
Stunden  &5\\
Velocity &1.2 sp\slash std\\
\bottomrule

\end{tabulary}
\end{minipage}
\end{table}



\begin{table}[htbp]
\begin{minipage}{\linewidth}
\setlength{\tymax}{0.5\linewidth}
\centering
\small
\begin{tabulary}{\textwidth}{|l| p{10cm}|} \toprule
 ID   &21\\


Name  &Programm löschen\\
Beschreibung&Als Benutzer will ich die Möglichkeit haben, ein bereits registriete Programm wieder zu entfernen. Damit ich nicht mehr vorhandene Programme löschen kann.\\
Akzeptanz &Nachdem der Benutzer den entsprechenden Knopf betätigt hat öffnet sich Dialog, indem der Benutzer den Löschvorgang bestätigen muss.\\
Story Points&4\\
Entwickler &Frederik\\
Iteration &2\\
Stunden  &4,5\\
Velocity &0.89\\
\bottomrule

\end{tabulary}
\end{minipage}
\end{table}



\begin{table}[htbp]
\begin{minipage}{\linewidth}
\setlength{\tymax}{0.5\linewidth}
\centering
\small
\begin{tabulary}{\textwidth}{|l| p{10cm}|} \toprule
 ID   &22\\


Name  &Programm bearbeiten\\
Beschreibung&Als Benutzer will ich die Möglichkeit haben ein bereits registriertes Programm zu bearbeiten. Damit ich nicht mehr aktuelle Konfigurationen ändern kann.\\
Akzeptanz &Nachdem betätigen des entsprechenden Knopfes öffnet sich ein Dialog in dem die momentane Konfiguration angezeigt wird. Die Änderungen werden nur gespeichert sofern die Änderung immer noch eine korrekte Konfiguration darstellt. Bei fehlerhaften Daten wird der Benutzer drauf hingewiesen.\\
Story Points&5\\
Entwickler &Tim\\
Iteration &3\\
Stunden  &3\\
Velocity &1.7 StoryPoints\slash Stunde\\
\bottomrule

\end{tabulary}
\end{minipage}
\end{table}



\begin{table}[htbp]
\begin{minipage}{\linewidth}
\setlength{\tymax}{0.5\linewidth}
\centering
\small
\begin{tabulary}{\textwidth}{|l| p{10cm}|} \toprule
 ID   &23\\


Name  &Datei anzeigen\\
Beschreibung&Als Benutzer will ich die Möglichkeit haben eine registrierte Datei, die zu einem Client gehört, im Webinterface einzusehen. Damit ich einen Überblick über alle Dateien habe.\\
Akzeptanz &Der Benutzer sieht die Dateien korrekt bei dem dazugehörigen Client.\\
Story Points&3\\
Entwickler &Frederik\\
Iteration &3\\
Stunden  &3\\
Velocity &1 Sp\slash h\\
\bottomrule

\end{tabulary}
\end{minipage}
\end{table}



\begin{table}[htbp]
\begin{minipage}{\linewidth}
\setlength{\tymax}{0.5\linewidth}
\centering
\small
\begin{tabulary}{\textwidth}{|l| p{10cm}|} \toprule
 ID   &24\\


Name  &Datei hinzufügen\\
Beschreibung&Als Benutzer will ich die Möglichkeit haben eine Datei für einen Client zu registrieren (in der Datenbank). Damit ich neue Dateien registrieren kann.\\
Akzeptanz &Nachdem der Benutzer den entsprechenden Knopf betätigt hat öffnet sich ein Dialog. Es werden nur korrekte Eingaben gespeichert. Falls der Benutzer nicht korrekte Daten angibt wird dies ihm mitgeteilt.\\
Story Points&4\\
Entwickler &Frederik\\
Iteration &3\\
Stunden  &4\\
Velocity &1 Sp\slash h\\
\bottomrule

\end{tabulary}
\end{minipage}
\end{table}



\begin{table}[htbp]
\begin{minipage}{\linewidth}
\setlength{\tymax}{0.5\linewidth}
\centering
\small
\begin{tabulary}{\textwidth}{|l| p{10cm}|} \toprule
 ID   &21\\


Name  &Datei löschen\\
Beschreibung&Als Benutzer will ich die Möglichkeit haben eine bereits registrierte Datei wieder zu löschen (aus der Datenbank). Damit ich veraltet Dateien entfernen kann.\\
Akzeptanz &Nachdem Betätigten des entsprechenden Knopfes öffnet sich ein Dialog, indem der Benutzer den Löschvorgang bestätigen muss. Nur falls der Benutzer den Löschvorgang bestätigt wird die registrierte Datei aus der Datenbank gelöscht.\\
Story Points&3\\
Entwickler &Frederik\\
Iteration &4\\
Stunden  &1.5\\
Velocity &2sp\slash h\\
\bottomrule

\end{tabulary}
\end{minipage}
\end{table}



\begin{table}[htbp]
\begin{minipage}{\linewidth}
\setlength{\tymax}{0.5\linewidth}
\centering
\small
\begin{tabulary}{\textwidth}{|l| p{10cm}|} \toprule
ID   &26\\


Name  &Datei editieren\\
Beschreibung&Als Benutzer will ich die Möglichkeit haben eine bereits registrierte Datei (in der Datenbank) zu bearbeiten. Damit ich auch nur kleine Änderungen einfach einspielen kann.\\
Akzeptanz &Nachdem betätigen des Knopfes öffnet sich das Dialog mit den momentanen Informationen. Nur wenn der Benutzer korrekte Änderungen angibt werden diese gespeichert. Fehlerhafte Änderungen werden dem Benutzer angezeigt.\\
Story Points&5\\
Entwickler &Frederik\\
Iteration &7\\
Stunden  &2\\
Velocity &2.5 Sp\slash h\\
\bottomrule

\end{tabulary}
\end{minipage}
\end{table}



\begin{table}[htbp]
\begin{minipage}{\linewidth}
\setlength{\tymax}{0.5\linewidth}
\centering
\small
\begin{tabulary}{\textwidth}{|l| p{10cm}|} \toprule
 ID   &27\\


Name  &Skript anzeigen\\
Beschreibung&Als Benutzer will ich die Möglichkeit haben ein registriertes Skript im Webinterface einzusehen. Damit ich einen Überblick über alle vorhanden Skripte bekomme.\\
Akzeptanz &Das Skript wird korrekt angezeigt.\\
Story Points&15\\
Entwickler &Jonas\\
Iteration &3\\
Stunden  &20\\
Velocity &0,75\\
\bottomrule

\end{tabulary}
\end{minipage}
\end{table}



\begin{table}[htbp]
\begin{minipage}{\linewidth}
\setlength{\tymax}{0.5\linewidth}
\centering
\small
\begin{tabulary}{\textwidth}{|l| p{10cm}|} \toprule
ID   &28\\


Name  &Skript hinzufügen\\
Beschreibung&Als Benutzer will ich die Möglichkeit haben ein Skript hinzuzufügen, damit ich neue Szenarien auch mit dem System laufen lassen kann.\\
Akzeptanz &Nachdem der Benutzer den entsprechenden Knopf betätigt hat, wird der Benutzer auf eine weitere Seite geleitet, wo er ein Skript erstellen kann. Bei fehlerhaften Eingaben wird der Benutzer darauf hingewiesen. Es können keine fehlerhafte Skripte hinzugefügt werden.\\
Story Points&7\\
Entwickler &Jonas\\
Iteration &4\\
Stunden  &5\\
Velocity &1.4\\
\bottomrule

\end{tabulary}
\end{minipage}
\end{table}



\begin{table}[htbp]
\begin{minipage}{\linewidth}
\setlength{\tymax}{0.5\linewidth}
\centering
\small
\begin{tabulary}{\textwidth}{|l| p{10cm}|} \toprule
ID   &29\\


Name  &Skript löschen\\
Beschreibung&Als Benutzer will ich die Möglichkeit haben ein bereits registrierte Skript zu löschen. Damit ich veraltete Skripte entfernen kann.\\
Akzeptanz &Nachdem der Benutzer den entsprechenden Knopf betätigt wird ein Dialog geöffnet, indem der Benutzer den Löschvorgang bestätigen muss. Sofern der Benutzer den Löschvorgang bestätigt wird das Skript gelöscht, ansonsten geschiet nichts.\\
Story Points&5\\
Entwickler &Jonas\\
Iteration &4\\
Stunden  &5\\
Velocity &1\\
\bottomrule

\end{tabulary}
\end{minipage}
\end{table}



\begin{table}[htbp]
\begin{minipage}{\linewidth}
\setlength{\tymax}{0.5\linewidth}
\centering
\small
\begin{tabulary}{\textwidth}{|l| p{10cm}|} \toprule
 ID   &2\\


Name  &Client Registrieren\\
Beschreibung&Als Benutzer muss ich die Möglichkeit haben einen Client zu registieren, um diese später verwalten zu können.\\
Akzeptanz &Durch das betätigen des entsprechenden Knopfes öffnet sich ein Dialog. In diesem Dialog befinden sich Felder in die der Benutzer die Daten des Clients eintragen kann. Ungültige Eingaben werden abgewiesen und der Benutzer wird darauf hingewiesen.\\
Story Points&8\\
Entwickler &Tim, Jonas, Frederik\\
Iteration &1\\
Stunden  &30\\
Velocity &0,26\\
\bottomrule

\end{tabulary}
\end{minipage}
\end{table}



\begin{table}[htbp]
\begin{minipage}{\linewidth}
\setlength{\tymax}{0.5\linewidth}
\centering
\small
\begin{tabulary}{\textwidth}{|l| p{10cm}|} \toprule
ID   &30\\


Name  &Skript bearbeiten\\
Beschreibung&Als Benutzer will ich die Möglichkeit haben ein bereits registriertes Skript zu bearbeiten. Damit ich auch kleine Änderungen einfach anpassen kann.\\
Akzeptanz &Der Benutzer wird auf die Seite weitergeleitet, nachdem der Benutzer den entsprechenden Knopf betätigt hat. Auf dieser Seite kann der der Benutzer das Skript bearbeiten kann. Falls die Änderungen des Benutzers korrekt sind, wird das Skript gespeichert. Falls das nicht der Fall ist wird dem Benutzer das mitgeteilt.\\
Story Points&5\\
Entwickler &Leonardo\\
Iteration &7\\
Stunden  &7\\
Velocity &0.71428571428\\
\bottomrule

\end{tabulary}
\end{minipage}
\end{table}



\begin{table}[htbp]
\begin{minipage}{\linewidth}
\setlength{\tymax}{0.5\linewidth}
\centering
\small
\begin{tabulary}{\textwidth}{|l| p{10cm}|} \toprule
ID   &31\\


Name  &Start Vorgang\\
Beschreibung&Als Benutzer will ich die Möglichkeit haben ein Skript meiner wahl nach einer bestimmten Zeit automatisch starten zu lassen, sobald ich auf diese Seite gehe. Damit ich nicht gezwungen bin anwesend zu sein, sobald ich den Master Rechner starte.\\
Akzeptanz &Sobald der Benutzer die Seite aufruft beginnt ein Countdown. Sobald dieser abgelaufen ist wird das zuletzt ausgeführt Skript gestartet. Sobald der Benutzer ein anderes Skript auswählt wird der Countdown abgebrochen und der Benutzer muss den ``Start Vorgang'' manuell anstoßen.\\
Story Points&15\\
Entwickler &Jonas\\
Iteration &4--5\\
Stunden  &30\\
Velocity &0.5\\
\bottomrule

\end{tabulary}
\end{minipage}
\end{table}



\begin{table}[htbp]
\begin{minipage}{\linewidth}
\setlength{\tymax}{0.5\linewidth}
\centering
\small
\begin{tabulary}{\textwidth}{|l| p{10cm}|} \toprule
ID   &32\\


Name  &Start Vorgang Status\\
Beschreibung&Als Benutzer will ich die Möglichkeit haben den momentanen Status des ``Start Vorganges'' einzusehen. Des weiteren soll die Möglichkeit bestehen zu sehen, wie weit eine einzelne Stufe fortgeschritten ist.\\
Akzeptanz &Für jede Stufe wird ein Dropdown-Menü angezeigt. Das Menü der derzeit ausgeführten Stufe wird automatisch geöffnet. In diesem Menü wird jedes Program \slash  jede Datei angezeigt, das\slash die Teil dieser Stufe ist. Falls ein Fehler während des Vorgangs entsteht wird korrekt angezeigt welches Program \slash  welche Datei ihn erzeugt hat.\\
Story Points&10\\
Entwickler &Tim\\
Iteration &9\\
Stunden  &10\\
Velocity &1\\
\bottomrule

\end{tabulary}
\end{minipage}
\end{table}



\begin{table}[htbp]
\begin{minipage}{\linewidth}
\setlength{\tymax}{0.5\linewidth}
\centering
\small
\begin{tabulary}{\textwidth}{|l| p{10cm}|} \toprule
 ID   &33\\


Name  &beenden eines Programmes\\
Beschreibung&Als Benutzer will ich die Möglichkeit haben, ein bereits registriertes und gestartetes Programm beenden zu können, damit ich diese nicht manuell beenden muss.\\
Akzeptanz &Nachdem der Benutzer das Programm seiner Wahl beendet hat, in dem er den entsprechenden Knopf betätigt hat, ändert sich der Status des Programmes und der Benutzer bekommt ggf. eine Rückmeldung über den Endstatus des Programmes.\\
Story Points&8\\
Entwickler &Tim\\
Iteration &4\\
Stunden  &9\\
Velocity &0.888 Storypoints\slash Stunde\\
\bottomrule

\end{tabulary}
\end{minipage}
\end{table}



\begin{table}[htbp]
\begin{minipage}{\linewidth}
\setlength{\tymax}{0.5\linewidth}
\centering
\small
\begin{tabulary}{\textwidth}{|l| p{10cm}|} \toprule
 ID   &34\\


Name  &Bootuptime eines Programmes\\
Beschreibung&Als Benutzer will ich die Möglichkeit haben, für eine Programm einzutragen wie lange dieses zum Starten braucht, um das gleichzeitige starten zweier Programme die voneinander abhängen zu verhindern.\\
Akzeptanz &Nachdem der Benutzer das Programm seiner Wahl beendet hat, in dem er den entsprechenden Knopf betätigt hat, ändert sich der Status des Programmes und der Benutzer bekommt ggf. eine Rückmeldung über den Endstatus des Programmes.\\
Story Points&1\\
Akzeptanz &Nachdem der Benutzer den entsprechenden Knopf zum hinzufügen\slash editieren eines Programmes betätigt hat, gibt es in dem jeweiligen Dialog ein weiteres Feld, in dem die Zeit in Sekunden eingetragen werden kann. Nach dem dies bestätigen wurde, wird der Eingrag korrekt in der Datenbank gespeichert.\\
Entwickler &Jonas\\
Iteration &4\\
Stunden  &1\\
Velocity &1\\
\bottomrule

\end{tabulary}
\end{minipage}
\end{table}



\begin{table}[htbp]
\begin{minipage}{\linewidth}
\setlength{\tymax}{0.5\linewidth}
\centering
\small
\begin{tabulary}{\textwidth}{|l| p{10cm}|} \toprule
 ID   &35\\


Name  &Überprüfen von Argumenten in Programmen\\
Beschreibung&Als Benutzer will ich die Möglichkeit haben, beim Eintragen von Argumenten eines Programmes daraufhingewiesen zu werden, wenn eine nicht parsbare Argumentenliste eingegeben wird.\\
Akzeptanz &Nachdem der Benutzer eine nicht parsbare Argumentenliste in das entsprechenden Feld während dem Hinzufügen\slash Editieren eine Programmeintrages eingetragen und bestätigt hat, erscheint eine entprechende Fehlermeldung.\\
Story Points&1\\
Entwickler &Tim\\
Iteration &4\\
Stunden  &1\\
Velocity &1 Storypoints\slash Stunde\\
\bottomrule

\end{tabulary}
\end{minipage}
\end{table}



\begin{table}[htbp]
\begin{minipage}{\linewidth}
\setlength{\tymax}{0.5\linewidth}
\centering
\small
\begin{tabulary}{\textwidth}{|l| p{10cm}|} \toprule
 ID   &36\\


Name  &Dateien vom Server herunterladen\\
Beschreibung&Als Benutzer will ich die Möglichkeit haben, Dateien, die auf dem Server in einem definierten Ordner liegen herunterzuladen.\\
Akzeptanz &Nachdem der Benutzer in der Navigationsleiste auf den ensprechenden Knopf gedrückt hat, werden alle Dateien die in dem definierten Ordner liegen mit ihrer Größe zusammen angezeigt. Diese wird dann nach dem Drücken des entsprechenden Knopfes heruntergeladen.\\
Story Points&3\\
Entwickler &Tim\\
Iteration &4\\
Stunden  &3\\
Velocity &1 Storypoints\slash Stunde\\
\bottomrule

\end{tabulary}
\end{minipage}
\end{table}



\begin{table}[htbp]
\begin{minipage}{\linewidth}
\setlength{\tymax}{0.5\linewidth}
\centering
\small
\begin{tabulary}{\textwidth}{|l| p{10cm}|} \toprule
 ID   &37\\


Name  &Logs von Programmen\\
Beschreibung&Als Benutzer will ich die Möglichkeit haben, den Log eines Programmes im Webinterface einzusehen.\\
Akzeptanz &Nachdem der Benutzer den ensprechenden Knopf neben einem Programmeintrag gedrückt hat, wird unter dem Programmeintrag die zuletzt gespeicherte Logdatei in einem Textfeld angezeigt.\\
Story Points&30\\
Entwickler &Tim\\
Iteration &5--7\\
Stunden  &40\\
Velocity &0.75\\
\bottomrule

\end{tabulary}
\end{minipage}
\end{table}



\begin{table}[htbp]
\begin{minipage}{\linewidth}
\setlength{\tymax}{0.5\linewidth}
\centering
\small
\begin{tabulary}{\textwidth}{|l| p{10cm}|} \toprule
ID   &38\\


Name  &Skript für Dateien\\
Beschreibung&Als Benutzer will ich die Möglichkeit haben auch Dateien in ein Skript mit zu bewegen.\\
Akzeptanz &Nachdem der Benutzer das Skript gestartet hat, wird werden auch die gewünschten Dateien bewegt. Falls eine kritischer Fehler bei den Dateien auftritt wird der Prozess gestoppt.\\
Story Points&5\\
Entwickler &Jonas\\
Iteration &7\\
Stunden  &12\\
Velocity &0.41\\
\bottomrule

\end{tabulary}
\end{minipage}
\end{table}



\begin{table}[htbp]
\begin{minipage}{\linewidth}
\setlength{\tymax}{0.5\linewidth}
\centering
\small
\begin{tabulary}{\textwidth}{|l| p{10cm}|} \toprule
ID   &39\\


Name  &Skript bearbeiten\\
Beschreibung&Als Benutzer will ich die Möglichkeit haben ein bereits erstelltes Skript zu kopieren dann zu bearbeiten und dann zu speichern.\\
Akzeptanz &Nachdem der Benutzer den entsprechenden Knopf für das Kopieren gedrückt hat, wird die Kopie in einem seperatem Menüeintrag angezeigt. Der Name der Kopie ist der originale Name mit dem Suffix ``\_copy''. Namenskonflikte werden mit aufsteigenden Zahlen als Zusatz zum Suffix gehandhabt.\\
Story Points&3\\
Entwickler &Tim\\
Iteration &7\\
Stunden  &1.5\\
Velocity &2\\
\bottomrule

\end{tabulary}
\end{minipage}
\end{table}



\begin{table}[htbp]
\begin{minipage}{\linewidth}
\setlength{\tymax}{0.5\linewidth}
\centering
\small
\begin{tabulary}{\textwidth}{|l| p{10cm}|} \toprule
 ID   &3\\


Name  &Hochfahren eines Clients\\
Beschreibung&Als Benutzer will ich die Möglichkeit haben einen Client mit der Hilfe des Webinterfaces hoch zu fahren, um später Programme auf diesem Client zu starten.\\
Akzeptanz &Nachdem der Benutzer den entsprechenden Knopf betätigt hat, wird der Client dazu aufgefordert zu starten und fährt ggf. dann hoch.\\
Story Points&3\\
Entwickler &Heiko\\
Iteration &2\\
Stunden  &7\\
Velocity &0.42\\
\bottomrule

\end{tabulary}
\end{minipage}
\end{table}



\begin{table}[htbp]
\begin{minipage}{\linewidth}
\setlength{\tymax}{0.5\linewidth}
\centering
\small
\begin{tabulary}{\textwidth}{|l| p{10cm}|} \toprule
ID   &40\\


Name  &Ungespeicherte Inhalte Warnung\\
Beschreibung&Als Benutzer will ich die Möglichkeit haben gewarnt zu werden sobald ich eine Eingabe verwerfen werde.\\
Akzeptanz &Der Benutzer wird mit eine Pop-Up Dialog gewarnt ob er fortfahren will sobald er eine ungespeicherte Eingabe verwerfen will.\\
Story Points&5\\
Entwickler &Heiko\\
Iteration &7\\
Stunden  &4\\
Velocity &1.25\\
\bottomrule

\end{tabulary}
\end{minipage}
\end{table}



\begin{table}[htbp]
\begin{minipage}{\linewidth}
\setlength{\tymax}{0.5\linewidth}
\centering
\small
\begin{tabulary}{\textwidth}{|l| p{10cm}|} \toprule
ID   &41\\


Name  &Skript stoppen\\
Beschreibung&Als Benutzer will ich die Möglichkeit haben ein Skript während der Ausführung zu stoppen, falls es nicht gestarttet werden sollte\slash es zu Fehlern kam.\\
Akzeptanz &Nachdem der Benutzer den entsprechenden Knopf betätigt hat, wird das Ausführen des Skripts gestoppt. Laufende Programme werden nicht beendet und der Ursprungszustand des Filesystems wird nicht wieder hergestellt.\\
Story Points&3\\
Entwickler &Frederik\\
Iteration &8\\
Stunden  &2\\
Velocity &1.5 Sp\slash h\\
\bottomrule

\end{tabulary}
\end{minipage}
\end{table}



\begin{table}[htbp]
\begin{minipage}{\linewidth}
\setlength{\tymax}{0.5\linewidth}
\centering
\small
\begin{tabulary}{\textwidth}{|l| p{10cm}|} \toprule
 ID   &42\\


Name  &Automatisches starten eines Skriptes\\
Beschreibung&Als Benutzer will ich die Möglichkeit haben beim Aufruf einer bestimmten Seite automatisch ein Skript starten zu lassen.\\
Akzeptanz &Sobald man die Seite aufruft wird ein sichtbarer Kurzeitwecker angezeigt. Sobald dieser abgelaufen ist wird das zuletzt erfolgreich getartet Skript gestartet.\\
Story Points&4\\
Entwickler &Jonas\\
Iteration &9\\
Stunden  &3\\
Velocity &1.3\\
\bottomrule

\end{tabulary}
\end{minipage}
\end{table}



\begin{table}[htbp]
\begin{minipage}{\linewidth}
\setlength{\tymax}{0.5\linewidth}
\centering
\small
\begin{tabulary}{\textwidth}{|l| p{10cm}|} \toprule
ID   &43\\


Name  &Alle Programme stoppen\\
Beschreibung&Als Benutzer will ich die Möglichkeit haben alle laufende Programme zu stoppen.\\
Akzeptanz &Nachdem der Benutzer den entsprechenden Knopf betätigt hat, werden alle derzeit laufenden Programme beendet. Verschobene Dateien werden dabei nicht verändert. Falls ein Skript ausgeführt wird, wird es beendet.\\
Story Points&4\\
Entwickler &Frederik\\
Iteration &9\\
Stunden  &3\\
Velocity &1.33\\
\bottomrule

\end{tabulary}
\end{minipage}
\end{table}



\begin{table}[htbp]
\begin{minipage}{\linewidth}
\setlength{\tymax}{0.5\linewidth}
\centering
\small
\begin{tabulary}{\textwidth}{|l| p{10cm}|} \toprule
ID   &44\\


Name  &Alle Dateien zurücksetzen\\
Beschreibung&Als Benutzer will ich die Möglichkeit haben alle vom Interface aus verschobenen Dateien zu entfernen und den Ursprungszustand des Systems wieder herzustellen.\\
Akzeptanz &Nachdem der Benutzer den entsprechenden Knopf betätigt hat, werden alle derzeit verschobenen Dateien entfernt. Wurde eine Datei beim ursprünglichen Kopieren ersetzt, soll sie wieder hergestellt werden. Ein laufendes Skript wird ebenfalls beendet.\\
Story Points&4\\
Entwickler &Frederik\\
Iteration &9\\
Stunden  &3.5\\
Velocity &1.14 sp\slash h\\
\bottomrule

\end{tabulary}
\end{minipage}
\end{table}



\begin{table}[htbp]
\begin{minipage}{\linewidth}
\setlength{\tymax}{0.5\linewidth}
\centering
\small
\begin{tabulary}{\textwidth}{|l| p{10cm}|} \toprule
ID   &45\\


Name  &Hilfstexte für Formularfelder\\
Beschreibung&Als Benutzer will ich die Möglichkeit haben mir zu allen Formularfeldern einen kleinen Hilfetext anzeigen zu lassen.\\
Akzeptanz &Nachdem der Nutzer auf das Hilfsicon gelickt hat öffnet sich ein Popup mit einem Hinweistext\\
Story Points&4\\
Entwickler &Leonardo\\
Iteration &8\\
Stunden  &4\\
Velocity &1\\
\bottomrule

\end{tabulary}
\end{minipage}
\end{table}



\begin{table}[htbp]
\begin{minipage}{\linewidth}
\setlength{\tymax}{0.5\linewidth}
\centering
\small
\begin{tabulary}{\textwidth}{|l| p{10cm}|} \toprule
ID   &46\\


Name  &Dropdown für Shutdown Funktionen\\
Beschreibung&Nach dem Klick auf ein Schutdownicon wird eine Liste aller möglichen Shutdown-Aktionen angezeigt.\\
Akzeptanz &Nachdem der Nutzer auf das Schutdownicon gelickt hat öffnet sich ein Dropdown mit Auswahlmöglichkeiten\\
Story Points&1\\
Entwickler &Leonardo\\
Iteration &9\\
Stunden  &1\\
Velocity &1\\
\bottomrule

\end{tabulary}
\end{minipage}
\end{table}



\begin{table}[htbp]
\begin{minipage}{\linewidth}
\setlength{\tymax}{0.5\linewidth}
\centering
\small
\begin{tabulary}{\textwidth}{|l| p{10cm}|} \toprule
ID   &47\\


Name  &Alles Ausschalten\\
Beschreibung&Als Benutzer will ich die Möglichkeit haben alle laufenden Clients und den Master über das Webinterface auszuschalten.\\
Akzeptanz &Nachdem der Benutzer den entsprechenden Knopf betätigt hat, werden alle Clients und der Master heruntergefahren. Falls ein Skript läuft wird dieses beendet, verschobene Dateien werden zurüchgesetzt und laufende Programme werden beendet.\\
Story Points&5\\
Entwickler &Frederik\\
Iteration &9\\
Stunden  &13\\
Velocity &0.71 sp\slash h\\
\bottomrule

\end{tabulary}
\end{minipage}
\end{table}



\begin{table}[htbp]
\begin{minipage}{\linewidth}
\setlength{\tymax}{0.5\linewidth}
\centering
\small
\begin{tabulary}{\textwidth}{|l| p{10cm}|} \toprule
ID   &48\\


Name  &Default Skript setzten\\
Beschreibung&Als Benutzer will ich die Möglichkeit haben ein Skript manuell zu setzten welches automatisch gestartet werden kann.\\
Akzeptanz &Nachdem der Benutzer den entsprechen Knopf gedrückt hat, wird beim nächsten Neustart das ausgewählte Skript gestartet.\\
Story Points&5\\
Entwickler &Jonas\\
Iteration &9\\
Stunden  &3\\
Velocity &1,6\\
\bottomrule

\end{tabulary}
\end{minipage}
\end{table}



\begin{table}[htbp]
\begin{minipage}{\linewidth}
\setlength{\tymax}{0.5\linewidth}
\centering
\small
\begin{tabulary}{\textwidth}{|l| p{10cm}|} \toprule
 ID   &49\\


Name  &automatische Neuverbindung eines Clients\\
Beschreibung&Der Nutzer will nicht einen Client neustarten, falls dieser für eine kurze Zeit die Verbindung verloren hat. Daher soll der Client selbst versuchen die Verbing neu aufzubauen.\\
Akzeptanz &Nachdem der Client die Verbindung verloren hat, versucht dieser die Verbindung neu aufzubauen. Falls der Verbindungsaufbau fehlschlägt wird der Vorgang wiederhohlt. Danach ist der Client wieder uneingeschränkt nutzbar.\\
Story Points&3\\
Entwickler &Tim\\
Iteration &9\\
Stunden  &2\\
Velocity &1.5\\
\multicolumn{2}{c}{Bemerkung }\\
\bottomrule

\end{tabulary}
\end{minipage}
\end{table}



\begin{table}[htbp]
\begin{minipage}{\linewidth}
\setlength{\tymax}{0.5\linewidth}
\centering
\small
\begin{tabulary}{\textwidth}{|l| p{10cm}|} \toprule
 ID   &4\\


Name  &Status eines Clients\\
Beschreibung&Als Benutzer will ich den Status jedes bereits registrierten Clients einsehen können, um mit diesen entsprechend Interagieren zu können.\\
Akzeptanz &Der Status wird korrekt angezeigt und wird erneuert sobald sich der Status des Clients geändert hat.\\
Story Points&4\\
Entwickler &Tim\\
Iteration &3 und 4\\
Stunden  &6\\
Velocity &0.666 Storypoints\slash Stunde\\
\bottomrule

\end{tabulary}
\end{minipage}
\end{table}



\begin{table}[htbp]
\begin{minipage}{\linewidth}
\setlength{\tymax}{0.5\linewidth}
\centering
\small
\begin{tabulary}{\textwidth}{|l| p{10cm}|} \toprule
ID   &5\\


Name  &Client herunterfahren\\
Beschreibung&Als Benutzer will einen Client per Webinterface herunterfahren, um mir Arbeit zu sparen.\\
Akzeptanz &Nachdem der Benutzer den entsprechenden Knopf betätigt hat wird der Client ordnungsgemäß heruntergefahren. D.h. alle Programm die durch den Benutzer gestartet worden sind werden gestoppt und dann wird das System dazu angehalten herunter zu fahren.\\
Story Points&5\\
Entwickler &Tim\\
Iteration &3\\
Stunden  &4\\
Velocity &1.25 Storypoints\slash Stunde\\
\bottomrule

\end{tabulary}
\end{minipage}
\end{table}



\begin{table}[htbp]
\begin{minipage}{\linewidth}
\setlength{\tymax}{0.5\linewidth}
\centering
\small
\begin{tabulary}{\textwidth}{|l| p{10cm}|} \toprule
 ID   &6\\


Name  &Kontextmenü\\
Beschreibung&Als Benutzer will ich die Möglichkeit haben alle Befehle für einen Client einzusehen, damit ich eine passende Aktion wählen kann.\\
Akzeptanz &Alle Befehle sind visuell mit einem Client assoziiert.\\
Story Points&10\\
Entwickler &Leonardo, Jonas, Tim\\
Iteration &1\\
Stunden  &10\\
Velocity &1\\
Bemerkung &Obsolet\\
\bottomrule

\end{tabulary}
\end{minipage}
\end{table}



\begin{table}[htbp]
\begin{minipage}{\linewidth}
\setlength{\tymax}{0.5\linewidth}
\centering
\small
\begin{tabulary}{\textwidth}{|l| p{10cm}|} \toprule
 ID   &7\\


Name  &Starten eines Programmes\\
Beschreibung&Als Benutzer will ich die Möglichkeit haben, ein bereits registriertes Programm starten zu können, damit ich diese nicht manuell starten muss.\\
Akzeptanz &Nachdem der Benutzer das Programm seiner Wahl gestartet hat, in dem er den entsprechenden Knopf betätigt hat, ändert sich der Status des Programmes und der Benutzer bekommt ggf. eine Rückmeldung über den Endstatus des Programmes.\\
Story Points&25\\
Entwickler &Jonas, Tim\\
Iteration &3\\
Stunden  &50\\
Velocity &0.5\\
\bottomrule

\end{tabulary}
\end{minipage}
\end{table}



\begin{table}[htbp]
\begin{minipage}{\linewidth}
\setlength{\tymax}{0.5\linewidth}
\centering
\small
\begin{tabulary}{\textwidth}{|l| p{10cm}|} \toprule
 ID   &8\\


Name  &Statusabfrage von Programmen auf einem Client\\
Beschreibung&Als Benutzer will ich die Möglichkeit haben den Status eines registrierten Programmes einzusehen, damit ich mir einen Überblick verschaffen kann.\\
Akzeptanz &Der Status des Prozesses ist aktuell, korrekt und wird bei dem entsprechenden Programm, auf dem Webinterface angezeigt wird.\\
Story Points&4\\
Entwickler &Tim\\
Iteration &3 und 4\\
Stunden  &6\\
Velocity &0.666 Storypoints\slash Stunde\\
\bottomrule

\end{tabulary}
\end{minipage}
\end{table}



\begin{table}[htbp]
\begin{minipage}{\linewidth}
\setlength{\tymax}{0.5\linewidth}
\centering
\small
\begin{tabulary}{\textwidth}{|l| p{10cm}|} \toprule
 ID   &9\\


Name  &Datei bewegen\\
Beschreibung&Als Benutzer will ich die Möglichkeit haben eine registrierte Datei auf einem registrierten Client zu bewegen, damit ich einfach Konfigurationen austauschen kann. Auch soll es Möglich sein den Prozess wieder rückgänig zu machen.\\
Akzeptanz &Nachdem der Benutzer den entsprechenden Kopf betätigt hat wird die Datei auf dem Client verschoben und der Benutzer wird darüber benachrichtigt. Sofern eine Datei bereits bewegt wurde under Benutzer erneut auf den entsprechenden Knopf drückt, wird die Datei wieder entfernt (sofern diese nicht in der zwischen Zeit vom Benutzer manuell überschrieben worden ist). \\
Story Points&10\\
Entwickler &Leonardo, Jonas\\
Iteration &7\\
Stunden  &30\\
Velocity &0.33\\
\bottomrule

\end{tabulary}
\end{minipage}
\end{table}
\begin{table}[htbp]
\begin{minipage}{\linewidth}
\setlength{\tymax}{0.5\linewidth}
\centering
\small
\begin{tabulary}{\textwidth}{|l| p{10cm}|} \toprule
\multicolumn{2}{c}{ ID   }\\


\multicolumn{2}{c}{Name  }\\
\multicolumn{2}{c}{Beschreibung}\\
\multicolumn{2}{c}{Akzeptanz }\\
Story Points&?\\
Entwickler &?\\
Iteration &?\\
Stunden  &?\\
Velocity &?\\
\multicolumn{2}{c}{Bemerkung }\\
\bottomrule

\end{tabulary}
\end{minipage}
\end{table}



\begin{table}[htbp]
\begin{minipage}{\linewidth}
\setlength{\tymax}{0.5\linewidth}
\centering
\small
\begin{tabulary}{\textwidth}{|l| p{10cm}|} \toprule
 ID   &10\\


Name  &Dateien herunterladen\\
Beschreibung&Über das Kontextmenü kann eine Datei vom Master auf den Slave geladen werden.\\
Akzeptanz &Die Datei wird vollständig ohne Fehler auf den richtigen Slave geladen\\
Story Points&?\\
Entwickler &?\\
Iteration &?\\
Stunden  &?\\
Velocity &?\\
Bemerkung &Obsolet\\
\bottomrule

\end{tabulary}
\end{minipage}
\end{table}



\begin{table}[htbp]
\begin{minipage}{\linewidth}
\setlength{\tymax}{0.5\linewidth}
\centering
\small
\begin{tabulary}{\textwidth}{|l| p{10cm}|} \toprule
 ID   &11\\


Name  &Datei löschen\\
Beschreibung&Über das Kontextmenü kann der User Dateien auf den Slaves Löschen. Zusätzlich wird eine Warnung bei jedem Löschvorgang angezeigt\\
Akzeptanz &Der Pfad zur datei wird über ein Popup eingegeben. Nur die Angegebene Datei wird gelöscht, und die Warnung wird korrekt angezeigt\\
Story Points&4\\
Entwickler &?\\
Iteration &?\\
Stunden  &?\\
Velocity &?\\
Bemerkung &Obsolet\\
\bottomrule

\end{tabulary}
\end{minipage}
\end{table}



\begin{table}[htbp]
\begin{minipage}{\linewidth}
\setlength{\tymax}{0.5\linewidth}
\centering
\small
\begin{tabulary}{\textwidth}{|l| p{10cm}|} \toprule
 ID   &12\\


Name  &Interface am Start\\
Beschreibung&Als Benutzer will ich die Möglichkeit haben ein Webbrowser meiner Wahl automatisch zu öffnen sobald ich die Master Software starte. Der Webbrowser soll auch auf die Seite des Webinterfaces navigieren, damit ich diese nicht manuell tun muss.\\
Akzeptanz &Die Master Software wird gestartet und das Webbrowser Fenster öffnet sich zu der passenden Seite.\\
Story Points&2\\
Entwickler &?\\
Iteration &?\\
Stunden  &?\\
Velocity &?\\
\bottomrule

\end{tabulary}
\end{minipage}
\end{table}



\begin{table}[htbp]
\begin{minipage}{\linewidth}
\setlength{\tymax}{0.5\linewidth}
\centering
\small
\begin{tabulary}{\textwidth}{|l| p{10cm}|} \toprule
 ID   &14\\


Name  &Client löschen\\
Beschreibung&Als Benutzer will die Möglichkeit haben einen bereits registrierten Client wieder zu entfernen, damit ich nicht mehr vorhandende Komponenten löschen kann.\\
Akzeptanz &Nachdem der Benutzer den entsprechenden Knopf betätigt hat öffnet sich ein Dialog. In diesem Dialog muss der Benutzer den Löschvorgang bestätigen. Nach der Bestätigung ist der Client nicht mehr für den Benutzer im Webinterface zu sehen.\\
Story Points&4\\
Entwickler &Frederik, Leonardo, Tim\\
Iteration &1\\
Stunden  &9\\
Velocity &0.444 sp\slash std\\
\bottomrule

\end{tabulary}
\end{minipage}
\end{table}



\begin{table}[htbp]
\begin{minipage}{\linewidth}
\setlength{\tymax}{0.5\linewidth}
\centering
\small
\begin{tabulary}{\textwidth}{|l| p{10cm}|} \toprule
 ID   &15\\


Name  &Client bearbeiten\\
Beschreibung&Als Benutzer will ich die Möglichkeit haben eine bereits registrierten Client zu bearbeiten, damit ich Änderung einfach übernehmen kann.\\
Akzeptanz &Nachdem der Benutzer den entsprechenden Knopf betätigt hat, öffnet sich ein Dialog in dem sich momentanen Daten des Clients befinden. Der Benutzer kann nur korrekte Daten angeben. Bei nicht korrekten Daten wir der Benutzer auf die Fehler hingewiesen.\\
Story Points&4\\
Entwickler &Tim\\
Iteration &1\\
Stunden  &3\\
Velocity &1.333\\
\bottomrule

\end{tabulary}
\end{minipage}
\end{table}



\begin{table}[htbp]
\begin{minipage}{\linewidth}
\setlength{\tymax}{0.5\linewidth}
\centering
\small
\begin{tabulary}{\textwidth}{|l| p{10cm}|} \toprule
 ID   &16\\


Name  &Client Anzeigen\\
Beschreibung&Als Benutzer will ich die Möglichkeit haben einen Client mit seiner IP Adresse und seiner MAC Adresse, im Webinterface zu sehen. Da ich so eine besseren Überblick über das aktuell System bekomme.\\
Akzeptanz &Der Benutzer sieht die Ip Adresse und die MAC Adresse des entsprechenden Clients und dessen Status.\\
Story Points&4\\
Entwickler &Jonas, Tim\\
Iteration &1\\
Stunden  &4\\
Velocity &1\\
\bottomrule

\end{tabulary}
\end{minipage}
\end{table}



\begin{table}[htbp]
\begin{minipage}{\linewidth}
\setlength{\tymax}{0.5\linewidth}
\centering
\small
\begin{tabulary}{\textwidth}{|l| p{10cm}|} \toprule
ID   &17\\


Name  &Website Navigation\\
Beschreibung&Als Benutzer will ich die Möglichkeit haben die verschiedenen Seiten über eine Navigationsleiste zu erreichen. Damit ich mich einfach durch das Webinterface navigieren kann.\\
Akzeptanz &Der Benutzer sieht alle für ihn zugängliche Seiten in der Navigationsleiste und kann mit diesen auch zu den Seiten navigieren.\\
Story Points&1\\
Entwickler &Jonas\\
Iteration &3\\
Stunden  &1\\
Velocity &1\\
\bottomrule

\end{tabulary}
\end{minipage}
\end{table}



\begin{table}[htbp]
\begin{minipage}{\linewidth}
\setlength{\tymax}{0.5\linewidth}
\centering
\small
\begin{tabulary}{\textwidth}{|l| p{10cm}|} \toprule
 ID   &19\\


Name  &Programm anzeigen\\
Beschreibung&Als Benutzer will ich die Möglichkeit haben alle registrierten Programme, zugeordnet zu ihrem jeweiligen Client, einzusehen. Damit ich nachvollziehen kann welche Programme registriert sind.\\
Akzeptanz &Die Programme werden korrekt zu dem zugehörigen Client angezeigt. Auch sind alle Programme, die der Benutzer registriert hat, zu sehen.\\
Story Points&8\\
Entwickler &Leonardo,Tim\\
Iteration &2\\
Stunden  &6\\
Velocity &1.3\\
\bottomrule

\end{tabulary}
\end{minipage}
\end{table}
\begin{table}[htbp]
\begin{minipage}{\linewidth}
\setlength{\tymax}{0.5\linewidth}
\centering
\small
\begin{tabulary}{\textwidth}{|l| p{10cm}|} \toprule
 ID   & 1 \\


Name  & Willkommensnachricht\\
Beschreibung& Wenn sich der User mithilfe eines Browsers mit dem Master verbindet wird eine Willkommensnachricht angezeigt \\
Akzeptanz &Nach dem Senden einer Korrekten Anfrage an \texttt{\slash welcome} gibt der Server eine Webpage mit dem Inhalt ``welcome'' zurück\\
Story Points&10\\
Entwickler &Heiko\\
Iteration &1\\
Stunden  &2\\
Velocity &5 sp\slash std\\
Bemerkung &Der Master Server wird aufgesetzt\\
\bottomrule

\end{tabulary}
\end{minipage}
\end{table}



\begin{table}[htbp]
\begin{minipage}{\linewidth}
\setlength{\tymax}{0.5\linewidth}
\centering
\small
\begin{tabulary}{\textwidth}{|l| p{10cm}|} \toprule
 ID   &20\\


Name  &Programm hinzufügen\\
Beschreibung&Als Benutzer will ich die Möglichkeit haben ein Programm, welches auf einem Client liegt, mit bestimmten Parametern zu registrieren. Damit ich neue Programme hinzufügen kann.\\
Akzeptanz &Nachdem der Benutzer den entsprechenden Knopf betätigt hat öffnet sich ein Dialog welches die entsprechenden Felder beinhaltet, die die benötigten Information abfragen. Falls der Benutzer invalide Dateien angibt, wird dies dem Benutzer mitgeteilt.\\
Story Points&6\\
Entwickler &Tim\\
Iteration &2\\
Stunden  &5\\
Velocity &1.2 sp\slash std\\
\bottomrule

\end{tabulary}
\end{minipage}
\end{table}



\begin{table}[htbp]
\begin{minipage}{\linewidth}
\setlength{\tymax}{0.5\linewidth}
\centering
\small
\begin{tabulary}{\textwidth}{|l| p{10cm}|} \toprule
 ID   &21\\


Name  &Programm löschen\\
Beschreibung&Als Benutzer will ich die Möglichkeit haben, ein bereits registriete Programm wieder zu entfernen. Damit ich nicht mehr vorhandene Programme löschen kann.\\
Akzeptanz &Nachdem der Benutzer den entsprechenden Knopf betätigt hat öffnet sich Dialog, indem der Benutzer den Löschvorgang bestätigen muss.\\
Story Points&4\\
Entwickler &Frederik\\
Iteration &2\\
Stunden  &4,5\\
Velocity &0.89\\
\bottomrule

\end{tabulary}
\end{minipage}
\end{table}



\begin{table}[htbp]
\begin{minipage}{\linewidth}
\setlength{\tymax}{0.5\linewidth}
\centering
\small
\begin{tabulary}{\textwidth}{|l| p{10cm}|} \toprule
 ID   &22\\


Name  &Programm bearbeiten\\
Beschreibung&Als Benutzer will ich die Möglichkeit haben ein bereits registriertes Programm zu bearbeiten. Damit ich nicht mehr aktuelle Konfigurationen ändern kann.\\
Akzeptanz &Nachdem betätigen des entsprechenden Knopfes öffnet sich ein Dialog in dem die momentane Konfiguration angezeigt wird. Die Änderungen werden nur gespeichert sofern die Änderung immer noch eine korrekte Konfiguration darstellt. Bei fehlerhaften Daten wird der Benutzer drauf hingewiesen.\\
Story Points&5\\
Entwickler &Tim\\
Iteration &3\\
Stunden  &3\\
Velocity &1.7 StoryPoints\slash Stunde\\
\bottomrule

\end{tabulary}
\end{minipage}
\end{table}



\begin{table}[htbp]
\begin{minipage}{\linewidth}
\setlength{\tymax}{0.5\linewidth}
\centering
\small
\begin{tabulary}{\textwidth}{|l| p{10cm}|} \toprule
 ID   &23\\


Name  &Datei anzeigen\\
Beschreibung&Als Benutzer will ich die Möglichkeit haben eine registrierte Datei, die zu einem Client gehört, im Webinterface einzusehen. Damit ich einen Überblick über alle Dateien habe.\\
Akzeptanz &Der Benutzer sieht die Dateien korrekt bei dem dazugehörigen Client.\\
Story Points&3\\
Entwickler &Frederik\\
Iteration &3\\
Stunden  &3\\
Velocity &1 Sp\slash h\\
\bottomrule

\end{tabulary}
\end{minipage}
\end{table}



\begin{table}[htbp]
\begin{minipage}{\linewidth}
\setlength{\tymax}{0.5\linewidth}
\centering
\small
\begin{tabulary}{\textwidth}{|l| p{10cm}|} \toprule
 ID   &24\\


Name  &Datei hinzufügen\\
Beschreibung&Als Benutzer will ich die Möglichkeit haben eine Datei für einen Client zu registrieren (in der Datenbank). Damit ich neue Dateien registrieren kann.\\
Akzeptanz &Nachdem der Benutzer den entsprechenden Knopf betätigt hat öffnet sich ein Dialog. Es werden nur korrekte Eingaben gespeichert. Falls der Benutzer nicht korrekte Daten angibt wird dies ihm mitgeteilt.\\
Story Points&4\\
Entwickler &Frederik\\
Iteration &3\\
Stunden  &4\\
Velocity &1 Sp\slash h\\
\bottomrule

\end{tabulary}
\end{minipage}
\end{table}



\begin{table}[htbp]
\begin{minipage}{\linewidth}
\setlength{\tymax}{0.5\linewidth}
\centering
\small
\begin{tabulary}{\textwidth}{|l| p{10cm}|} \toprule
 ID   &25\\


Name  &Datei löschen\\
Beschreibung&Als Benutzer will ich die Möglichkeit haben eine bereits registrierte Datei wieder zu löschen (aus der Datenbank). Damit ich veraltet Dateien entfernen kann.\\
Akzeptanz &Nachdem Betätigten des entsprechenden Knopfes öffnet sich ein Dialog, indem der Benutzer den Löschvorgang bestätigen muss. Nur falls der Benutzer den Löschvorgang bestätigt wird die registrierte Datei aus der Datenbank gelöscht.\\
Story Points&3\\
Entwickler &Frederik\\
Iteration &5\\
Stunden  &1.5\\
Velocity &2sp\slash h\\
\bottomrule

\end{tabulary}
\end{minipage}
\end{table}



\begin{table}[htbp]
\begin{minipage}{\linewidth}
\setlength{\tymax}{0.5\linewidth}
\centering
\small
\begin{tabulary}{\textwidth}{|l| p{10cm}|} \toprule
ID   &26\\


Name  &Datei editieren\\
Beschreibung&Als Benutzer will ich die Möglichkeit haben eine bereits registrierte Datei (in der Datenbank) zu bearbeiten. Damit ich auch nur kleine Änderungen einfach einspielen kann.\\
Akzeptanz &Nachdem betätigen des Knopfes öffnet sich das Dialog mit den momentanen Informationen. Nur wenn der Benutzer korrekte Änderungen angibt werden diese gespeichert. Fehlerhafte Änderungen werden dem Benutzer angezeigt.\\
Story Points&5\\
Entwickler &Frederik\\
Iteration &7\\
Stunden  &2\\
Velocity &2.5 Sp\slash h\\
\bottomrule

\end{tabulary}
\end{minipage}
\end{table}



\begin{table}[htbp]
\begin{minipage}{\linewidth}
\setlength{\tymax}{0.5\linewidth}
\centering
\small
\begin{tabulary}{\textwidth}{|l| p{10cm}|} \toprule
 ID   &27\\


Name  &Skript anzeigen\\
Beschreibung&Als Benutzer will ich die Möglichkeit haben ein registriertes Skript im Webinterface einzusehen. Damit ich einen Überblick über alle vorhanden Skripte bekomme.\\
Akzeptanz &Das Skript wird korrekt angezeigt.\\
Story Points&15\\
Entwickler &Jonas\\
Iteration &3\\
Stunden  &20\\
Velocity &0,75\\
\bottomrule

\end{tabulary}
\end{minipage}
\end{table}



\begin{table}[htbp]
\begin{minipage}{\linewidth}
\setlength{\tymax}{0.5\linewidth}
\centering
\small
\begin{tabulary}{\textwidth}{|l| p{10cm}|} \toprule
ID   &28\\


Name  &Skript hinzufügen\\
Beschreibung&Als Benutzer will ich die Möglichkeit haben ein Skript hinzuzufügen, damit ich neue Szenarien auch mit dem System laufen lassen kann.\\
Akzeptanz &Nachdem der Benutzer den entsprechenden Knopf betätigt hat, wird der Benutzer auf eine weitere Seite geleitet, wo er ein Skript erstellen kann. Bei fehlerhaften Eingaben wird der Benutzer darauf hingewiesen. Es können keine fehlerhafte Skripte hinzugefügt werden.\\
Story Points&7\\
Entwickler &Jonas\\
Iteration &4\\
Stunden  &5\\
Velocity &1.4\\
\bottomrule

\end{tabulary}
\end{minipage}
\end{table}



\begin{table}[htbp]
\begin{minipage}{\linewidth}
\setlength{\tymax}{0.5\linewidth}
\centering
\small
\begin{tabulary}{\textwidth}{|l| p{10cm}|} \toprule
ID   &29\\


Name  &Skript löschen\\
Beschreibung&Als Benutzer will ich die Möglichkeit haben ein bereits registrierte Skript zu löschen. Damit ich veraltete Skripte entfernen kann.\\
Akzeptanz &Nachdem der Benutzer den entsprechenden Knopf betätigt wird ein Dialog geöffnet, indem der Benutzer den Löschvorgang bestätigen muss. Sofern der Benutzer den Löschvorgang bestätigt wird das Skript gelöscht, ansonsten geschiet nichts.\\
Story Points&5\\
Entwickler &Jonas\\
Iteration &4\\
Stunden  &5\\
Velocity &1\\
\bottomrule

\end{tabulary}
\end{minipage}
\end{table}



\begin{table}[htbp]
\begin{minipage}{\linewidth}
\setlength{\tymax}{0.5\linewidth}
\centering
\small
\begin{tabulary}{\textwidth}{|l| p{10cm}|} \toprule
 ID   &2\\


Name  &Client Registrieren\\
Beschreibung&Als Benutzer muss ich die Möglichkeit haben einen Client zu registieren, um diese später verwalten zu können.\\
Akzeptanz &Durch das betätigen des entsprechenden Knopfes öffnet sich ein Dialog. In diesem Dialog befinden sich Felder in die der Benutzer die Daten des Clients eintragen kann. Ungültige Eingaben werden abgewiesen und der Benutzer wird darauf hingewiesen.\\
Story Points&8\\
Entwickler &Tim, Jonas, Frederik\\
Iteration &1\\
Stunden  &30\\
Velocity &0,26\\
\bottomrule

\end{tabulary}
\end{minipage}
\end{table}



\begin{table}[htbp]
\begin{minipage}{\linewidth}
\setlength{\tymax}{0.5\linewidth}
\centering
\small
\begin{tabulary}{\textwidth}{|l| p{10cm}|} \toprule
ID   &30\\


Name  &Skript bearbeiten\\
Beschreibung&Als Benutzer will ich die Möglichkeit haben ein bereits registriertes Skript zu bearbeiten. Damit ich auch kleine Änderungen einfach anpassen kann.\\
Akzeptanz &Der Benutzer wird auf die Seite weitergeleitet, nachdem der Benutzer den entsprechenden Knopf betätigt hat. Auf dieser Seite kann der der Benutzer das Skript bearbeiten kann. Falls die Änderungen des Benutzers korrekt sind, wird das Skript gespeichert. Falls das nicht der Fall ist wird dem Benutzer das mitgeteilt.\\
Story Points&5\\
Entwickler &Leonardo\\
Iteration &7\\
Stunden  &7\\
Velocity &0.71428571428\\
\bottomrule

\end{tabulary}
\end{minipage}
\end{table}



\begin{table}[htbp]
\begin{minipage}{\linewidth}
\setlength{\tymax}{0.5\linewidth}
\centering
\small
\begin{tabulary}{\textwidth}{|l| p{10cm}|} \toprule
ID   &31\\


Name  &Start Vorgang\\
Beschreibung&Als Benutzer will ich die Möglichkeit haben ein Skript meiner wahl nach einer bestimmten Zeit automatisch starten zu lassen, sobald ich auf diese Seite gehe. Damit ich nicht gezwungen bin anwesend zu sein, sobald ich den Master Rechner starte.\\
Akzeptanz &Sobald der Benutzer die Seite aufruft beginnt ein Countdown. Sobald dieser abgelaufen ist wird das zuletzt ausgeführt Skript gestartet. Sobald der Benutzer ein anderes Skript auswählt wird der Countdown abgebrochen und der Benutzer muss den ``Start Vorgang'' manuell anstoßen.\\
Story Points&15\\
Entwickler &Jonas\\
Iteration &4--5\\
Stunden  &30\\
Velocity &0.5\\
\bottomrule

\end{tabulary}
\end{minipage}
\end{table}



\begin{table}[htbp]
\begin{minipage}{\linewidth}
\setlength{\tymax}{0.5\linewidth}
\centering
\small
\begin{tabulary}{\textwidth}{|l| p{10cm}|} \toprule
ID   &32\\


Name  &Start Vorgang Status\\
Beschreibung&Als Benutzer will ich die Möglichkeit haben den momentanen Status des ``Start Vorganges'' einzusehen. Des weiteren soll die Möglichkeit bestehen zu sehen, wie weit eine einzelne Stufe fortgeschritten ist.\\
Akzeptanz &Für jede Stufe wird ein Dropdown-Menü angezeigt. Das Menü der derzeit ausgeführten Stufe wird automatisch geöffnet. In diesem Menü wird jedes Program \slash  jede Datei angezeigt, das\slash die Teil dieser Stufe ist. Falls ein Fehler während des Vorgangs entsteht wird korrekt angezeigt welches Program \slash  welche Datei ihn erzeugt hat.\\
Story Points&10\\
Entwickler &Tim\\
Iteration &9\\
Stunden  &10\\
Velocity &1\\
\bottomrule

\end{tabulary}
\end{minipage}
\end{table}



\begin{table}[htbp]
\begin{minipage}{\linewidth}
\setlength{\tymax}{0.5\linewidth}
\centering
\small
\begin{tabulary}{\textwidth}{|l| p{10cm}|} \toprule
 ID   &33\\


Name  &beenden eines Programmes\\
Beschreibung&Als Benutzer will ich die Möglichkeit haben, ein bereits registriertes und gestartetes Programm beenden zu können, damit ich diese nicht manuell beenden muss.\\
Akzeptanz &Nachdem der Benutzer das Programm seiner Wahl beendet hat, in dem er den entsprechenden Knopf betätigt hat, ändert sich der Status des Programmes und der Benutzer bekommt ggf. eine Rückmeldung über den Endstatus des Programmes.\\
Story Points&8\\
Entwickler &Tim\\
Iteration &4\\
Stunden  &9\\
Velocity &0.888 Storypoints\slash Stunde\\
\bottomrule

\end{tabulary}
\end{minipage}
\end{table}



\begin{table}[htbp]
\begin{minipage}{\linewidth}
\setlength{\tymax}{0.5\linewidth}
\centering
\small
\begin{tabulary}{\textwidth}{|l| p{10cm}|} \toprule
 ID   &34\\


Name  &Bootuptime eines Programmes\\
Beschreibung&Als Benutzer will ich die Möglichkeit haben, für eine Programm einzutragen wie lange dieses zum Starten braucht, um das gleichzeitige starten zweier Programme die voneinander abhängen zu verhindern.\\
Akzeptanz &Nachdem der Benutzer das Programm seiner Wahl beendet hat, in dem er den entsprechenden Knopf betätigt hat, ändert sich der Status des Programmes und der Benutzer bekommt ggf. eine Rückmeldung über den Endstatus des Programmes.\\
Story Points&1\\
Akzeptanz &Nachdem der Benutzer den entsprechenden Knopf zum hinzufügen\slash editieren eines Programmes betätigt hat, gibt es in dem jeweiligen Dialog ein weiteres Feld, in dem die Zeit in Sekunden eingetragen werden kann. Nach dem dies bestätigen wurde, wird der Eingrag korrekt in der Datenbank gespeichert.\\
Entwickler &Jonas\\
Iteration &4\\
Stunden  &1\\
Velocity &1\\
\bottomrule

\end{tabulary}
\end{minipage}
\end{table}



\begin{table}[htbp]
\begin{minipage}{\linewidth}
\setlength{\tymax}{0.5\linewidth}
\centering
\small
\begin{tabulary}{\textwidth}{|l| p{10cm}|} \toprule
 ID   &35\\


Name  &Überprüfen von Argumenten in Programmen\\
Beschreibung&Als Benutzer will ich die Möglichkeit haben, beim Eintragen von Argumenten eines Programmes daraufhingewiesen zu werden, wenn eine nicht parsbare Argumentenliste eingegeben wird.\\
Akzeptanz &Nachdem der Benutzer eine nicht parsbare Argumentenliste in das entsprechenden Feld während dem Hinzufügen\slash Editieren eine Programmeintrages eingetragen und bestätigt hat, erscheint eine entprechende Fehlermeldung.\\
Story Points&1\\
Entwickler &Tim\\
Iteration &4\\
Stunden  &1\\
Velocity &1 Storypoints\slash Stunde\\
\bottomrule

\end{tabulary}
\end{minipage}
\end{table}



\begin{table}[htbp]
\begin{minipage}{\linewidth}
\setlength{\tymax}{0.5\linewidth}
\centering
\small
\begin{tabulary}{\textwidth}{|l| p{10cm}|} \toprule
 ID   &36\\


Name  &Dateien vom Server herunterladen\\
Beschreibung&Als Benutzer will ich die Möglichkeit haben, Dateien, die auf dem Server in einem definierten Ordner liegen herunterzuladen.\\
Akzeptanz &Nachdem der Benutzer in der Navigationsleiste auf den ensprechenden Knopf gedrückt hat, werden alle Dateien die in dem definierten Ordner liegen mit ihrer Größe zusammen angezeigt. Diese wird dann nach dem Drücken des entsprechenden Knopfes heruntergeladen.\\
Story Points&3\\
Entwickler &Tim\\
Iteration &4\\
Stunden  &3\\
Velocity &1 Storypoints\slash Stunde\\
\bottomrule

\end{tabulary}
\end{minipage}
\end{table}



\begin{table}[htbp]
\begin{minipage}{\linewidth}
\setlength{\tymax}{0.5\linewidth}
\centering
\small
\begin{tabulary}{\textwidth}{|l| p{10cm}|} \toprule
 ID   &37\\


Name  &Logs von Programmen\\
Beschreibung&Als Benutzer will ich die Möglichkeit haben, den Log eines Programmes im Webinterface einzusehen.\\
Akzeptanz &Nachdem der Benutzer den ensprechenden Knopf neben einem Programmeintrag gedrückt hat, wird unter dem Programmeintrag die zuletzt gespeicherte Logdatei in einem Textfeld angezeigt.\\
Story Points&30\\
Entwickler &Tim\\
Iteration &5--7\\
Stunden  &40\\
Velocity &0.75\\
\bottomrule

\end{tabulary}
\end{minipage}
\end{table}



\begin{table}[htbp]
\begin{minipage}{\linewidth}
\setlength{\tymax}{0.5\linewidth}
\centering
\small
\begin{tabulary}{\textwidth}{|l| p{10cm}|} \toprule
ID   &38\\


Name  &Skript für Dateien\\
Beschreibung&Als Benutzer will ich die Möglichkeit haben auch Dateien in ein Skript mit zu bewegen.\\
Akzeptanz &Nachdem der Benutzer das Skript gestartet hat, wird werden auch die gewünschten Dateien bewegt. Falls eine kritischer Fehler bei den Dateien auftritt wird der Prozess gestoppt.\\
Story Points&5\\
Entwickler &Jonas\\
Iteration &7\\
Stunden  &12\\
Velocity &0.41\\
\bottomrule

\end{tabulary}
\end{minipage}
\end{table}



\begin{table}[htbp]
\begin{minipage}{\linewidth}
\setlength{\tymax}{0.5\linewidth}
\centering
\small
\begin{tabulary}{\textwidth}{|l| p{10cm}|} \toprule
ID   &39\\


Name  &Skript bearbeiten\\
Beschreibung&Als Benutzer will ich die Möglichkeit haben ein bereits erstelltes Skript zu kopieren dann zu bearbeiten und dann zu speichern.\\
Akzeptanz &Nachdem der Benutzer den entsprechenden Knopf für das Kopieren gedrückt hat, wird die Kopie in einem seperatem Menüeintrag angezeigt. Der Name der Kopie ist der originale Name mit dem Suffix ``\_copy''. Namenskonflikte werden mit aufsteigenden Zahlen als Zusatz zum Suffix gehandhabt.\\
Story Points&3\\
Entwickler &Tim\\
Iteration &7\\
Stunden  &1.5\\
Velocity &2\\
\bottomrule

\end{tabulary}
\end{minipage}
\end{table}



\begin{table}[htbp]
\begin{minipage}{\linewidth}
\setlength{\tymax}{0.5\linewidth}
\centering
\small
\begin{tabulary}{\textwidth}{|l| p{10cm}|} \toprule
 ID   &3\\


Name  &Hochfahren eines Clients\\
Beschreibung&Als Benutzer will ich die Möglichkeit haben einen Client mit der Hilfe des Webinterfaces hoch zu fahren, um später Programme auf diesem Client zu starten.\\
Akzeptanz &Nachdem der Benutzer den entsprechenden Knopf betätigt hat, wird der Client dazu aufgefordert zu starten und fährt ggf. dann hoch.\\
Story Points&3\\
Entwickler &Heiko\\
Iteration &2\\
Stunden  &7\\
Velocity &0.42\\
\bottomrule

\end{tabulary}
\end{minipage}
\end{table}



\begin{table}[htbp]
\begin{minipage}{\linewidth}
\setlength{\tymax}{0.5\linewidth}
\centering
\small
\begin{tabulary}{\textwidth}{|l| p{10cm}|} \toprule
ID   &40\\


Name  &Ungespeicherte Inhalte Warnung\\
Beschreibung&Als Benutzer will ich die Möglichkeit haben gewarnt zu werden sobald ich eine Eingabe verwerfen werde.\\
Akzeptanz &Der Benutzer wird mit eine Pop-Up Dialog gewarnt ob er fortfahren will sobald er eine ungespeicherte Eingabe verwerfen will.\\
Story Points&5\\
Entwickler &Heiko\\
Iteration &7\\
Stunden  &4\\
Velocity &1.25\\
\bottomrule

\end{tabulary}
\end{minipage}
\end{table}



\begin{table}[htbp]
\begin{minipage}{\linewidth}
\setlength{\tymax}{0.5\linewidth}
\centering
\small
\begin{tabulary}{\textwidth}{|l| p{10cm}|} \toprule
ID   &41\\


Name  &Skript stoppen\\
Beschreibung&Als Benutzer will ich die Möglichkeit haben ein Skript während der Ausführung zu stoppen, falls es nicht gestarttet werden sollte\slash es zu Fehlern kam.\\
Akzeptanz &Nachdem der Benutzer den entsprechenden Knopf betätigt hat, wird das Ausführen des Skripts gestoppt. Laufende Programme werden nicht beendet und der Ursprungszustand des Filesystems wird nicht wieder hergestellt.\\
Story Points&3\\
Entwickler &Frederik\\
Iteration &8\\
Stunden  &2\\
Velocity &1.5 Sp\slash h\\
\bottomrule

\end{tabulary}
\end{minipage}
\end{table}



\begin{table}[htbp]
\begin{minipage}{\linewidth}
\setlength{\tymax}{0.5\linewidth}
\centering
\small
\begin{tabulary}{\textwidth}{|l| p{10cm}|} \toprule
 ID   &42\\


Name  &Automatisches starten eines Skriptes\\
Beschreibung&Als Benutzer will ich die Möglichkeit haben beim Aufruf einer bestimmten Seite automatisch ein Skript starten zu lassen.\\
Akzeptanz &Sobald man die Seite aufruft wird ein sichtbarer Kurzeitwecker angezeigt. Sobald dieser abgelaufen ist wird das zuletzt erfolgreich getartet Skript gestartet.\\
Story Points&4\\
Entwickler &Jonas\\
Iteration &9\\
Stunden  &3\\
Velocity &1.3\\
\bottomrule

\end{tabulary}
\end{minipage}
\end{table}



\begin{table}[htbp]
\begin{minipage}{\linewidth}
\setlength{\tymax}{0.5\linewidth}
\centering
\small
\begin{tabulary}{\textwidth}{|l| p{10cm}|} \toprule
ID   &43\\


Name  &Alle Programme stoppen\\
Beschreibung&Als Benutzer will ich die Möglichkeit haben alle laufende Programme zu stoppen.\\
Akzeptanz &Nachdem der Benutzer den entsprechenden Knopf betätigt hat, werden alle derzeit laufenden Programme beendet. Verschobene Dateien werden dabei nicht verändert. Falls ein Skript ausgeführt wird, wird es beendet.\\
Story Points&4\\
Entwickler &Frederik\\
Iteration &9\\
Stunden  &3\\
Velocity &1.33\\
\bottomrule

\end{tabulary}
\end{minipage}
\end{table}



\begin{table}[htbp]
\begin{minipage}{\linewidth}
\setlength{\tymax}{0.5\linewidth}
\centering
\small
\begin{tabulary}{\textwidth}{|l| p{10cm}|} \toprule
ID   &44\\


Name  &Alle Dateien zurücksetzen\\
Beschreibung&Als Benutzer will ich die Möglichkeit haben alle vom Interface aus verschobenen Dateien zu entfernen und den Ursprungszustand des Systems wieder herzustellen.\\
Akzeptanz &Nachdem der Benutzer den entsprechenden Knopf betätigt hat, werden alle derzeit verschobenen Dateien entfernt. Wurde eine Datei beim ursprünglichen Kopieren ersetzt, soll sie wieder hergestellt werden. Ein laufendes Skript wird ebenfalls beendet.\\
Story Points&4\\
Entwickler &Frederik\\
Iteration &9\\
Stunden  &3.5\\
Velocity &1.14 sp\slash h\\
\bottomrule

\end{tabulary}
\end{minipage}
\end{table}



\begin{table}[htbp]
\begin{minipage}{\linewidth}
\setlength{\tymax}{0.5\linewidth}
\centering
\small
\begin{tabulary}{\textwidth}{|l| p{10cm}|} \toprule
ID   &45\\


Name  &Hilfstexte für Formularfelder\\
Beschreibung&Als Benutzer will ich die Möglichkeit haben mir zu allen Formularfeldern einen kleinen Hilfetext anzeigen zu lassen.\\
Akzeptanz &Nachdem der Nutzer auf das Hilfsicon geklickt hat öffnet sich ein Popup mit einem Hinweistext\\
Story Points&4\\
Entwickler &Leonardo\\
Iteration &8\\
Stunden  &4\\
Velocity &1\\
\bottomrule

\end{tabulary}
\end{minipage}
\end{table}



\begin{table}[htbp]
\begin{minipage}{\linewidth}
\setlength{\tymax}{0.5\linewidth}
\centering
\small
\begin{tabulary}{\textwidth}{|l| p{10cm}|} \toprule
ID   &46\\


Name  &Dropdown für Shutdown Funktionen\\
Beschreibung&Nach dem Klick auf ein Schutdownicon wird eine Liste aller möglichen Shutdown-Aktionen angezeigt.\\
Akzeptanz &Nachdem der Nutzer auf das Schutdownicon gelickt hat öffnet sich ein Dropdown mit Auswahlmöglichkeiten\\
Story Points&1\\
Entwickler &Leonardo\\
Iteration &9\\
Stunden  &1\\
Velocity &1\\
\bottomrule

\end{tabulary}
\end{minipage}
\end{table}



\begin{table}[htbp]
\begin{minipage}{\linewidth}
\setlength{\tymax}{0.5\linewidth}
\centering
\small
\begin{tabulary}{\textwidth}{|l| p{10cm}|} \toprule
ID   &47\\


Name  &Alles Ausschalten\\
Beschreibung&Als Benutzer will ich die Möglichkeit haben alle laufenden Clients und den Master über das Webinterface auszuschalten.\\
Akzeptanz &Nachdem der Benutzer den entsprechenden Knopf betätigt hat, werden alle Clients und der Master heruntergefahren. Falls ein Skript läuft wird dieses beendet, verschobene Dateien werden zurüchgesetzt und laufende Programme werden beendet.\\
Story Points&8\\
Entwickler &Frederik, Heiko\\
Iteration &9\\
Stunden  &20\\
Velocity &0.4 sp\slash h\\
\bottomrule

\end{tabulary}
\end{minipage}
\end{table}



\begin{table}[htbp]
\begin{minipage}{\linewidth}
\setlength{\tymax}{0.5\linewidth}
\centering
\small
\begin{tabulary}{\textwidth}{|l| p{10cm}|} \toprule
ID   &48\\


Name  &Default Skript setzten\\
Beschreibung&Als Benutzer will ich die Möglichkeit haben ein Skript manuell zu setzten welches automatisch gestartet werden kann.\\
Akzeptanz &Nachdem der Benutzer den entsprechen Knopf gedrückt hat, wird beim nächsten Neustart das ausgewählte Skript gestartet.\\
Story Points&5\\
Entwickler &Jonas\\
Iteration &9\\
Stunden  &3\\
Velocity &1,6\\
\bottomrule

\end{tabulary}
\end{minipage}
\end{table}



\begin{table}[htbp]
\begin{minipage}{\linewidth}
\setlength{\tymax}{0.5\linewidth}
\centering
\small
\begin{tabulary}{\textwidth}{|l| p{10cm}|} \toprule
 ID   &49\\


Name  &automatische Neuverbindung eines Clients\\
Beschreibung&Der Nutzer will nicht einen Client neustarten, falls dieser für eine kurze Zeit die Verbindung verloren hat. Daher soll der Client selbst versuchen die Verbing neu aufzubauen.\\
Akzeptanz &Nachdem der Client die Verbindung verloren hat, versucht dieser die Verbindung neu aufzubauen. Falls der Verbindungsaufbau fehlschlägt wird der Vorgang wiederhohlt. Danach ist der Client wieder uneingeschränkt nutzbar.\\
Story Points&3\\
Entwickler &Tim\\
Iteration &9\\
Stunden  &2\\
Velocity &1.5\\
Bemerkung &not merged\\
\bottomrule

\end{tabulary}
\end{minipage}
\end{table}



\begin{table}[htbp]
\begin{minipage}{\linewidth}
\setlength{\tymax}{0.5\linewidth}
\centering
\small
\begin{tabulary}{\textwidth}{|l| p{10cm}|} \toprule
 ID   &4\\


Name  &Status eines Clients\\
Beschreibung&Als Benutzer will ich den Status jedes bereits registrierten Clients einsehen können, um mit diesen entsprechend Interagieren zu können.\\
Akzeptanz &Der Status wird korrekt angezeigt und wird erneuert sobald sich der Status des Clients geändert hat.\\
Story Points&4\\
Entwickler &Tim\\
Iteration &3 und 4\\
Stunden  &6\\
Velocity &0.666 Storypoints\slash Stunde\\
\bottomrule

\end{tabulary}
\end{minipage}
\end{table}



\begin{table}[htbp]
\begin{minipage}{\linewidth}
\setlength{\tymax}{0.5\linewidth}
\centering
\small
\begin{tabulary}{\textwidth}{|l| p{10cm}|} \toprule
ID   &5\\


Name  &Client herunterfahren\\
Beschreibung&Als Benutzer will einen Client per Webinterface herunterfahren, um mir Arbeit zu sparen.\\
Akzeptanz &Nachdem der Benutzer den entsprechenden Knopf betätigt hat wird der Client ordnungsgemäß heruntergefahren. D.h. alle Programm die durch den Benutzer gestartet worden sind werden gestoppt und dann wird das System dazu angehalten herunter zu fahren.\\
Story Points&5\\
Entwickler &Tim\\
Iteration &3\\
Stunden  &4\\
Velocity &1.25 Storypoints\slash Stunde\\
\bottomrule

\end{tabulary}
\end{minipage}
\end{table}



\begin{table}[htbp]
\begin{minipage}{\linewidth}
\setlength{\tymax}{0.5\linewidth}
\centering
\small
\begin{tabulary}{\textwidth}{|l| p{10cm}|} \toprule
 ID   &6\\


Name  &Kontextmenü\\
Beschreibung&Als Benutzer will ich die Möglichkeit haben alle Befehle für einen Client einzusehen, damit ich eine passende Aktion wählen kann.\\
Akzeptanz &Alle Befehle sind visuell mit einem Client assoziiert.\\
Story Points&10\\
Entwickler &Leonardo, Jonas, Tim\\
Iteration &1\\
Stunden  &10\\
Velocity &1\\
Bemerkung &Obsolet\\
\bottomrule

\end{tabulary}
\end{minipage}
\end{table}



\begin{table}[htbp]
\begin{minipage}{\linewidth}
\setlength{\tymax}{0.5\linewidth}
\centering
\small
\begin{tabulary}{\textwidth}{|l| p{10cm}|} \toprule
 ID   &7\\


Name  &Starten eines Programmes\\
Beschreibung&Als Benutzer will ich die Möglichkeit haben, ein bereits registriertes Programm starten zu können, damit ich diese nicht manuell starten muss.\\
Akzeptanz &Nachdem der Benutzer das Programm seiner Wahl gestartet hat, in dem er den entsprechenden Knopf betätigt hat, ändert sich der Status des Programmes und der Benutzer bekommt ggf. eine Rückmeldung über den Endstatus des Programmes.\\
Story Points&25\\
Entwickler &Jonas, Tim\\
Iteration &3\\
Stunden  &50\\
Velocity &0.5\\
\bottomrule

\end{tabulary}
\end{minipage}
\end{table}



\begin{table}[htbp]
\begin{minipage}{\linewidth}
\setlength{\tymax}{0.5\linewidth}
\centering
\small
\begin{tabulary}{\textwidth}{|l| p{10cm}|} \toprule
 ID   &8\\


Name  &Statusabfrage von Programmen auf einem Client\\
Beschreibung&Als Benutzer will ich die Möglichkeit haben den Status eines registrierten Programmes einzusehen, damit ich mir einen Überblick verschaffen kann.\\
Akzeptanz &Der Status des Prozesses ist aktuell, korrekt und wird bei dem entsprechenden Programm, auf dem Webinterface angezeigt wird.\\
Story Points&4\\
Entwickler &Tim\\
Iteration &3 und 4\\
Stunden  &6\\
Velocity &0.666 Storypoints\slash Stunde\\
\bottomrule

\end{tabulary}
\end{minipage}
\end{table}



\begin{table}[htbp]
\begin{minipage}{\linewidth}
\setlength{\tymax}{0.5\linewidth}
\centering
\small
\begin{tabulary}{\textwidth}{|l| p{10cm}|} \toprule
 ID   &9\\


Name  &Datei bewegen\\
Beschreibung&Als Benutzer will ich die Möglichkeit haben eine registrierte Datei auf einem registrierten Client zu bewegen, damit ich einfach Konfigurationen austauschen kann. Auch soll es Möglich sein den Prozess wieder rückgänig zu machen.\\
Akzeptanz &Nachdem der Benutzer den entsprechenden Kopf betätigt hat wird die Datei auf dem Client verschoben und der Benutzer wird darüber benachrichtigt. Sofern eine Datei bereits bewegt wurde under Benutzer erneut auf den entsprechenden Knopf drückt, wird die Datei wieder entfernt (sofern diese nicht in der zwischen Zeit vom Benutzer manuell überschrieben worden ist). \\
Story Points&10\\
Entwickler &Leonardo, Jonas\\
Iteration &7\\
Stunden  &30\\
Velocity &0.33\\
\bottomrule

\end{tabulary}
\end{minipage}
\end{table}
\begin{table}[htbp]
\begin{minipage}{\linewidth}
\setlength{\tymax}{0.5\linewidth}
\centering
\small
\begin{tabulary}{\textwidth}{@{}lll@{}} \toprule
Iteration&Zeitraum&User Stories\\


1&13.11--26.11&1, 2, 6, 14, 15, 16, \\
2&27.11--10.12&3, 19, 20, 21, \\
3&11.12--07.01&5, 7, 17, 22, 23, 24, 27, \\
4&08.01--21.01&4, 8, 28, 29, 33, 34, 35, 36, \\
5&22.01--04.02&25, 31, \\
6&\multicolumn{2}{c}{05.02--18.02}\\
7&19.02--04.03&9, 26, 30, 37, 38, 39, 40, \\
8&05.03--18.03&41, 45, \\
9&19.03--28.03&32, 42, 43, 44, 46, 47, 48\\
Deprecated&inf&10, 11, 12, (49)\\
doesn't exist&error 404&13, 18, \\
\bottomrule

\end{tabulary}
\end{minipage}
\end{table}
\begin{table}[htbp]
\begin{minipage}{\linewidth}
\setlength{\tymax}{0.5\linewidth}
\centering
\small
\begin{tabulary}{\textwidth}{|l| p{10cm}|} \toprule
\multicolumn{2}{c}{ ID   }\\


\multicolumn{2}{c}{Name  }\\
\multicolumn{2}{c}{Beschreibung}\\
\multicolumn{2}{c}{Akzeptanz }\\
Story Points&?\\
Entwickler &?\\
Iteration &?\\
Stunden  &?\\
Velocity &?\\
\multicolumn{2}{c}{Bemerkung }\\
\bottomrule

\end{tabulary}
\end{minipage}
\end{table}

	% 3 Codeauszüge
	\section{Quellcodeauszug}

\definecolor{deepblue}{rgb}{0,0,0.5}
\definecolor{deepred}{rgb}{0.6,0,0}
\definecolor{deepgreen}{rgb}{0,0.5,0}
\definecolor{string-color}{rgb}{0.901, 0.290, 0.098}

\lstset
{
    language=Python,
    basicstyle=\footnotesize,
    otherkeywords={self},             % Add keywords here
    keywordstyle=\color{deepblue},
    commentstyle=\color{string-color},
    emph={MyClass,__init__},          % Custom highlighting
    emphstyle=\color{deepred},    % Custom highlighting style
    stringstyle=\color{deepgreen},
    numbers=left,
    stepnumber=1,
    tabsize=2,
    breaklines=true,
}

Das von uns verwendete Webframework erlaubt die funktionale und objektorientierte Programmierung. Wir haben uns für die funktionale Programmierung
entschieden, weswegen wir drei Module (anstatt drei Klassen) angehängt haben, die repräsentativ für die meisten
anderen Module des Projekts stehen.

\subsection{Das Modul Api.py}
%Description
Das Modul Api.py enthält Funktionen, die sich um die Anfragen an die REST API kümmern.
Da das Webinterface seine Anfragen direkt an die REST API sendet, ist diese Modul ein essentieller
Bestandteil des Backends.

Da das Modul sehr groß ist, wurde hier nur ein kleiner Ausschnitt angehängt.
Die vollständige Fassung ist im Repository zu finden\footnote{\url{https://github.com/bp-flugsimulator/server/blob/master/frontend/api.py}}.
\lstinputlisting[language=Python, lastline=308]{api.py}

\subsection{Das Modul Consumers.py}
Da sowohl für die Kommunikation zwischen Clients und Master als auch zwischen Browser und
Master \textit{websockets}\footnote{\url{https://tools.ietf.org/html/rfc6455}} verwendet werden,
ist die Behandlung von eingehenden Websocketrequests ein zentraler Bestandteil. Dies ist die
Aufgabe des Moduls "`consumer.py"', welches als erste Instanz Nachrichten an den Master annimmt und
weiter verteilt.
Da das Modul sehr groß ist, wurde hier nur ein kleiner Ausschnitt angehängt.
Die vollständige Fassung ist im Repository zu finden\footnote{\url{https://github.com/bp-flugsimulator/server/blob/master/frontend/consumers.py}}.

\lstinputlisting[language=Python, lastline=336]{consumers.py}

\subsection{Das Modul Controller.py}
Die Steuerung der angeschlossenen Clients ist einer der Hauptaufgaben des
Master. Dies besteht vorallem daraus Datenbankeinträge zu modifizieren und
basierend darauf Nachrichten über die Websockets zu verschicken. Um weiteren
Komponenten der Mastersoftware, die Arbeit mit den Clients zu erleichtern,
wurden die möglichen Aktionen im "`controller.py"'-Modul implementiert. Dies
entspricht dem Model-View-Controller Konzept.

Da das Modul sehr groß ist, wurde hier nur ein kleiner Ausschnitt angehängt.
Die vollständige Fassung ist im Repository zu finden\footnote{\url{https://github.com/bp-flugsimulator/server/blob/master/frontend/controller.py}}.

\lstinputlisting[language=Python,lastline=232]{controller.py}

	% Manual
	\section{Benutzerhandbuch}
	Wie im QS-Ziel "`Bedienbarkeit"' bereits beschrieben, wurde ein Benutzerhandbuch
	für die entwickelte Software erstellt. Dieses wurde in enger Zusammenarbeit mit
	den Auftraggebern und aus dem Feedback der Nutzerstudien erstellt und bildet die
	Grundlage für die Bedienung des Simulators und der Software, gerade da einige
	zugrundeliegenden Konzepte nicht unbedingt intuitiv sind.
	\includepdf[pages={-}]{../bedienungsanleitung/manual.pdf}
	% Details zu allen Nutzer*innenstudien
	\section{Nutzer*innenstudien}
Wie bereits in der Beschreibung des Qualitätsziels Bedienbarkeit (Abschnitt \ref{bedienbarkeit_qs})
erwähnt, wurden während des Projekts zwei Nutzer*innenstudien durchgeführt. Diese wurde sowohl mit
Mitarbeiter*innen des Fachgebiets als auch mit Studierenden aus anderen Fachbereichen durchgeführt.
Dadurch wurde getestet, ob die Oberfläche sowohl für Personen mit Flugsimulatorefahrung als auch
von Laien bedienbar ist.
\\\\
Beide Nutzer*innenstudien hatten jeweils einen anderen Fokus:
Die erste Studie konzentrierte sich auf das Aufsetzen der Software, während in der zweiten Studie
die Ablaufprogrammierung von Startsequenzen im Mittelpunkt stand. Daher ergaben sich zwei unterschiedliche
Fragebögen.
\\\\
Eine Zusammenfassung der Ergebnisse findet sich jeweils nach dem Fragebogen.

\subsection{Nutzer*innenstudie 1}
Die erste Studie hatte das Ziel zu testen, ob die vorgesehenen Arbeitsabläufe für die Nutzer*innen
intuitiv durchführbar waren. Zudem wurde zum ersten Mal die Oberfläche der Software Mitarbeitenden und Studierenden
vorgeführt, welche nicht Teil des BP-Teams bzw. Auftraggeber*innen waren. Sie fand am 12.02.2018 statt.
\includepdf[pages={1-},scale=0.9]{../Nutzerstudie/Studie1/Aufgaben.pdf}
\includepdf[pages={1-},scale=0.9]{../Nutzerstudie/Studie1/fragebogen.pdf}
\subsubsection{Ergebnisse}
Viele Teilnehmenden an der Studie wünschten sich kleinere Designänderungen, beispielsweise die Vergrößerung von
Buttons oder ähnliche kleinere Veränderungen an der Oberfläche. Dies wurden vom Team während einer Refactoring-Phase
eingearbeitet. Die sonstigen Maßnahmen wurden in Absprache mit den Auftraggeber*innen zu Userstorys umgewandelt und
in der nächsten Iteration fertiggestellt. Die Zuweisung findet sich in Tabelle \ref{studie1_us}
\\\\
{\large Feedback der Nutzer*innen\\}
\begin{enumerate}
\item Allgemein
\begin{enumerate}
	\item Alle Slaves in Clients umbenennen (gefunden in den Skripten)
	\item Übersichtsseite mit allen Events, nach Clients sortiert
	\item Tooltips für alle Felder
	\item Pluszeichen zu klein/alternatives Symbol oder Satz verwenden
	\item (Event-)Log-Dateien
	\item Inputhints
	\item Dateien und Programme über Clientgrenzen kopieren
	\item Abtrennung zwischen den Feldern verstärken
	\item Breiter Statusbalken
	\item Home Button
	\item Mehr Bilder/Icons
\end{enumerate}
\item Anlegen eines Programms
\begin{enumerate}
	\item Startzeit ist nicht erklärt, Wert -1 verwirrt
	\item Dateisystembrowser
\end{enumerate}
\item Anlegen eines Clients
\begin{enumerate}
	\item Nach Anlegen eines Clients soll dieser automatisch ausgewählt werden
	\item Verfügbare IP/MAC-Adressen anzeigen
	\item Beschreibungstext
\end{enumerate}
\item Anlegen einer Datei
\begin{enumerate}
	\item Anzeigename in Name umbennen
\end{enumerate}
\item Anlegen eines Scripts
\begin{enumerate}
	\item Nachfrage bevor Tab/Browser geschlossen wird
	\item alternativ autosave
	\item Apply statt edit
\end{enumerate}
\item Run Script
\begin{enumerate}
	\item Run Knopf hervorheben
	\item Reload nach Klick auf 'Run Script'
	\item Gesamtprozess in Skript anzeigen
\end{enumerate}
\item Delete Fenster
\begin{enumerate}
	\item Statt OK explizit delete schreiben
\end{enumerate}
\end{enumerate}

\begin{table}[H]
	\centering
	\begin{tabular}{c|c}
		Anmerkung & Userstory \\\hline
		1.c & 45\\
		1.e & 37\\
		2.a & 45\\
		3.c & 45\\
		5.a & 40\\
		5.c & 37\\
	\end{tabular}
	\caption{Zuordnung von Anmerkung zu Userstory}
	\label{studie1_us}
\end{table}

\subsection{Nutzer*innenstudie 2}
Das Ziel dieser Studie war das Feintuning der Oberfläche und das Testen des neuen
Skripteditors. Die Studie fand am 12.03.2018 statt. Die Studie wurde durchgeführt,
obwohl das Projekt bereits kurz vor Ende stand, da die Auftraggeber*innen eine
Weiterentwicklung wünschten und die Ergebnisse an das nachfolgende Entwickler*innen
weitergegeben werden.
\includepdf{../Nutzerstudie/Studie2/Aufgaben.pdf}
\includepdf{../Nutzerstudie/Studie2/fragebogen.pdf}
\subsubsection{Ergebnisse}
Diese Studie ergab sehr wenig Feedback, da alle Anmerkungen der ersten Studie bereits
eingearbeitet wurden. Die Designänderungen wurden wie in der ersten Nutzer*innenstudie
vom Team eingearbeitet. Die Zuweisung zu den passenden Userstorys existiert hier nicht,
da sich keine Userstorys aus dieser Studie ergeben haben.
\\\\
{\large Feedback der Nutzer*innen\\}
\begin{enumerate}
\item Interface
\begin{enumerate}
	\item Reset Filesystem in Restore Filesystem umbenennen.
	\item Deutsche Übersetzung
	\item INDEX bei Skripterstellung durch einfacheren Begriff ersetzen (Starting order)
	\item Namensgebung insgesamt überarbeiten
	\item Bei fehlerhaften Pfaden bleibt der Log leer
	\item Gestoppte Programme irgendwann zurücksetzen bzw. roter Statusbalken entfernen
	\item Kurzanleitungen innerhalb der Oberfläche anzeigen
\end{enumerate}
\item Run
\begin{enumerate}
	\item Nur anzeigen, wenn Skript läuft und in Running oder Running Script umbennen
\end{enumerate}
\item Program
\begin{enumerate}
	\item Bei Arguments vermerken, dass \textbackslash\textbackslash statt \textbackslash verwendet werden muss.
	\item Zeigt den Status vom Datenpool überall an/Benachrichtigung über Programmabsturz
	\item Übersicht aller laufenden Programme
\end{enumerate}
\item Forms
\begin{enumerate}
		\item Felder sind teilweise zu groß (Programpfad), wodurch die Form größer als ein Bildschirm ist
\end{enumerate}
\end{enumerate}

	% Automatische Tests und wie sie aufgebaut sind
	\section{Automatisierte Tests}
Während der Entwicklung wurden diverse automatisierte Tools verwendet, welche direkt mit der Versionskontrollsoftware
verbunden waren. Dadurch wurde jeder Commit mit den beschriebenen Tools (siehe Tabelle \ref{qs_tools}) überprüft.
\begin{table}[H]
	\centering
	\begin{tabular}{|l|p{18em}|l|}
		\hline
		\textbf{Tool} & \textbf{Beschreibung} & \textbf{Notwendig für...} \\\hline
		AppVeyor & Ausführung von Unittests unter verschiedenen Windowsversionen & Portabilität, Korrektheit \\\hline
		Codacy & Überprüfung der Codequalität basierend auf Styleguides der verschiedenen Sprachen& keine \\\hline
		Travis CI & Ausführung von Unittests unter verschiedenen Linux- und Pythonversionen & Portabilität, Korrektheit \\\hline
		coveralls & Überprüfung der Testabdeckung & Korrektheit \\\hline
		pyup.io & Überprüfung der verwendeten Bibliotheksversionen auf Updates & keine \\\hline
	\end{tabular}
	\caption{Automatisierte Tools für die Qualitätssicherung}
	\label{qs_tools}
\end{table}

Durch die Integration in die Github-Infrastruktur wurde für jeden Pull-Request (in unserem Modell entsprach ein Pull-Request einer Userstory)
ein entsprechender Report generiert (Beispiel in Abbildung \ref{github_checks}).

\begin{figure}[h]
	\centering
	\includegraphics[width=.7\textwidth]{github_checks}
	\caption{Übersicht der durchlaufenden Checks}
	\label{github_checks}
\end{figure}

Die Unittests, welche von AppVeyor und Travis CI ausgeführt wurden, deckten große Teile des Python Codes ab.
Die Abdeckung lag immer bei über 90\% und war ein Kriterium für die Abnahme einer Userstory. Eine Übersicht der Abdeckung
findet sich in Abbildung \ref{coverage_pdf}.
Die Tests fanden unter folgenden Umgebungen statt:
\begin{itemize}
	\item Windows (Windows Server 2016, jeder Test mit 32 und 64 Bit Versionen)
		\begin{itemize}
			\item Python 3.4
			\item Python 3.5
			\item Python 3.6
		\end{itemize}
	\item Linux (Ubuntu, 64 Bit)
		\begin{itemize}
			\item Python 3.4
			\item Python 3.5
			\item Python 3.6
		\end{itemize}
\end{itemize}
\begin{figure}[t]
	\centering
\includegraphics[width=.8\textwidth]{img/coverage.pdf}
	\caption{Abdeckungsübersicht am Ende des Projekts}
	\label{coverage_pdf}
\end{figure}
Der Stand von Codacy zum Ende des Projekts findet sich in Abbildung \ref{codacy_png}. Während
es keine stilistischen Fehler gibt ('Issues 0\%'), gibt es sehr viel doppelten Code, da viele Methoden
in der API einen ähnlichen bis gleichen Anfang haben.

Ein erfolgreicher Testrun von Travis\footnote{\url{https://travis-ci.org/bp-flugsimulator}} ist in Abbildung \ref{travis_png} zu sehen, von AppVeyor\footnote{\url{https://ci.appveyor.com/project/GreenM0nst3r/server/history}} in Abbildung \ref{appveyor_png}.
Die Ergebnisse finden sich unter 


\begin{figure}[t]
	\centering
\includegraphics[width=.9\textwidth]{codacy}
	\caption{Stilüberprüfung am Ende des Projekts}
	\label{codacy_png}
\end{figure}

\begin{figure}[h]
	\centering
\includegraphics[width=.9\textwidth]{travis}
	\caption{Erfolgreicher Testdurchlauf von Travis}
	\label{travis_png}
\end{figure}

\begin{figure}[h]
	\centering
\includegraphics[width=.9\textwidth]{appveyor}
	\caption{Erfolgreicher Testdurchlauf von AppVeyor}
	\label{appveyor_png}
\end{figure}

	% Details zu den Iterationen
	\section{QS-Prozess nach Iteration}
So wie das Projekt ist auch der QS-Prozess im Projektverlauf angewachsen. Es wurden neue Tools hinzugefügt
und neue Testumgebungen entwickelt, da die Auftraggeber ihre Anforderungen geändert haben. Daher gibt
es nicht für alle Iterationen vollständige Testausgaben. Zudem haben einige Tools (AppVeyor und Travis CI)
die gleichen Tests durchgeführt. Dort wurde dann nur ein Output angehängt.

Die Iterationslänge betrug zwei Wochen.

\subsubsection{1. Iteration}
\begin{table}[H]
\begin{center}
	\begin{tabular}{| l | l |}
		\hline
		\textbf{Zeitraum} & 13.11.2017 - 26.11.2017\\\hline
		\textbf{Abgegebene Userstorys} & 1, 2, 6, 14, 15, 16 \\\hline
		\textbf{Commit Hash} & \texttt{f3bbc82b0eb9ef5c0159843152fc54fb7c5c8309} \\\hline
	\end{tabular}
	\caption{Übersicht 1. Iteration}
\end{center}
\end{table}

\subsubsection{Testoutput (\texttt{python manage.py test})}
\lstinputlisting{test_output/01_iteration_python}

\subsubsection{2. Iteration}
\begin{table}[H]
\begin{center}
	\begin{tabular}{| l | l |}
		\hline
		\textbf{Zeitraum} & 27.11.2017 - 10.12.2017\\\hline
		\textbf{Abgegebene Userstorys} & 3, 19, 20, 21\\\hline
		\textbf{Commit Hash} & \texttt{eef5badcf7928d55461bfddb40c3f7f7194a0fa3} \\\hline
	\end{tabular}
	\caption{Übersicht 2. Iteration}
\end{center}
\end{table}
\subsubsection{Testoutput (\texttt{python manage.py test})}
\lstinputlisting{test_output/02_iteration_python}

\subsubsection{3. Iteration}
\begin{table}[H]
\begin{center}
	\begin{tabular}{| l | l |}
		\hline
		\textbf{Zeitraum} & 11.12.2017 - 07.01.2018\\\hline
		\textbf{Abgegebene Userstorys} & 5, 7, 17, 22, 23, 24, 27\\\hline
		\textbf{Commit Hash} & \texttt{b5c84ab3496ade220316b8b119c7fddc3d49d383} \\\hline
	\end{tabular}
	\caption{Übersicht 3. Iteration}
\end{center}
\end{table}
\subsubsection{Testoutput (\texttt{python manage.py test})}
\lstinputlisting{test_output/03_iteration_python}
\subsubsection{Coverage}
\begin{figure}[H]
	\centering
\includegraphics[width=.9\textwidth]{test_output/03_iteration_coverage.pdf}
	\caption{Coverage in Iteration 3}
\end{figure}

\subsubsection{4. Iteration}
\begin{table}[H]
\begin{center}
	\begin{tabular}{| l | l |}
		\hline
		\textbf{Zeitraum} & 08.01.2018 - 21.01.2018\\\hline
		\textbf{Abgegebene Userstorys} & 4, 8, 28, 29, 33, 34, 35, 36, \\\hline
		\textbf{Commit Hash} & \texttt{b06b3f8b23a022a127061c2dae05215c0bc21f23} \\\hline
	\end{tabular}
	\caption{Übersicht 4. Iteration}
\end{center}
\end{table}
\subsubsection{Testoutput (\texttt{python manage.py test})}
\lstinputlisting{test_output/04_iteration_python}
\subsubsection{Coverage}
\begin{figure}[H]
	\centering
\includegraphics[width=.9\textwidth]{test_output/04_iteration_coverage.pdf}
	\caption{Coverage in Iteration 4}
\end{figure}

\subsubsection{5. Iteration}
\begin{table}[H]
\begin{center}
	\begin{tabular}{| l | l |}
		\hline
		\textbf{Zeitraum} &  22.01.2018 - 04.02.2018\\\hline
		\textbf{Abgegebene Userstorys} & 25, 31\\\hline
		\textbf{Commit Hash} & \texttt{8a194372e807cc043f3ea705671d515fd0fe0bb5} \\\hline
	\end{tabular}
	\caption{Übersicht 5. Iteration}
\end{center}
\end{table}
\subsubsection{Testoutput }
\lstinputlisting{test_output/05_iteration_python}
\subsubsection{Coverage}
\begin{figure}[H]
	\centering
\includegraphics[width=.9\textwidth]{test_output/05_iteration_coverage.pdf}
	\caption{Coverage in Iteration 5}
\end{figure}

\subsubsection{6. Iteration}
\begin{table}[H]
\begin{center}
	\begin{tabular}{| l | l |}
		\hline
		\textbf{Zeitraum} &  05.02.2018 - 18.02.2018\\\hline
		\textbf{Abgegebene Userstorys} & keine\\\hline
		\textbf{Commit Hash} & - \\\hline
	\end{tabular}
	\caption{Übersicht 6. Iteration}
\end{center}
\end{table}
In dieser Iteration wurden alle Userstorys von den Auftraggebern abgelehnt und wurden daher in der nächsten Iteration mit
den angemerkten Änderungen neu eingereicht.

\subsubsection{7. Iteration}
\begin{table}[H]
\begin{center}
	\begin{tabular}{| l | l |}
		\hline
		\textbf{Zeitraum} &  19.02.2018 - 04.03.2018\\\hline
		\textbf{Abgegebene Userstorys} & 9,26,30,37,38,39,40\\\hline
		\textbf{Commit Hash} & \texttt{6003099c26804bff6995ac4ff0d0d5571dfc2840} \\\hline
	\end{tabular}
	\caption{Übersicht 7. Iteration}
\end{center}
\end{table}
\subsubsection{Testoutput }
\lstinputlisting{test_output/07_iteration_python}
\subsubsection{Coverage}
\begin{figure}[H]
	\centering
\includegraphics[width=.9\textwidth]{test_output/07_iteration_coverage.pdf}
	\caption{Coverage in Iteration 7}
\end{figure}

\subsubsection{8. Iteration}
\begin{table}[H]
\begin{center}
	\begin{tabular}{| l | l |}
		\hline
		\textbf{Zeitraum} &  05.03.2018 - 18.03.2018\\\hline
		\textbf{Abgegebene Userstorys} & 41, 45\\\hline
		\textbf{Commit Hash} & \texttt{13351d683a279c48b2890d23add9bfa57963fea8} \\\hline
	\end{tabular}
	\caption{Übersicht 8. Iteration}
\end{center}
\end{table}
\subsubsection{Testoutput }
\lstinputlisting{test_output/08_iteration_python}
\subsubsection{Coverage}
\begin{figure}[H]
	\centering
\includegraphics[width=.9\textwidth]{test_output/08_iteration_coverage.pdf}
	\caption{Coverage in Iteration 8}
\end{figure}

\subsubsection{9. Iteration}
\begin{table}[H]
\begin{center}
	\begin{tabular}{| l | l |}
		\hline
		\textbf{Zeitraum} &  19.03.2018 - 28.03.2018\\\hline
		\textbf{Abgegebene Userstorys} & 32, 42, 43, 44, 46, 47, 48\\\hline
		\textbf{Commit Hash} & \texttt{459b7edca36d0602017a1d724e2f159ce544aa1d} \\\hline
	\end{tabular}
	\caption{Übersicht 9. Iteration}
\end{center}
\end{table}
\subsubsection{Testoutput }
\lstinputlisting{test_output/09_iteration_python}
\subsubsection{Coverage}
\begin{figure}[H]
	\centering
\includegraphics[width=.9\textwidth]{test_output/09_iteration_coverage.pdf}
	\caption{Coverage in Iteration 9}
\end{figure}

	% Ausgefüllte Code Reviews
	\section{Code Reviews}
Nach der Implementierung einer Userstory mussten alle Änderungen am Code von einem weiteren Entwickler
gesichtet werden. Dabei wurde folgende Checkliste als Referenz verwendet:
\begin{itemize}
	\item Test Coverage (auf unseren Dateien) mindestens 90\%
	\item Jede Klasse und Funktion ist grob Dokumentiert
	\item Schwierige Codestellen sind dokumentiert
	\item Alle Test laufen fehlerfrei durch
	\item Der Code ist korrekt formatiert
	\item Code erfüllt das Akzeptanzkriterium der Userstory (und dabei spezifisch nur das Akzeptanzkriterium dieser Userstory)
	\item Die Userstory ist ausgefüllt (Datum, Zeit)
\end{itemize}
Es folgen die ausgefüllten Checklisten. Da die ersten beiden US in gemeinsamer Arbeit entstanden ist, gibt es keine Checkliste dafür.
\begin{figure}[H]
\centering
\includegraphics[width=.8\textwidth]{code_review/us03}
	\caption{Review zur Userstory 3}
\end{figure}

\begin{figure}[H]
\centering
\includegraphics[width=.8\textwidth]{code_review/us04}
\caption{Review zur Userstory 04}
\end{figure}

\begin{figure}[H]
\centering
\includegraphics[width=.8\textwidth]{code_review/us05}
\caption{Review zur Userstory 05}
\end{figure}

\begin{figure}[H]
\centering
\includegraphics[width=.8\textwidth]{code_review/us06}
\caption{Review zur Userstory 06}
\end{figure}

\begin{figure}[H]
\centering
\includegraphics[width=.8\textwidth]{code_review/us07}
\caption{Review zur Userstory 07}
\end{figure}

\begin{figure}[H]
\centering
\includegraphics[width=.8\textwidth]{code_review/us08}
\caption{Review zur Userstory 08}
\end{figure}

\begin{figure}[H]
\centering
\includegraphics[width=.8\textwidth]{code_review/us09}
\caption{Review zur Userstory 09}
\end{figure}

\begin{figure}[H]
\centering
\includegraphics[width=.8\textwidth]{code_review/us14}
\caption{Review zur Userstory 14}
\end{figure}

\begin{figure}[H]
\centering
\includegraphics[width=.8\textwidth]{code_review/us15}
\caption{Review zur Userstory 15}
\end{figure}

\begin{figure}[H]
\centering
\includegraphics[width=.8\textwidth]{code_review/us16}
\caption{Review zur Userstory 16}
\end{figure}

\begin{figure}[H]
\centering
\includegraphics[width=.8\textwidth]{code_review/us17}
\caption{Review zur Userstory 17}
\end{figure}

\begin{figure}[H]
\centering
\includegraphics[width=.8\textwidth]{code_review/us19}
\caption{Review zur Userstory 19}
\end{figure}

\begin{figure}[H]
\centering
\includegraphics[width=.8\textwidth]{code_review/us20}
\caption{Review zur Userstory 20}
\end{figure}

\begin{figure}[H]
\centering
\includegraphics[width=.8\textwidth]{code_review/us21}
\caption{Review zur Userstory 21}
\end{figure}

\begin{figure}[H]
\centering
\includegraphics[width=.8\textwidth]{code_review/us22}
\caption{Review zur Userstory 22}
\end{figure}

\begin{figure}[H]
\centering
\includegraphics[width=.8\textwidth]{code_review/us23}
\caption{Review zur Userstory 23}
\end{figure}

\begin{figure}[H]
\centering
\includegraphics[width=.8\textwidth]{code_review/us24}
\caption{Review zur Userstory 24}
\end{figure}

\begin{figure}[H]
\centering
\includegraphics[width=.8\textwidth]{code_review/us25}
\caption{Review zur Userstory 25}
\end{figure}

\begin{figure}[H]
\centering
\includegraphics[width=.8\textwidth]{code_review/us26}
\caption{Review zur Userstory 26}
\end{figure}

\begin{figure}[H]
\centering
\includegraphics[width=.8\textwidth]{code_review/us27}
\caption{Review zur Userstory 27}
\end{figure}

\begin{figure}[H]
\centering
\includegraphics[width=.8\textwidth]{code_review/us28}
\caption{Review zur Userstory 28}
\end{figure}

\begin{figure}[H]
\centering
\includegraphics[width=.8\textwidth]{code_review/us29}
\caption{Review zur Userstory 29}
\end{figure}

\begin{figure}[H]
\centering
\includegraphics[width=.8\textwidth]{code_review/us30}
\caption{Review zur Userstory 30}
\end{figure}

\begin{figure}[H]
\centering
\includegraphics[width=.8\textwidth]{code_review/us31}
\caption{Review zur Userstory 31}
\end{figure}

\begin{figure}[H]
\centering
\includegraphics[width=.8\textwidth]{code_review/us32}
\caption{Review zur Userstory 32}
\end{figure}

\begin{figure}[H]
\centering
\includegraphics[width=.8\textwidth]{code_review/us33}
\caption{Review zur Userstory 33}
\end{figure}

\begin{figure}[H]
\centering
\includegraphics[width=.8\textwidth]{code_review/us34}
\caption{Review zur Userstory 34}
\end{figure}

\begin{figure}[H]
\centering
\includegraphics[width=.8\textwidth]{code_review/us35}
\caption{Review zur Userstory 35}
\end{figure}

\begin{figure}[H]
\centering
\includegraphics[width=.8\textwidth]{code_review/us36}
\caption{Review zur Userstory 36}
\end{figure}

\begin{figure}[H]
\centering
\includegraphics[width=.8\textwidth]{code_review/us37}
\caption{Review zur Userstory 37}
\end{figure}

\begin{figure}[H]
\centering
\includegraphics[width=.8\textwidth]{code_review/us38}
\caption{Review zur Userstory 38}
\end{figure}

\begin{figure}[H]
\centering
\includegraphics[width=.8\textwidth]{code_review/us39}
\caption{Review zur Userstory 39}
\end{figure}

\begin{figure}[H]
\centering
\includegraphics[width=.8\textwidth]{code_review/us40}
\caption{Review zur Userstory 40}
\end{figure}

\begin{figure}[H]
\centering
\includegraphics[width=.8\textwidth]{code_review/us41}
\caption{Review zur Userstory 41}
\end{figure}

\begin{figure}[H]
\centering
\includegraphics[width=.8\textwidth]{code_review/us42}
\caption{Review zur Userstory 42}
\end{figure}

\begin{figure}[H]
\centering
\includegraphics[width=.8\textwidth]{code_review/us43}
\caption{Review zur Userstory 43}
\end{figure}

\begin{figure}[H]
\centering
\includegraphics[width=.8\textwidth]{code_review/us44}
\caption{Review zur Userstory 44}
\end{figure}

\begin{figure}[H]
\centering
\includegraphics[width=.8\textwidth]{code_review/us45}
\caption{Review zur Userstory 45}
\end{figure}

\begin{figure}[H]
\centering
\includegraphics[width=.8\textwidth]{code_review/us46}
\caption{Review zur Userstory 46}
\end{figure}

\begin{figure}[H]
\centering
\includegraphics[width=.8\textwidth]{code_review/us47}
\caption{Review zur Userstory 47}
\end{figure}

\begin{figure}[H]
\centering
\includegraphics[width=.8\textwidth]{code_review/us48}
\caption{Review zur Userstory 48}
\end{figure}
\missingfigure{Restliche Reviews zu Userstorys hinzufügen}

	% Projekttagebuch
	\documentclass[accentcolor=tud9c,12pt,paper=a4]{tudreport}

\usepackage[utf8]{inputenc}
\usepackage{ngerman}
\usepackage{parcolumns}
\usepackage{hyperref}

\newcommand{\titlerow}[2]{
	\begin{parcolumns}[colwidths={1=.17\linewidth}]{2}
		\colchunk[1]{#1:}
		\colchunk[2]{#2}
	\end{parcolumns}
	\vspace{0.2cm}
}

\title{Steuerungsprogramm für einen Flugsimulator}
\subtitle{Projekttagebuch}
\subsubtitle{%
	\titlerow{Gruppe 19}{%
		Frederik Bark <frederikalexander.bark@stud.tu-darmstadt.de>\\
		Heiko Carrasco <heiko.carrascohuertas@stud.tu-darmstadt.de>\\
		Jonas Meurer <jonas.meurer@stud.tu-darmstadt.de>\\
		Tim Weißmantel <tim.weissmantel@stud.tu-darmstadt.de>\\
		Leonardo Zaninelli <leonardo.zaninelli@stud.tu-darmstadt.de>}
	\titlerow{Teamleiter}{Hendrik Bode <hendrik.bode@stud.tu-darmstadt.de>}
	\titlerow{Auftraggeber}{%
		Jonas Schulze <Schulze@fsr.tu-darmstadt.de>\\
		Torben Bernatzky <Bernatzky@fsr.tu-darmstadt.de>\\
		Technische Universität Darmstadt\\
		Flugsysteme und Regelungstechnik}
	\titlerow{Abgabedatum}{31.03.2018}}
\institution{Bachelor-Praktikum WS-2017/18\\ Fachbereich Informatik}

\begin{document}

	\maketitle
	\tableofcontents

	\chapter{Lizenz}
		Da sich die Auftraggeber eine größtmögliche Freiheit im Bezug
		auf die Verwendung des Codes gewünscht haben,
		wurde sämtlicher Code unter die MIT-Lizenz
		\footnote{\url{https://github.com/bp-flugsimulator/server/blob/master/LICENSE.md}}
		gestellt.
		
	\chapter{Projektgefährdende Ereignisse}
	\textit{keine}


\end{document}



\end{document}
