\documentclass[accentcolor=tud9c,12pt,paper=a4]{tudreport}

\usepackage[utf8]{inputenc}
\usepackage{ngerman}
\usepackage{parcolumns}
\usepackage{hyperref}

\newcommand{\titlerow}[2]{
	\begin{parcolumns}[colwidths={1=.17\linewidth}]{2}
		\colchunk[1]{#1:}
		\colchunk[2]{#2}
	\end{parcolumns}
	\vspace{0.2cm}
}

\title{Steuerungsprogramm für einen Flugsimulator}
\subtitle{Qualitätssicherungsdokument}
\subsubtitle{%
	\titlerow{Gruppe 19}{%
		Frederik Bark <frederikalexander.bark@stud.tu-darmstadt.de>\\
		Heiko Carrasco <heiko.carrascohuertas@stud.tu-darmstadt.de>\\
		Jonas Meurer <jonas.meurer@stud.tu-darmstadt.de>\\
		Tim Weißmantel <tim.weissmantel@stud.tu-darmstadt.de>\\
		Leonardo Zaninelli <leonardo.zaninelli@stud.tu-darmstadt.de>}
	\titlerow{Teamleiter}{Hendrik Bode <hendrik.bode@stud.tu-darmstadt.de>}
	\titlerow{Auftraggeber}{%
		Jonas Schulze <Schulze@fsr.tu-darmstadt.de>\\
		Torben Bernatzky <Bernatzky@fsr.tu-darmstadt.de>\\
		Technische Universität Darmstadt\\
		Flugsysteme und Regelungstechnik}
	\titlerow{Abgabedatum}{31.03.2018}}
\institution{Bachelor-Praktikum WS-2017/18\\ Fachbereich Informatik}

\begin{document}

	\maketitle
	\tableofcontents

	\chapter{Einleitung}
		Die Auftraggeber sind im Besitz eines Flugsimulators, der aus vielen 
		Teilprogrammen besteht. Diese Teilprogramme sind über mehrere Computer im
		gleichen Netzwerk verteilt und müssen alle in einer komplexen Reihenfolge per Hand 
		gestartet werden. Aus diesem Grund wünschen sich die Auftraggeber eine Software, welche
		das Starten des Simulators automatisiert.\\[5pt]
		Es ist dabei zu beachten, dass auf den Teilsystemen unterschiedliche 
		Betriebsysteme benutzt werden (verschiedene Windowsversionen und ggf. Linux). 
		Des Weiteren ist es erwünscht, einen generischer Ansatz zu wählen, um das Einfügen 
		einer neuen Startreihenfolge im Nachhinein zu vereinfachen.\\[5pt]
		Hierzu wird ein Master/Slave-System eingesetzt, wobei über einen Rechner alle 
		Systeme, die Teil des Simulators sind, zentral gesteuert werden. Ein Nutzer kann dann über
		ein Webinterface diesen Master konfigurieren und steuern.\\[5pt]
		Die Software für Master und Slaves ist in Python geschrieben und benutzt
		das Django Framework für das Webinterface, sowie RPC zur Kommunikation zwischen 
		Master und Slaves.
		

	\chapter{Qualitätsziele}
		\section{Portabilität}
		Wie in der Einleitung erwähnt, muss die Slave Software unter verschiedenen 
		Betriebsystemen laufen (Windows XP, Windows 7, Windows 10 und ggf. Linux).
		Mit den Auftraggebern wurde abgesprochen, dass der Master keine 
		Betriebsystemversion unter Windows 7 einsetzt, weswegen die Tests für Windows
		XP hier entfallen.
		\\[5pt]
		Vorraussetzung für die Abnahme ist eine 90\% Testabdeckung des durch das
		BP-Teams geschriebenen Codes.
		Um Portabilität sicherzustellen, werden alle Tests auf allen eingesetzten
		Betriebssystemen nach jedem Commit ausgeführt. Spätestens zur Abgabe einer
		US müssen diese Tests fehlerfrei durchlaufen, sonst wird die Userstory nicht
		zur Abnahme präsentiert. Die Entscheidung zur Präsentation trifft der für diese
		Userstory bestimmte Reviewer. Der/die zuständigen Entwickler behebt/beheben dann die Fehler
		in der nächsten Iteration.
		Als Testumgebung wird verwendet:
		\begin{itemize}
			\item Coveralls zum Bestimmen der Testabdeckung
			\item Travis CI zum Testen unter Linux
			\item AppVeyor zum Testen für Windowsversionen ab Windows 7
			\item Jenkins zum Testen auf Windows XP
		\end{itemize}
		Um die Installation auf allen Betriebssystemen zu vereinfachen, werden alle benötigten
		Bibliotheken mitgeliefert. Diese werden von den Testumgebungen verwendet, sodass
		auch die Installation auf den verschiedenen Betriebsystemen getestet wird.
		\section{Korrektheit}
		Der Flugsimulator ist ein komplexes System, welches aus mehreren verbundenen Komponenten besteht.
		Sollten zum Beispiel beim Start Fehler auftreten und werden diese nicht korrekt behandelt, so
		ist ein korrekter Start des Simulators nicht möglich. Daher ist es für das Projekt
		sehr wichtig, dass alle Komponenten korrekt spezifiziert und implementiert sind,
		sodass z.B. eventuell auftretende Fehler richtig behandelt werden können.
		\\[5pt]
		Der QS Prozess für dieses Projekt sieht dementsprechend wie folgt aus:
		Der Entwickler entwirft auf Basis der Userstory eine Spezifikation, welche am Ende
		in der Codedokumentation und in Unittests festgehalten wird.
		Nach der Implementierung wird die Userstory von mindestens einem weiteren
		Entwickler geprüft. Dabei wird sichergestellt, dass die Testabdeckung des Projekts
		nicht unter 90\% fällt.
		\\[5pt]
		Zudem wird überprüft, ob die Dokumentation aller implementierten
		Codeteile mit dem zugehörigen Code übereinstimmen und ob die Implementierung der
		Userstory entspricht. Dabei wird die angehängte Checkliste als Grundlage verwendet.
		Sollten Punkte nicht eingehalten werden, wird die Userstory nicht den Auftraggebern zur
		Abnahme vorgestellt. Der/die verantwortliche/n Entwickler beheben dann die Anmerkungen.
		Erst wenn alle Punkte der Checkliste erfüllt sind, wird die Userstory den Auftraggebern
		zur Abnahme empfohlen.
				
		\section{Bedienbarkeit}
		Damit auch weitere, eventuell nicht technisch versierte, Mitarbeiter die Software
		problemlos verwenden können, muss die 
		Bedienoberfläche verständlich und leicht bedienbar sein. Eine Benutzung wird ohne
		vorherige längere Einarbeitung möglich sein.
		\\[5pt]
		Durch die Verwendung des Webframeworks Bootstrap\footnote{\url{https://getbootstrap.com/}}
		wird das Design der Seite vereinheitlicht. Dies wird durch die Checkliste (im Anhang)
		vor der Präsentation einer Userstory für die Auftraggeber durch den Reviewer überprüft. 
		Die Icons\footnote{\url{https://material.io/icons/}}
		stammen aus dem Material Design, welches zur Zeit auf über 79.8\% der Android Geräte
		verwendet wird\footnote{\url{https://developer.android.com/about/dashboards/index.html#Platform}}. 
		Damit ist es einer Vielzahl an Nutzern bereits bekannt.
		\\[5pt]
		Da die Bedienbarkeit nicht objektiv messbar ist, werden zusätzlich Nutzerstudien durchgeführt.
		Diese sind für Anfang Februar und März 2018 angesetzt.
		Die Testkandidaten kommen vorzugsweise auf dem
		Fachbereich/Fachgebiet des Arbeitgebers, aber auch Kandidaten aus anderen
		Fachbereichen werden miteinbezogen.
		\\[5pt]
		Für die Studie erhalten die Nutzer eine Auswahl an Aufgaben (siehe Anhang), 
		welche sie zunächst ohne Anleitung durchführen sollen. 
		Gibt es bei einer Aufgabe Probleme, so wird die Bedienungsanleitung 
		der Software zur Verfügung gestellt.
		Am Ende wird ein Fragebogen ausgehändigt (ebenfalls im Anhang),
		in welchem die Nutzer das Design (z.B. Farbschema, Größe und Beschriftung von Buttons), den
		durch die Software vorgegeben Arbeitsablauf sowie die Anleitung bewerten können. Zudem
		besteht die Möglichkeit allgemeines Feedback abzugeben.
		\\[5pt]
		Mithilfe der ausgefüllten Fragebögen erstellt das Team einen Katalog an
		Maßnahmen bezüglich des Designs, Workflows oder der Anleitung, 
		welche den Auftraggebern beim nächsten Treffen präsentiert werden. 
		Diese entscheiden, welche Maßnahmen umgesetzt werden, woraufhin ein
		Entwickler diese in der nächsten Iteration umsetzt. Die Abnahme erfolgt dann
		wiederum durch die Auftraggeber.

\appendix
	\chapter{Anhang}
		(Am Ende des Projekts nachzureichen)\\
		Belege für durchgeführte Maßnahmen, bzw. falls nicht durchgeführt eine Begründung 
		wieso die Durchführung nicht möglich oder nicht erfolgt ist. \\
		Weitere Anforderungen sind den Unterlagen und der Vorlesung zur Projektbegleitung 
		zu entnehmen.

\end{document}
