\section{Quellcodeauszug}

\definecolor{deepblue}{rgb}{0,0,0.5}
\definecolor{deepred}{rgb}{0.6,0,0}
\definecolor{deepgreen}{rgb}{0,0.5,0}
\definecolor{string-color}{rgb}{0.901, 0.290, 0.098}

\lstset
{
    language=Python,
    basicstyle=\footnotesize,
    otherkeywords={self},             % Add keywords here
    keywordstyle=\color{deepblue},
    commentstyle=\color{string-color},
    emph={MyClass,__init__},          % Custom highlighting
    emphstyle=\color{deepred},    % Custom highlighting style
    stringstyle=\color{deepgreen},
    numbers=left,
    stepnumber=1,
    tabsize=2,
    breaklines=true,
}

Das von uns verwendete Webframework erlaubt die funktionale und objektorientierte Programmierung. Wir haben uns für die funktionale Programmierung
entschieden, weswegen wir drei Module angehängt haben, die repräsentativ für das Projekt stehen.

\subsection{Das Modul Api.py}
%Description
Das Modul Api.py enthält Funktionen, die sich um die Anfragen an die REST API kümmern. 
Da das Webinterface seine Anfragen direkt an die REST API sendet, ist diese Modul ein essentieller
Bestandteil des Backends.
\lstinputlisting[language=Python]{api.py}

\subsection{Das Modul Consumers.py}
\todo[inline]{Beschreibung von consumers.py}
%Description

\lstinputlisting[language=Python]{consumers.py}

\subsection{Das Modul Controller.py}
\todo[inline]{Beschreibung von controller.py}

\lstinputlisting[language=Python]{controller.py}
