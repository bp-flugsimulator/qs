\documentclass[colorback, accentcolor=tud1c, paper=a4, nochapname]{tudexercise}
\usepackage[ngerman]{babel}
\usepackage[utf8]{inputenc}

\title{Protokoll Auftraggebertreffen 06.11.2017}
\subtitle{Frederik Bark, Heiko Carrasco, Jonas Meurer, Tim Weißmantel, Leonardo Zaninelli}
\subsubtitle{\textbf{Protokollant:} Heiko Carrasco\\
\textbf{Anwesende:} Leonardo, Jonas, Tim, Frederik, Heiko, Hendrik, Jonas (Auftraggeber), Torben (Auftraggeber)
}
\author{Heiko Carrasco}


\begin{document}
\maketitle

\section{Aktueller Stand des Projekts}

\begin{itemize}
	\item Momentan gibt es einen KVM-Switch, welcher die einzelnen Systeme ansteuert.
	\item Abhängigkeiten beim Starten zwischen den Programmen sorgen für häufiges Wechseln
	\item Verschiedene Betriebssystemversionen (XP bis 10)

\end{itemize}

\section{Wünsche}
\textbf{Wichtig}
\begin{itemize}
	\item Grafische Nutzeroberfläche
	\item Größtenteils automatisierter Start aller Programme und Rechner
	\item Verschiedene Konfigurationen auswählbar machen
		\begin{itemize}
			\item X-Plane Plugins
			\item Szenerienerstellung
			\item Flugzeugauswahl
		\end{itemize}
	\item Einfaches Hinzufügen neuer Szenarien, Programme und Plugins
	\item Herunterfahren der Anlage
	\item Monitoring von Rechnern und Programmen
\end{itemize}

\textbf{Optional}
\begin{itemize}
	\item X-Plane steuern (API Schnittstelle ansprechen,...)
\end{itemize}

\section{Technische Anforderungen}
\begin{itemize}
	\item Multiplattform Software
	\item Bedienbare Nutzeroberfläche
	\item Windowsspezifische Kommandos ausführen
\end{itemize}
\end{document}
