\documentclass[accentcolor=tud9c,12pt,paper=a4]{tudreport}

\usepackage[utf8]{inputenc}
\usepackage{ngerman}
\usepackage{parcolumns}

\newcommand{\titlerow}[2]{
	\begin{parcolumns}[colwidths={1=.17\linewidth}]{2}
		\colchunk[1]{#1:}
		\colchunk[2]{#2}
	\end{parcolumns}
	\vspace{0.2cm}
}

\title{Projektthema}
\subtitle{Qualitätssicherungsdokument}
\subsubtitle{%
	\titlerow{19}{%
		Frederik Bark <frederikalexander.bark@stud.tu-darmstadt.de>\\
		Heiko Carrasco <heiko.carrascohuertas@stud.tu-darmstadt.de>\\
		Jonas Meurer <jonas.meurer@stud.tu-darmstadt.de>\\
		Tim Weißmantel <tim.weissmantel@stud.tu-darmstadt.de>\\
		Leonardo Zaninelli <leonardo.zaninelli@stud.tu-darmstadt.de>}
	\titlerow{Teamleiter}{Henrik Bode <hendrik.bode@stud.tu-darmstadt.de>}
	\titlerow{Auftraggeber}{%
		Jonas Schulze <Schulze@fsr.tu-darmstadt.de>\\
		Torben Bernatzky <Bernatzky@fsr.tu-darmstadt.de>\\
		Technische Universität Darmstadt\\
		Flugsysteme und Regelungstechnik}
	\titlerow{Abgabedatum}{31.03.2018}}
\institution{Bachelor-Praktikum WS-2017/18\\ Fachbereich Informatik}

\begin{document}

	\maketitle
	\tableofcontents

	\chapter{Einleitung}
		Die Arbeitgeber wünschen sich eine Software die, die Bedienung des
		Flugsimlators vereinfacht. Es wird sich eine Software gewünscht
		in der man Programme auf verschiedenen Rechner von einander Abhängig
		starten kann. Es ist wichtig dass die Software auf verschieden
		Windows Versionen und ggf. auf Linux läuft. Die Software ist in Python
		geschrieben und benutzt Django und RPC um mit den Slaves zu
		kommunizieren. Die Software sollte so generisch wie Möglich sein
		damit in Zukunft andere Software einfach mit aufgenommen werden kann.

	\chapter{Qualitätsziele}
		\section{Portabilität}
			Es ist dem Benutzer möglich die Software auf allen verwendeten
			Geräten bzw. von ihm spezifizierten Geräten (WindowsXP, Windows 7,
			Windows 10, ggf. Linux). Auch läuft die Software auf diesen
			Plattformen wie vom Arbeitgeber abgenommen. Wobei im Vorraus
			geklärt ist, dass der Master nicht unbedingt auf älteren Systemen
			(WindowsXP) laufen muss.

			Um das Ziel zu erreichen werden mehrere Testumgebungen verwendet
			(Travis, Jenkins, AppVeyor), auf denen verschieden Windows
			Versionen und Linux Versionen laufen. Sollte ein Problem auftreten
			wird der zugeteilten Entwickler benachrichtigt. Die Problembehebung
			wird vom jeweiligen zugeteilten Entwickler geändert bzw.
			beaufsichtigt. Dieser kann sich auch Hilfe bzw. Meinungen von
			anderen holen.

			Die Tests werden automatisch bei jeder neuen Änderung ausgeführt.
			Sobald die Änderung nicht lauffähig ist wird diese nicht akzeptiert
			und muss abgeändert werden.
		\section{Bedienbarkeit}
			Die Bedienoberfläche soll leicht verständlich sein und von jedem
			belieben Mitarbeiter bedient werden können.

			Um das Ziel zu erreichen werden Userstudien durchgeführt.
			Diese werden aber nicht in regelmäßigen Abständen durchgeführt,
			sonder nur wenn eine mindest Anzahl von Features bereits
			verfügbar ist. Die Testkandiaten kommen vorzugsweise aus dem
			Bereich des Arbeitgebers. Aber auch Kanidaten aus anderen
			Fachrichtungen werden miteinbezogen.

			Durch die Auswertung der Studie erhält die Gruppe Feedback.
			Falls dem Benutzer die Bedienung nicht möglich ist, muss die
			Oberfläche nach dem Feedback angepasst werden.

\appendix
	\chapter{Anhang}
		(Am Ende des Projekts nachzureichen)\\
		Beleg für durchgeführte Maßnahmen, bzw. falls nicht durchgeführt eine Begründung wieso die Durchführung nicht möglich oder nicht erfolgt ist. \\
		Weitere Anforderungen sind den Unterlagen und der Vorlesung zur Projektbegleitung zu entnehmen.

\end{document}
