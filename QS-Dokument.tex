\documentclass[accentcolor=tud9c,12pt,paper=a4]{tudreport}

\usepackage[utf8]{inputenc}
\usepackage{ngerman}
\usepackage{parcolumns}

\newcommand{\titlerow}[2]{
	\begin{parcolumns}[colwidths={1=.17\linewidth}]{2}
		\colchunk[1]{#1:}
		\colchunk[2]{#2}
	\end{parcolumns}
	\vspace{0.2cm}
}

\title{Projektthema}
\subtitle{Qualitätssicherungsdokument}
\subsubtitle{%
	\titlerow{19}{%
		Frederik Bark <frederikalexander.bark@stud.tu-darmstadt.de>\\
		Heiko Carrasco <heiko.carrascohuertas@stud.tu-darmstadt.de>\\
		Jonas Meurer <jonas.meurer@stud.tu-darmstadt.de>\\
		Tim Weißmantel <tim.weissmantel@stud.tu-darmstadt.de>\\
		Leonardo Zaninelli <leonardo.zaninelli@stud.tu-darmstadt.de>}
	\titlerow{Teamleiter}{Hendrik Bode <hendrik.bode@stud.tu-darmstadt.de>}
	\titlerow{Auftraggeber}{%
		Jonas Schulze <Schulze@fsr.tu-darmstadt.de>\\
		Torben Bernatzky <Bernatzky@fsr.tu-darmstadt.de>\\
		Technische Universität Darmstadt\\
		Flugsysteme und Regelungstechnik}
	\titlerow{Abgabedatum}{31.03.2018}}
\institution{Bachelor-Praktikum WS-2017/18\\ Fachbereich Informatik}

\begin{document}

	\maketitle
	\tableofcontents

	\chapter{Einleitung}
		Die Auftraggeber sind im Besitz eines Flugsimulators, der aus vielen 
		Teilprogrammen besteht. Diese Teilprogramme sind über mehrere Systeme im
		gleichen Netzwerk verteilt und müssen alle in einer komplexen Reihenfolge per Hand 
		gestartet werden. Aus diesem Grund ist eine Software gewünscht, die das Starten des gesamten
		Simulators automatisiert.\\[5pt]
		Es ist dabei zu beachten, dass auf den Teilsystemen unterschiedliche 
		Betriebsysteme benutzt werden (verschiedene Windowsversionen und ggf. Linux). Ebenfalls 
		soll ein generischer Ansatz gewählt werden, um das Einfügen einer neuen Startreihenfolge im Nachhinein zu vereinfachen.\\[5pt]
		Hierzu wird ein Master/Slave-System eingesetzt, wobei über einen Rechner alle 
		Systeme, die Teil des Simulators sind, gesteuert werden. Ein Nutzer kann dann über
		ein Webinterface diesen Master konfigurieren und steuern.\\[5pt]
		Die Software für Master und Slaves ist in Python geschrieben, und benutzt
		das Django Framework für das Webinterface, sowie RPC zur Kommunikation zwischen 
		Master und Slaves.
		

	\chapter{Qualitätsziele}
		\section{Portabilität}
		Wie in der Einleitung erwähnt, muss die Software auf vielen verschiedenen 
		Betriebsystemen laufen. (Windows XP, Windows 7, Windows 10 und ggf. Linux)
		Dabei wurde im Vorraus besprochen, dass die Software für den Master nicht 
		unbedingt auf älteren Systemen laufen muss (Windows XP).
		\\[5pt]
		Um gute Portabilität zu erreichen, setzen wir auf gute Testabdeckung 
			(mindestens 90\% für nicht generierte Dateien) und eine umfangreiche 
			Testumgebung. Dazu verwenden wir:
			\begin{itemize}	
				\item Coveralls zum Bestimmen der Testabdeckung
				\item Travis CI zum Testen unter Linux
				\item AppVeyor zum Testen für Windowsversionen ab Windows 7
				\item Jenkins zum Testen auf Windows XP
			\end{itemize}
			Alle Tools sind in das Githubprojekt integriert und werden nach jedem Push auf 
			Origin ausgeführt.
			\\[5pt]
			Die Userstorys werden in Github über Pullrequests gehandhabt. Wenn der/die 
			Entwickler der Meinung ist/sind, dass das Akzeptanzkriterium erfüllt ist, geben sie 
			den Pullrequest zum Review frei. Ein Review muss von mindestens einem anderen Entwickler erstellt werden. Er arbeitet dabei folgende Liste ab und fügt sie als Kommentar im Pullrequest ein:
			\begin{itemize}	
				\item Test Coverage (auf unseren Dateien) mindestens 90\%
				\item Jede Klasse und Funktion ist grob Dokumentiert
				\item Schwierige Codestellen sind dokumentiert
				\item Alle Test laufen fehlerfrei durch
				\item Der Code ist korrekt formatiert
				\item Der Code erfüllt das Akzeptanzkriterium der Userstory (und dabei spezifisch nur das Akzeptanzkriterium dieser Userstory)
				\item Die Userstory ist ausgefüllt (Datum, Zeit)
			\end{itemize}
			Falls Teile der Liste nicht erfüllt werden, muss/müssen der/die Entwickler 
			der Userstory die Probleme beheben. Danach muss erneut ein Review 
			durchgeführt werden. Erst danach wird die Userstory in den Master Rebased/Gemerged.
				
		\section{Bedienbarkeit}
		Die Bedienoberfläche soll leicht verständlich sein und jeder
		Mitarbeiter soll sie bedienen können.
			\\[5pt]
			Um diese Ziel zu erreichen, werden Nutzerstudien durchgeführt.
			Diese werden nicht in regelmäßigen Abständen durchgeführt,
			sonder nur wenn eine mindest Anzahl von Features bereits
			verfügbar ist. Die Testkandidaten kommen vorzugsweise aus dem
			Fachbereich/Fachgebiet des Arbeitgebers. Aber auch Kandidaten aus anderen
			Fachgebieten sollen miteinbezogen werden.
			\\[5pt]
			Durch die Auswertung der Studie erhält die Gruppe Feedback.
			Falls die Benutzer Probleme bei der Bedienung der Oberfläche haben, muss 
			diese nach dem Feedback angepasst werden.
			\\[5pt]
			Des Weiteren wird eine Anleitung geschrieben, die an die Anwender der Software
			gerichtet ist.

\appendix
	\chapter{Anhang}
		(Am Ende des Projekts nachzureichen)\\
		Belege für durchgeführte Maßnahmen, bzw. falls nicht durchgeführt eine Begründung wieso die Durchführung nicht möglich oder nicht erfolgt ist. \\
		Weitere Anforderungen sind den Unterlagen und der Vorlesung zur Projektbegleitung zu entnehmen.

\end{document}
